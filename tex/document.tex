\documentclass[12pt]{article}
\usepackage[a4paper, left=3.17cm, right=3.17cm, top=2.54cm, bottom=2.54cm]{geometry}
\usepackage[fontset=mac]{ctex}
\usepackage[T1]{fontenc}
\usepackage{mathptmx}
\usepackage{amsmath}
\usepackage{amsfonts}
\usepackage{chemformula}
\usepackage{cite}
\usepackage[colorlinks, linkcolor=black, anchorcolor=black, citecolor=black]{hyperref}
\usepackage{indentfirst}

\usepackage{graphicx}
\setlength{\parskip}{0.5em}
\title{《高性能计算引论》第一次作业}
\author{\textup{罗文水}}
\begin{document}
	
	\begin{titlepage}
		\newcommand{\HRule}{\rule{\linewidth}{0.5mm}}
		\includegraphics[width=8cm]{title}\\[1cm] 
		\center 
		\quad\\[1.5cm]
		\textsl{\Large \textbf{Nanjing University of Science and Technology} }\\[0.5cm] 
		\textsl{\large School of Computer Science and Engineering}\\[0.5cm] 
		\makeatletter
		\HRule \\[0.4cm]
		{ \huge \bfseries \@title}\\[0.25cm] 
		\HRule \\[1.5cm]
	\begin{minipage}{0.42\textwidth}
		\begin{flushleft}
			
			\Large{\emph{姓名:罗文水}}
			\\
			\Large{\emph{学号:918106840738}}
			\\
			\Large{\emph{班级:计科一班}}
			\\
			\Large{\emph{课程:高性能计算引论}}
			\\
			\Large{\emph{授课教师:李翔宇}}
			\\
		\end{flushleft}
	\end{minipage}
		\vspace{7em} 
		
		{\large \today}\\[2cm] 
		\vfill 
	\end{titlepage}
	
	\newpage
\section{问题一}
	(1) 1、核心数目:\\

	(2)最\\
	
\section{问题二}

	首先,假设消息传递时间为$t_1$,则用于计算和消息传递的总时间计算方式如下:
\begin{eqnarray}\label{px}
	T_{total}&=&T_{message}+T_{compute}\\ \notag
	&=&
	\frac{10^{6}}{p}+10^{9}\cdot(p-1)\cdot t_1 \quad sec
\end{eqnarray}

	(1)将$t_1=10^{-9}\;p=1000$带入上式得$T_{total}=1999\;sec$\\
	
	(2)将$t_1=10^{-3}\;p=1000$带入上式得$T_{total}=999001\cdot10^{3}\;sec$\\
	
	(3)根据问题一和问题二,容易知道问题一中消息传递的时间占总体时间的比重约为$
	50\%$,而问题二中消息传递时间几乎占据了总时间的$100\%$。从中可以得到结论:在多处理机环境下,消息传递的时间是不容忽视的部分,在消息传递时间非常小的情况下,也可能占据任务总体时间花费的$50\%$,与计算时间相当。而当消息传递时间较大时(本题中为0.001sec),则会严重增加任务的绝对执行时间,并且执行时间中的绝大部分都用在了消息传递上。所以,在考虑多处理机计算环境时,消息传递机制与策略是至关重要的部分。处理机之间的消息传递(计算结果之间的相互访问)若是以互联网络方式实现则需要充分考虑到节点之间访问的逻辑结构与开销,若是在共享存储器的方式下,可以访问公有的存储体从而开销降低。
\end{document}
