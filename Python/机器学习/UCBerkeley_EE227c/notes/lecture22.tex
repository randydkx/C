\section{Non-convex constraints II: projected gradient descent} 


Last time we discussed optimization of a convex function $f$ over non-convex sets. A typical example for such a problem is $\min \| Ax-y\|_2^2$ subject to $\|x\|_0 \leq s$. Another type of non-convex optimization is to minimize a non-convex function over a convex set. A typical example for this kind of non-convex optimization is known as $l_0$-minimization where $\min \| x\|_0$ subject to an affine convex constraint like $Ax=y$. In sum, the $l_0$-minimization is: 

\begin{align*}
\min_{Ax=y} \|x\|_0
\end{align*}

One option for solving the above non-convex optimization problem is to relax the $l_0$-objective to the convex $l_1$-objective as:

\begin{align*}
\min_{Ax=y} \|x\|_1
\end{align*}


% A variation of the above problem is Basis Pursuit Denoising (BPDN),

%\begin{align}
%\min \|x\|_1 \text{ s.t. } \|Ax - y \|_2 \leq \epsilon.
%\end{align}


Note that under appropriate assumptions on $A$, $x$ and $y$, we observed that $\ell_1$-minimization still gives the right answer (assumptions such as RIP and the restricted nullspace property).

% Sparse linear regression has received a lot of attention and there are many related optimization programs like:
% BPDN (allows for noise)
%\begin{align*}
%\min_{\|Ax-y\|_2 \leq s} \|x\|_1
%\end{align*}

%Constrained Lasso
%\begin{align*}
%\min_{\|x\|_1 \leq \lambda }\|Ax-y\|_2^2 
%\end{align*}

%Lagrangian Lasso
%\begin{align*}
%\min \|Ax-y\|_2^2 +  \lambda \|x\|_1
%\end{align*}

In our setting, we have \\

\centering{nonconvex constraint} 
\begin{align*}
\downarrow
\end{align*}
\centering{cvx relaxation }
\begin{align*}
\downarrow
\end{align*}
\centering{PGD (Projected Gradient Descent)},

or more directly, nonconvex constraint $\rightarrow$ PGD.  


We discuss some takeaways from the jupyter notebook. 
While convex relaxations can be solved efficiently (e.g., using interior point methods), scaling to large problem instances is an issue. Hence it makes sense to consider first order methods such as projected gradient descent (PGD) to speed up computation. We find that it works directly with projections onto the non-convex set. And when we're running PGD, it's natural to ask whether we need the convex relaxation in the first place or can just directly run PGD for the non-convex set.

Consider the test problem of form $y = Ax$ with $s$-sparse $x$ where $A$ has dimension $n \times d$ and we sample the entries of $A$ as i.i.d. Gaussians so the matrix satisfies the restricted isometry property if we take enough samples $n$. 
We can check how many samples are needed for $\ell_1$-relaxation to work. We have that an i.i.d. Gaussian matrix requires $O(s \log d/s)$ rows to satisfy RIP, so this corresponds to $n = O(s \log d/s)$ samples. 
The running time of using just the interior point methods is somewhat slow. 
\begin{itemize}
\item $d=100, n=50$: 10 ms, 
\item $d=1000, n=500$: 5-6 seconds,
\item $d=4000, n=2000$: 112 seconds. 
\end{itemize}

Therefore, to solve very large instances, we should also consider first-order methods. We may directly run projected gradient descent with the non-convex set of sparse vectors, also known as Iterative Hard Thresholding since the projection step (to find the closest $s$-sparse vector) corresponds to hard thresholding the vector (keep only the $s$ largest entries and set the rest to 0). It is 1000 times faster as it took $0.0357$ seconds.  




Now, we discuss the Iterative Hard Thresholding (IHT) which is known as a projected gradient descent (PGD) for sparse vectors.

Iterative Hard Thresholding (IHT) is used with Gradient Descent for the following problem.


Given the setup:
\begin{align}
y = A x + e 
\end{align}

where $y \in \R^M, x \in \R^N, A \in \R^{M \times N}$, $e$ is observation noise, the goal is to estimate $x$ given $y$ and $A$ when $M <<N$ and $x$ is approximately $K$-sparse.

The IHT algorithm uses the iteration
\begin{align}
x^{n+1} = P_K ( x^n + A^{\trans} (y - A x^n))
\end{align}

where $P_K$ is a hard thresholding operator that keeps the largest $K$ elements of a vector.

Consider the following objective function.

\begin{align}
f(x) &= \frac{1}{2} \| Ax - y \|_2^2 \\
\nabla f(x) &= A^{\trans}(Ax - y)
\end{align}

If it is possible to directly optimize over nonconvex sets, then we don't need convex relaxation.

We want to show that the algorithm IHT outputs a solution. The algorithm is defined as:

\begin{align*}
\text{IHT}(y, A, t, s) \\
x^1 &\leftarrow 0 \\
\text{ for } &i=1 ,\cdots, t, \\
\tilde{x}^{i+1} &\leftarrow x^i - A^{\trans} (Ax - y) \\
x^{i+1} &\leftarrow P_S( \tilde{x}^{i+1} ) \\
\text{ return }& \hat{x} \leftarrow x^{t+1}
\end{align*}

Note that $P_s$ is a projection onto a set of s-sparse vectors. In the rest of the lecture we will see how this projection works and how fast the method is.


We study a nice property of matrices, the Restricted Isometry Property. It is useful since it implies that the difference between two $s$-sparse vectors cannot be mapped to 0, and the RIP also implies the restricted nullspace property.
Here, the Restricted Isometry Property allows optimization over nonconvex sets.

\begin{definition}
RIP: matrix $A$ satisfies the $(s,\delta)$-RIP if for all $s$-sparse vectors we have:

\begin{align*}
(1-\delta) \|x\|_2^2 \leq \|Ax\|_2^2 \leq (1+\delta)\|x\|_2^2
\end{align*}

\end{definition}

Consequences: For all supports $S$ of size $s$, we have:
\begin{align*}
\|I-X_S^{\trans} X_S\|_2 \leq \delta
\end{align*}
where $X_S$ is $X$ restricted to $S$ (for detailed description please see the notes for lecture 21). \\

Consider the following setup: \\
\begin{itemize}
\item $y = Ax^* + e$ \\
\item $x^*$ is $s$-sparse with support $S^*$ \\ 
\item $A$ has ($3s,\frac{1}{4})$-RIP \\
\item $e$ is arbitrary noise.
\end{itemize}
Then the following holds,
\begin{theorem}
$$\|x^{i+1}-x^*\|_2 \leq \frac{1}{2} \|x^{i}-x^*\|_2 + \max_{|S| \leq 3s} \|A_S^{\trans} e\|_2$$
\end{theorem}

\begin{proof}
Let $S_i$ be the support of  $(x^i)$ and let $ S\prime = S^{i+1} \cup S^i \cup S^*$ (so $|S \prime | \leq 3s$) . Then:

\begin{align*}
\|x^{i+1}-x^*\|_2 &\leq \|x^{i+1} - \tilde x_{S\prime }^{i+1}\|_2 + \|\tilde x_{S\prime}^{i+1} -x^*\|_2\  \tag{ by triangle inequality}\\
			 &\leq 2 \|\tilde x_{ S\prime}^{i+1} -x^*\|_2\\ 
			 &= 2 \|x_{ S\prime}^i - A_ { S\prime}^{\trans}(Ax^i -y)-x^*\|\\
			 &= 2 \|x^i - A_ {S\prime}^{\trans}(A_ { S \prime}x^i -A_ { S\prime}x^*-e)-x^*\|\\
			 &= 2 \|x^i - x^*- A_ { S\prime}^{\trans}A_ { S\prime} (x^i-x^*) + A_ {S \prime}^{\trans} e\|\\
			 &\leq 2\| I -  A_ {S \prime}^{\trans} A_ {S \prime} (x^i-x^*)\|_2 +   2\| A_ {S \prime}^{\trans} e\|_2 \\
			 &\leq 2 \delta \| (x^i-x^*)\|_2 +   2 \max_{|S| \leq 3s}\| A_ {S}^{\trans} e\|_2
\end{align*}
\end{proof}

The second inequality follows from the first inequality because $x^{i+1}$ is the $s$-sparse projection of $\tilde x^{i+1}$. In particular, this also means that $x^{i+1}$ is the best $s$-sparse approximation of $\tilde x_{S\prime}^{i+1}$ and hence we have:

$$\| x^{i+1} - x_{S\prime }^{i+1} \| \leq \| \tilde x_{S\prime}^{i+1} -x^*\|_2$$ 

In the theorem above, we have shown that the error goes down by a factor $1/2$ in every iteration (up to the noise threshold). From there, it is fairly straightforward to get a linear convergence rate as in the following corollary:
\begin{corollary}
$\| \hat x - x^*\|_2 \leq (\frac{1}{2})^t \|x^* \|_2 + 5\|e\|_2$
so $t = \log \frac{\| x^*\|_2}{\epsilon}$ iterations suffice for $\| \hat x - x^*\|_2 \leq \epsilon + 5\|e\|_2$.
\end{corollary}
 We have a linear rate and PSG, but the analysis looks somewhat different (no convexity/smoothness) and we did not need a step size.
 
Now consider function $f =  \frac{1}{2}\|Ax-y \|_2^2$ and hence $\nabla f(x) = A^{\trans}(Ax-y)$.

\begin{definition}
Smoothness:
\begin{align*}
f(x+\Delta) \leq f(x) + \langle \nabla f(x), \Delta \rangle + \frac{L}{2} \| \Delta\|_2^2 
\end{align*}

which is valid for all $\Delta \in \R^d$.
\end{definition}

What does the smoothness mean for the above function $f$?
\begin{align*}
\frac{1}{2}\|A(x+\Delta) - y\|_2^2 \leq \|Ax-y\|_2^2 + \Delta^{\trans} A^{\trans}(Ax-y) + \frac{L}{2}\|\delta\|_2^2
\end{align*}

then:
\begin{align*}
\frac{1}{2}(x+\Delta)^{\trans} A^{\trans}A(x+\Delta) + \frac{1}{2}y^{\trans} y \leq \frac{1}{2} x^{\trans} A^{\trans} Ax-y^{\trans} Ax + \frac{1}{2}y^{\trans} y + \Delta^{\trans} A^{\trans} Ax- \Delta^{\trans} A^{\trans} y + \frac{L}{2} \|\Delta\|_2^2 \end{align*}

hence:
\begin{align*}
\frac{1}{2}\Delta^{\trans} A^{\trans} A \Delta \leq \frac{L}{2}\|\Delta\|_2^2,
\end{align*}

and as a result:

\begin{align*}
\|A\Delta \|_2^2 \leq L\|\Delta\|_2^2
\end{align*}

Taking $L= 1+ \delta$ gives that the above equals 
\begin{align*}
L/ \ell \approx \frac{1+ \delta}{1- \delta} \approx 1.
\end{align*}

Note that the above inequality relates being smooth with following the RIP, just replace the condition "for any $\triangle$" (for smoothness) by "for any $s$-sparse $\triangle$" (for RIP).
We have similar results for strong convexity.
\begin{definition}
Strong Convexity:
\begin{align*}
f(x+\Delta) \geq f(x) + \langle \nabla f(x), \Delta \rangle + \frac{l}{2} \| \Delta\|_2^2 
\end{align*}

which is valid for all $\Delta \in \R^d$.
\end{definition}
We can easily conclude that strong convexity is equivalent to $\|A\Delta \|_2^2 \geq l\|\Delta\|_2^2$.

\begin{definition}
Restricted Strong Convexity (RSC):
\begin{align*}
f(x+\Delta) \geq f(x) + \langle \nabla f(x), \Delta \rangle + \frac{l}{2} \| \Delta\|_2^2 
\end{align*}
which is valid for all $s$-sparse $\Delta$.
\end{definition}


We can say that:
\begin{center}
RIP = RSC + RSM, 
\end{center}
with very good ($ L \approx 1 + \delta$) condition number $L/\ell$ (hence constant step size $\approx 1/L$).
There is a lot of work on weakening this assumption and further constraints sets.

Convex relaxation involves mostly optimal dependence on condition number.
PGD can be made to work for arbitrary condition number but with a worse statistical rate.
Can we match convex relaxation with non-convex PGD? Yes!

Given the sparsity condition, it is possible to do hard thresholding in $O(d)$ time. Given the low-rank condition, one can compute the SVD and find the largest singular values and for a $d_1 \times d_2$ matrix, in $O(d_1 d_2 \min( d_1, d_2))$ time.


\begin{definition}
\emph{Group Sparsity:}
We are given a family of groups $G_i \subseteq [d]$, support $\text{supp}(x^*) = \cup_{j \in J} G_j$ for some $|J| \leq g.$ This is NP-hard via set cover. 
\end{definition}

\begin{definition}
\emph{Graph Sparsity:}
Given graph $G$ defined on $[d]$ (nodes are indices in $\{ 1, \cdots, d\}$), $\text{supp}(x^*)$ is a connected subgraph in $G$. Here, projection on set is NP-hard (Steiner trees). 
\end{definition}

\begin{definition}
\emph{Approximate Projection}:
(Tail approximation) Given input $\tilde{x} \in \mathbb{R}^d$, find $x \in C$ ($C$ is the constraint set) such that 
\begin{align*}
\|x - \tilde{x} \|_2 \leq \alpha \min_{x' \in C} \| x'- \tilde{x} \|_2. 
\end{align*}
\end{definition}

For $x^*$ that is $1$-sparse and the problem
\begin{align*}
y = Ax^*,
\end{align*}

$A = \pm \frac{1}{\sqrt{n}}$ gives $n= O(\log d)$ and then we have RIP, and PGD should work. 

In the first iteration of PGD, 
\begin{center}
$x' \leftarrow 0$ \\
$x^2 \leftarrow \hat{P}_S(A^{\trans} y)$
\end{center}

where $A^{\trans} y = A^{\trans} A x^* = b$.

Consider the following setup. 

\begin{align*}
b_1 = 1 \\
\mathbb{E}[b_i^2] = \frac{1}{n} \\
\mathbb{E}[ \|b \|_2^2] = 1 + \frac{d-1}{n}.
\end{align*}


The approximate error of the best projection is on the order $\frac{d-1}{n}$.

\begin{definition}
Head Projection:

Given $\tilde{x} \in \R^d,$ find a support $S$ such that
\begin{align}
\| \tilde{x}_S \|_2 \geq \beta \max_{S' \in \text{Supp}(C)} \| \tilde{x}_{S'} \|_2.
\end{align}
\end{definition}

When this is combined with approximate projection, PGD still works.

