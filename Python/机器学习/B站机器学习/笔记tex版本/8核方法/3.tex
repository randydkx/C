\documentclass[a4paper]{article}
\usepackage[UTF8]{ctex}
\usepackage{geometry}
\usepackage{graphicx}
\usepackage{url}
\usepackage{multirow}
\usepackage{array}
\usepackage{booktabs}
\usepackage{url}
\usepackage{enumitem}
\usepackage{graphicx}
\usepackage{float}
\usepackage{amssymb}
\usepackage{amsmath}
\usepackage{subfig}
\usepackage{longtable}
\usepackage{pifont}
\usepackage{color}
\usepackage{listings}
\usepackage{xcolor}

\allowdisplaybreaks

\geometry{a4paper, scale=0.78}

% \begin{figure}[H]
%     \centering
%     \includegraphics[width=.55\textwidth]{E.png}
%     \caption{矩阵与列向量的乘法}
%     \label{fig:my_label_1}
% \end{figure}

% \left\{
% \begin{array}{ll}
%       x+2x+z=2 & \\
%       3x+8y+z=12 & \\
%       4y+z=2
% \end{array}
% \right.

% \begin{enumerate}[itemindent = 1em, itemsep = 0.4pt, parsep=0.5pt, topsep = 0.5pt]

% \end{enumerate}

%\stackrel{a}{\longrightarrow}

\title{Kernel Method 03 Necessary and Sufficient Conditions}
\author{Chen Gong}
\date{22 November 2019}

\begin{document}
\maketitle
在上一小节中,我们描述了正定核的两个定义,并且认为这两个定义之间是相互等价的。下面我们就要证明他们之间的等价性。
\section{充分性证明}
大家注意到在上一节的描述中,我似乎没有谈到对称性,实际上是因为对称性的证明比较的简单。就没有做过多的解释,那么我重新描述一下我们需要证明的问题。

已知:$K(x,z) = <\phi(x),\phi(z)>$,证:Gram Matrix是半正定的,且$K(x,z)$是对称矩阵。

对称性:已知:
\begin{equation}
    K(x,z)=<\phi(x),\phi(z)> \qquad K(z,x) = <\phi(z),\phi(x)>
\end{equation}

又因为,内积运算具有对称性,所以可以得到:
\begin{equation}
    <\phi(x),\phi(z)> = <\phi(z),\phi(x)>
\end{equation}

所以,我们很容易得到:$K(x,z)=K(z,x)$,所以对称性得证。

~\\

正定性:我们想要证的是Gram Matrix$=K[K(x_i,x_j)]_{N\times N}$是半正定的。那么,对一个矩阵$A_{N\times N}$,我们如何判断这是一个半正定矩阵?大概有两种方法,1. 这个矩阵的所有特征值大于等于0;2. 对于$\forall \alpha \in \mathbb{R}^N,\ \alpha^T A \alpha \geq 0$。这个是充分必要条件。那么,这个问题上我们要使用的方法就是,对于$\forall \alpha \in \mathbb{R}^N,\ \alpha^T A \alpha \geq 0$。
\begin{align}
    \alpha^TK\alpha = & 
    \begin{bmatrix}
        \alpha_1 & \alpha_2 & \cdots & \alpha_N
    \end{bmatrix}
    \begin{bmatrix}
        K_{11} & K_{12} & \cdots & K_{1N} \\
        K_{21} & K_{22} & \cdots & K_{2N} \\
        \vdots & \vdots & \ddots & \vdots \\
        K_{N1} & K_{N2} & \cdots & K_{NN} \\
    \end{bmatrix}
    \begin{bmatrix}
        \alpha_1 \\
        \alpha_2 \\ 
        \vdots \\ 
        \alpha_N
    \end{bmatrix} \\
    = & \sum_{i=1}^N\sum_{j=1}^N \alpha_i\alpha_jK_{ij} \\
    = & \sum_{i=1}^N\sum_{j=1}^N \alpha_i\alpha_j<\phi(x_i),\phi(x_j)> \\
    = & \sum_{i=1}^N\sum_{j=1}^N \alpha_i\alpha_j\phi(x_i)^T\phi(x_j) \\ 
    = & \sum_{i=1}^N\phi(x_i)^T\sum_{j=1}^N\phi(x_j) \\
    = & \left[  \sum_{i=1}^N\phi(x_i) \right]^T \left[  \sum_{j=1}^N\phi(x_j) \right] \\
    = & \left|\left| \sum_{i=1}^N \alpha_i\phi(x_i) \right|\right|^2 \geq 0
\end{align}

所以,我们可以得到$K$是半正定的,必要性得证。

\section{必要性证明}
充分性得到证明之后,必要性的证明就会变得很简单了。这个证明可以被我们描述为:
 
已知:Gram Matrix是半正定的,且$K(x,z)$是对称矩阵。证:存在一个映射$\phi:\mathcal{X}\mapsto\mathbb{R}^p$,使得$K(x,z) = <\phi(x),\phi(z)>$。

对于我们建立的一个映射$\phi(x)= K(x,\cdot)$,我们可以得到$K(x,\cdot)K(z,\cdot) = K(x,z)$。所以有$K(x,z) = K(x,\cdot)K(z,\cdot) = \phi(x)\phi(z)$。我们就得证了,具体的理解可以参考我之前写的关于可再生核希尔伯特空间的理解。

另外一种证明方法:
对K进行特征值分解,$K=V \Lambda V^T$,那么令$\phi(x_i)=\sqrt{\lambda_i}V_i$,于是构造了$K(x_i,x_j)=\sqrt{\lambda_i\lambda_j}V_iV_j$。












\end{document}
