\documentclass[a4paper]{article}
\usepackage[UTF8]{ctex}
\usepackage{geometry}
\usepackage{graphicx}
\usepackage{url}
\usepackage{multirow}
\usepackage{array}
\usepackage{booktabs}
\usepackage{url}
\usepackage{enumitem}
\usepackage{graphicx}
\usepackage{float}
\usepackage{amssymb}
\usepackage{amsmath}
\usepackage{subfig}
\usepackage{longtable}
\usepackage{pifont}
\usepackage{color}

\allowdisplaybreaks

\geometry{a4paper, scale=0.78}

% \begin{figure}[H]
%     \centering
%     \includegraphics[width=.55\textwidth]{E.png}
%     \caption{矩阵与列向量的乘法}
%     \label{fig:my_label_1}
% \end{figure}

% \left\{
% \begin{array}{ll}
%       x+2x+z=2 & \\
%       3x+8y+z=12 & \\
%       4y+z=2
% \end{array}
% \right.

% \begin{enumerate}[itemindent = 1em, itemsep = 0.4pt, parsep=0.5pt, topsep = 0.5pt]

% \end{enumerate}

\title{Bayes Linear Classification 03 Prediction \& Conclusion}
\author{Chen Gong}
\date{06 November 2019}

\begin{document}
\maketitle
根据上一节中提到的Inference,我们已经成功的推断出了$p(w|Data)$的分布。表述如下所示:
\begin{equation}
    p(w|X,Y) \sim \mathcal{N}(\mu_w, \Sigma_w)
\end{equation}

其中,
\begin{equation}
    \Sigma_w^{-1}=\sigma^{-2}X^TX+\Sigma_p^{-1} \qquad \mu_w = \sigma^{-2}A^{-1}X^TY \qquad \Sigma_w^{-1}=A
\end{equation}

而我们的Prediction过程,可以被描述为,给定一个$x^\ast$如果计算得到$y^\ast$。而我们的模型建立如下所示:
\begin{equation}
\left\{
\begin{array}{ll}
      f(x)=w^Tx = x^Tw & \\
      y = f(x) + \varepsilon & \varepsilon \sim \mathcal{N}(0,\sigma^2)
\end{array}
\right.    
\end{equation}

\section{Prediction}
模型预测的第一步为,
\begin{equation}
    f(x^\ast) = {x^\ast}^T w 
\end{equation}

而在Inference部分,我们得到了$p(w|Data)= \mathcal{N}(\mu_w,\Sigma_w)$。所以,我们可以推断出,
\begin{equation}
    f(x^\ast) = {x^\ast}^T w \sim \mathcal{N}({x^\ast}^T\mu_w, {x^\ast}^T\Sigma_w{x^\ast})
\end{equation}

那么公式(5)我们可以写作:
\begin{equation}
    p(f(x^\ast)|Data,x^\ast) \sim \mathcal{N}({x^\ast}^T\mu_w, {x^\ast}^T\Sigma_w{x^\ast})
\end{equation}

又因为$y = f(x) + \varepsilon$,所以
\begin{equation}
    p(y^\ast|Data,x^\ast) \sim \mathcal{N}({x^\ast}^T\mu_w, {x^\ast}^T\Sigma_w{x^\ast}+\sigma^2)
\end{equation}

那么计算到这里,我们的模型预测也算是完成了。

\section{Conclusion}
Data:$D=\{(x_i,y_i)\}^{N}_{i=1}$,其中$x_i\in\mathbb{R}^{p}$,$y_i\in\mathbb{R}$。

Model:
\begin{equation}
\left\{
\begin{array}{ll}
      f(x)=w^TX = x^Tw & \\
      y = f(x) + \varepsilon & \varepsilon \sim \mathcal{N}(0,\sigma^2)
\end{array}
\right.    
\end{equation}

Bayesian Method:$w$不在是一个未知的常数,$w$而是一个概率分布。
贝叶斯线性分类可以被分成两个部分,Inference和Prediction。

1. Inference:$p(w|Data)$是一个posterior分布,假定$p(w|Data)=\mathcal{N}(\mu_w, \Sigma_w) \propto likelihood \times prior$。这里使用了共轭的小技巧,得到posterior一定是一个Gaussian Distribution。在这一步中,我们的关键是求出$\mu_w$和$\Sigma_w$。

2. Prediction:这个问题实际上也就是,给定一个$x^\ast$如果计算得到$y^\ast$。我们可以描述为:
\begin{equation}
    p(y^\ast|Data,x^\ast) = \int_w p(y^\ast|w,Data,x^\ast)p(w|Data,x^\ast) dw 
\end{equation}

又因为,$y^\ast$只依赖于$w$和$x^\ast$,不依赖于历史数据,所以$p(y^\ast|w,Data,x^\ast)=p(y^\ast|w,x^\ast)$。并且,$w$的获得与$x^\ast$没有关系,所以$p(w|Data)$。所以:
\begin{equation}
    p(y^\ast|Data,x^\ast) = \int_w p(y^\ast|w,x^\ast)p(w|Data) dw = \mathbb{E}_{w\sim p(w|Data)}[p(y^\ast|w,x^\ast)] 
\end{equation}

之后通过自共轭特性不用计算积分即可得到服从的正态分布。

\end{document}
