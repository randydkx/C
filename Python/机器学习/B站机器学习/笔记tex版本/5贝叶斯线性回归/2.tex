\documentclass[a4paper]{article}
\usepackage[UTF8]{ctex}
\usepackage{geometry}
\usepackage{graphicx}
\usepackage{url}
\usepackage{multirow}
\usepackage{array}
\usepackage{booktabs}
\usepackage{url}
\usepackage{enumitem}
\usepackage{graphicx}
\usepackage{float}
\usepackage{amssymb}
\usepackage{amsmath}
\usepackage{subfig}
\usepackage{longtable}
\usepackage{pifont}
\usepackage{color}

\allowdisplaybreaks

\geometry{a4paper, scale=0.78}

% \begin{figure}[H]
%     \centering
%     \includegraphics[width=.55\textwidth]{E.png}
%     \caption{矩阵与列向量的乘法}
%     \label{fig:my_label_1}
% \end{figure}

% \left\{
% \begin{array}{ll}
%       x+2x+z=2 & \\
%       3x+8y+z=12 & \\
%       4y+z=2
% \end{array}
% \right.

% \begin{enumerate}[itemindent = 1em, itemsep = 0.4pt, parsep=0.5pt, topsep = 0.5pt]

% \end{enumerate}

\title{Bayes Linear Classification 02 Inference}
\author{Chen Gong}
\date{05 November 2019}

\begin{document}

\maketitle

数据集$D=\{(x_i,y_i)\}^{N}_{i=1}$,其中$x_i\in\mathbb{R}^{p}$,$y_i\in\mathbb{R}$。数据矩阵为:(这样可以保证每一行为一个数据点)

\begin{equation}
    X=(x_1, x_2, \cdots, x_N)^T=
    \begin{pmatrix}
    x_1^T \\ 
    x_2^T \\
    \vdots\\
    x_N^T \\
    \end{pmatrix} =
    \begin{pmatrix}
    x_{11} & x_{12} & \dots & x_{1p}\\
    x_{21} & x_{32} & \dots & x_{2p}\\
    \vdots & \vdots & \ddots & \vdots\\
    x_{N1} & x_{N2} & \dots & x_{Np}\\
    \end{pmatrix}_{N\times P}
\end{equation}
\begin{equation}
    Y=
    \begin{pmatrix}
    y_1 \\ 
    y_2 \\
    \vdots\\
    y_N \\
    \end{pmatrix}_{N\times 1}
\end{equation}

拟合函数我们假设为:$f(x) = w^Tx = x^Tw$。

预测值$y=f(x)+\varepsilon$,其中$\varepsilon$是一个Guassian Noise,并且$\varepsilon \sim \mathcal{N}(0,\sigma^2)$。

并且,$x,y,\varepsilon$都是Random variable。

贝叶斯估计方法(Bayesian Method),可以分为两个步骤,1.Inference,2.Prediction。Inference的关键在于估计posterior$(w)$;而Prediction的关键在于对于给定的$x^{\ast}$求出预测值$y^{\ast}$。

\section{Bayesian Method模型建立}
首先我们需要对公式使用贝叶斯公式进行分解,便于计算:
\begin{equation}
    p(w|Data) = p(w|X,Y) = \frac{p(w,Y|X)}{p(Y|X)} = \frac{p(Y|X,w)p(w)}{\int_w p(Y|X,w)p(w)dw}
\end{equation}

其中$p(Y|X,w)$是似然函数(likelihood function),$p(w)$是一个先验函数(prior function)。实际这里省略了一个过程,$p(w,Y|X)=p(Y|X,w)p(w|X)$。但是很显然,$p(w|X)$中$X$与$w$之间并没有直接的联系(也就是说每个数据样本中的$x$都是从数据总体分布$p(x)$中抽样得到的,与先验分布无关)。所以$p(w|X)=p(w)$。

似然函数的求解过程为:
\begin{equation}
    p(Y|X,w) = \prod_{i=1}^N p(y_i|x_i,w) 
\end{equation}

又因为$y=w^Tx+\varepsilon$,并且$\varepsilon \sim \mathcal{N}(0,\sigma^2)$。所以
\begin{equation}
    p(y_i|x_i,w) = \mathcal{N}(w^Tx_i,\sigma^2)
\end{equation}

所以,
\begin{equation}
    p(Y|X,w) = \prod_{i=1}^N p(y_i|x_i,w) = \prod_{i=1}^N \mathcal{N}(w^Tx_i,\sigma^2)
\end{equation}

而下一步,我们假设$p(w)=\mathcal{N}(0,\Sigma_p)$。又因为$p(Y|X)$与参数$w$无关,所以这是一个定值。所以,我们可以将公式改写为:
\begin{equation}
    p(w|X,Y) \propto p(Y|w,X)p(w) 
\end{equation}

在这里我们将使用到一个共轭的技巧,{\color{red} 因为likelihood function和prior function都是Gaussian Distribution,所有posterior也一定是Gaussian Distribution。}所以,我们可以将公式改写为:
\begin{equation}
    p(w|Data) \sim \mathcal{N}(\mu_w,\Sigma_w) \propto \prod_{i=1}^N \mathcal{N}(w^Tx_i,\sigma^2) \mathcal{N}(0,\Sigma_p)
\end{equation}

我们的目的就是求解$\mu_w = ?,\Sigma_w = ?$。

\section{模型的求解}
对于likelihood function的化简如下所示:
\begin{align}
    p(Y|X,w) 
    = & \prod_{i=1}^N \frac{1}{(2\pi)^{\frac{1}{2}}\sigma} exp\left\{ -\frac{1}{2\sigma^2}(y_i - w^Tx_i)^2 \right\} \\
    = & \frac{1}{(2\pi)^{\frac{N}{2}}\sigma^N} exp\left\{ -\frac{1}{2\sigma^2}\sum_{i=1}^N(y_i - w^Tx_i)^2 \right\}
\end{align}

下一步,我们希望将$\sum_{i=1}^N(y_i - w^Tx_i)^2$改写成矩阵相乘的形式,
\begin{equation}
    \begin{split}
        \sum_{i=1}^N(y_i - w^Tx_i)^2 = &
    \begin{bmatrix}
        y_1 - w^Tx_1 & y_2 - w^Tx_2 & \cdots & y_i - w^Tx_i \\
    \end{bmatrix}
    \begin{bmatrix}
        y_1 - w^Tx_1\\
        y_2 - w^Tx_2\\
        \vdots \\
        y_i - w^Tx_i \\
    \end{bmatrix} \\
    = & (Y^T - w^TX^T)(Y^T - w^TX^T)^T \\
    = & (Y^T - w^TX^T)(Y - Xw)
    \end{split}
\end{equation}

所以,
\begin{equation}
    \begin{split}
        p(Y|X,w) = &  \frac{1}{(2\pi)^{\frac{N}{2}}\sigma^N} exp\left\{ -\frac{1}{2\sigma^2}\sum_{i=1}^N(Y^T - w^TX^T)(Y - Xw) \right\} \\
        = &  \frac{1}{(2\pi)^{\frac{N}{2}}\sigma^N} exp\left\{ -\frac{1}{2}\sum_{i=1}^N(Y^T - w^TX^T)\sigma^{-2}I(Y - Xw) \right\} \\
        & p(Y|X,w) \sim \mathcal{N}(Xw,\sigma^2I)
    \end{split}
\end{equation}

那么,将化简后的结果带入有:
\begin{equation}
    p(w|Data) \sim \mathcal{N}(\mu_w,\Sigma_w) \propto \mathcal{N}(Xw,\sigma^2I) \mathcal{N}(0,\Sigma_p)
\end{equation}
\begin{equation}
    \begin{split}
        \mathcal{N}(Xw,\sigma^2I) \mathcal{N}(0,\Sigma_p) \propto & exp\left\{ -\frac{1}{2}(Y-Xw)^T\sigma^{-2}I(Y-Xw) - \frac{1}{2} w^T\Sigma_p^{-1}w \right\}\\
        = & exp\left\{ -\frac{1}{2\sigma^2}(Y^TY - 2Y^TXw + w^TX^TXw) - \frac{1}{2} w^T\Sigma_p^{-1}w \right\} \\
    \end{split}
\end{equation}

那么这个公式长得怎么的难看我们怎么确定我们想要的$\mu_w,\Sigma_w$。由于知道posterior必然是一个高斯分布,那么我们采用待定系数法来类比确定参数的值即可。对于一个分布$p(x)\sim \mathcal{N}(\mu,\Sigma)$,他的指数部分为:
\begin{equation}
    exp\left\{ -\frac{1}{2}(x-\mu)^T\Sigma^{-1}(x-\mu) \right\} 
    = 
    exp\left\{ -\frac{1}{2}(x^T\Sigma^{-1}x - 2\mu^T\Sigma^{-1}x + \triangle) \right\}
\end{equation}

常数部分已经不重要了,对于我们的求解来说没有任何的用处,所以,我们直接令它为$\triangle$。那么,我们类比一下就可以得到,
\begin{equation}
    w^T\Sigma^{-1}_ww = w^T\left(\sigma^{-2}X^TX+ \Sigma_p^{-1}\right)W
\end{equation}

所以,我们可以得到$\Sigma_w^{-1}=\sigma^{-2}X^TX+\Sigma_p^{-1}$。并且,我们令$\Sigma_w^{-1}=A$。

从二次项中我们得到了$\Sigma_w^{-1}$,那么,下一步,我们期望可以从一次项中得到$\mu_A$的值。我们将一次项提取出来进行观察,可以得到。
\begin{gather}
    \mu^TA = \sigma^{-2}Y^TX \\
    (\mu^TA)^T = (\sigma^{-2}Y^TX)^T \\
    A^T\mu = \sigma^{-2}X^TY \\
    \mu = \sigma^{-2}(A^T)^{-1}X^TY 
\end{gather}

又因为,$\Sigma_w$是一个协方差矩阵,那么他一定是对称的,所以$A^T=A$。于是
\begin{equation}
    \mu_w = \sigma^{-2}A^{-1}X^TY
\end{equation}

\section{小结}
我们利用贝叶斯推断的方法来确定参数之间的分布,也就是确定$p(W|X,Y)$。我们使用Bayes的方法,确定为$p(W|X,Y)\propto p(Y|W,X)p(W)$。并且确定一个噪声分布$\varepsilon\sim\mathcal{N}(0,\sigma^2)$。那么,
\begin{gather}
    p(Y|w,X) \sim \mathcal{N}(Xw,\sigma^2) \\
    P(w) \sim \mathcal{N}(0,\Sigma_p)
\end{gather}

通过推导,我们可以得出,
\begin{equation}
    p(w|X,Y) \sim \mathcal{N}(\mu_w, \Sigma_w)
\end{equation}

其中,
\begin{equation}
    \Sigma_w^{-1}=\sigma^{-2}X^TX+\Sigma_p^{-1} \qquad \mu_w = \sigma^{-2}A^{-1}X^TY \qquad \Sigma_w^{-1}=A
\end{equation}




\end{document}
