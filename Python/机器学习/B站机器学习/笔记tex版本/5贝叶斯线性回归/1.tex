\documentclass[a4paper]{article}
\usepackage[UTF8]{ctex}
\usepackage{geometry}
\usepackage{graphicx}
\usepackage{url}
\usepackage{multirow}
\usepackage{array}
\usepackage{booktabs}
\usepackage{url}
\usepackage{enumitem}
\usepackage{graphicx}
\usepackage{float}
\usepackage{amssymb}
\usepackage{amsmath}
\usepackage{subfig}
\usepackage{longtable}
\usepackage{pifont}

\allowdisplaybreaks


\geometry{a4paper, scale=0.78}

% \begin{figure}[H]
%     \centering
%     \includegraphics[width=.55\textwidth]{E.png}
%     \caption{矩阵与列向量的乘法}
%     \label{fig:my_label_1}
% \end{figure}

% \left\{
% \begin{array}{ll}
%       x+2x+z=2 & \\
%       3x+8y+z=12 & \\
%       4y+z=2
% \end{array}
% \right.

% \begin{enumerate}[itemindent = 1em, itemsep = 0.4pt, parsep=0.5pt, topsep = 0.5pt]

% \end{enumerate}

\title{Bayes Linear Classification 01 Background}
\author{Chen Gong}
\date{05 November 2019}

\begin{document}
\maketitle

数据集$D=\{(x_i,y_i)\}^{N}_{i=1}$,其中$x_i\in\mathbb{R}^{p}$,$y_i\in\mathbb{R}$。

数据矩阵为:(这样可以保证每一行为一个数据点)

\begin{equation}
    X=(x_1, x_2, \cdots, x_N)^T=
    \begin{pmatrix}
    x_1^T \\ 
    x_2^T \\
    \vdots\\
    x_N^T \\
    \end{pmatrix} =
    \begin{pmatrix}
    x_{11} & x_{12} & \dots & x_{1p}\\
    x_{21} & x_{32} & \dots & x_{2p}\\
    \vdots & \vdots & \ddots & \vdots\\
    x_{N1} & x_{N2} & \dots & x_{Np}\\
    \end{pmatrix}_{N\times P}
\end{equation}
\begin{equation}
    Y=
    \begin{pmatrix}
    y_1 \\ 
    y_2 \\
    \vdots\\
    y_N \\
    \end{pmatrix}_{N\times 1}
\end{equation}

拟合函数我们假设为:$f(x) = w^Tx = x^Tw$。

预测值$y=f(x)+\varepsilon$,其中$\varepsilon$是一个Guassian Noise,并且$\varepsilon \sim \mathcal{N}(0,\sigma^2)$。

并且,$x,y,\varepsilon$都是Random variable。

\section{最小二乘估计(Least Square Estimation)}
这实际上就是一个利用数据点的极大似然估计(MLE),并且有一个默认的隐含条件,也就是噪声$\varepsilon$符合Gaussian Distribution。我们的目标是通过估计找到$w$,使得:
\begin{equation}
    w_{MLE} = argmax_w p(Data|w)
\end{equation}

而如果仅仅只是这样来使用,很容易会出现过拟合的问题。所以,我们引入了Regularized LSE,也就是正则化最小二乘法。同时也有一个默认的隐含条件,也是噪声$\varepsilon$符合Gaussian Distribution。在Liner Regression中我们提到了有两种方法来进行思考,也就是Lasso和Ridge Regression。在这里我们可以使用一个Bayes公式,那么:
\begin{equation}
    \begin{split}
        p(w|Data) \propto p(Data|w)p(w) 
    \end{split}
\end{equation}
\begin{equation}
    w_{MAP} = argmax_w p(w|Data) = argmax_wp(Data|w)p(w) 
\end{equation}

那么假设$p(w)$符合一个高斯分布$\mathcal{N}(\mu_0,\Sigma_0)$时,这时是属于Ridge(具体在线性回章节有介绍,也就是正则化的最小二乘估计$\Leftrightarrow$先验服从高斯分布的极大后验估计);而如果$p(w)$符合一个Laplace分布,这是就是Lasso。从概率的角度来思考和统计的角度来思想,我们其实获得的结果是一样的,这在Linear Regression中有证明。但是,我们只证明了Ridge的部分。

\section{贝叶斯估计与频率派估计}
其实在第一部分,我们讲的都是点估计,频率派估计的部分。因为在这些思路中,我们把参数$w$当成a unknown random variable。这实际上就是一个优化问题。而在Beyesian method中,认为$w$是一个随机变量,也就是一个分布,那么我们求的$w$不再是一个数了,而是一个分布。下面我们将要进行Bayes Linear Regression的部分。


\end{document}
