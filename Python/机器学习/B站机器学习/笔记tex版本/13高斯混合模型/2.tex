\documentclass[a4paper]{article}
\usepackage[UTF8]{ctex}
\usepackage{geometry}
\usepackage{graphicx}
\usepackage{url}
\usepackage{multirow}
\usepackage{array}
\usepackage{booktabs}
\usepackage{url}
\usepackage{enumitem}
\usepackage{graphicx}
\usepackage{float}
\usepackage{amssymb}
\usepackage{amsmath}
\usepackage{subfig}
\usepackage{longtable}
\usepackage{pifont}
\usepackage{color}

\allowdisplaybreaks

\geometry{a4paper, scale=0.78}

% \begin{figure}[H]
%     \centering
%     \includegraphics[width=.55\textwidth]{E.png}
%     \caption{矩阵与列向量的乘法}
%     \label{fig:my_label_1}
% \end{figure}

% \left\{
% \begin{array}{ll}
%       x+2x+z=2 & \\
%       3x+8y+z=12 & \\
%       4y+z=2
% \end{array}
% \right.

% \begin{enumerate}[itemindent = 1em, itemsep = 0.4pt, parsep=0.5pt, topsep = 0.5pt]

% \end{enumerate}

%\stackrel{a}{\longrightarrow}

%\underbrace{}_{} %下括号

\title{Gaussian Mixture Model 02 Maximum Likelihood Estimation}
\author{Chen Gong}
\date{24 December 2019}

\begin{document}
\maketitle
本节我们想使用极大似然估计来求解Gaussian Mixture Model (GMM)的最优参数结果。首先,我们明确一下参数的意义:

$X$:Observed data,$X = (x_1, x_2, \cdots, x_N)$。

$(X,Z)$:Complete data,$(X,Z) = \{ (x_1,z_1),(x_2,z_2),\cdots,(x_N,z_N) \}$。

$\theta$:parameter,$\theta=\{ P_1, \cdots, P_k, \mu_1, \cdots, \mu_k,\Sigma_1,\cdots,\Sigma_k \}$。

\section{Maximum Likelihood Estimation求解参数}
\begin{equation}
    \begin{split}
        P(x) 
        = & \sum_Z P(X,Z) \\
        = & \sum_{k=1}^K P(X,z = C_k) \\
        = & \sum_{k=1}^K P(z = C_k)\cdot P(X|z=C_k) \\
        = & \sum_{k=1}^K P_k \cdot \mathcal{N}(X|\mu_k,\Sigma_k)
    \end{split}
\end{equation}

其中,$P_k$也就是数据点去第$k$个高斯分布的概率。下面我们开始使用MLE来求解$\theta$:
\begin{equation}
    \begin{split}
        \hat{\theta}_{MLE} 
        = & \arg\max_{\theta} \log P(X) \\
        = & \arg\max_{\theta} \log \prod_{i=1}^N P(x_i) \\
        = & \arg\max_{\theta}  \sum_{i=1}^N  \log P(x_i) \\
        = & \arg\max_{\theta}  \sum_{i=1}^N  \log \sum_{k=1}^K P_k \cdot \mathcal{N}(x_i|\mu_k,\Sigma_k) \\
    \end{split}
\end{equation}

我们想要求的$\theta$包括,$\theta=\{ P_1, \cdots, P_k, \mu_1, \cdots, \mu_k,\Sigma_1,\cdots,\Sigma_k \}$。

\section{MLE的问题}
按照之前的思路,我们就要分布对每个参数进行求偏导来计算最终的结果。但是问题马上就来了,大家有没有看到$\log$函数里面是一个求和的形式,而不是一个求积的形式。这意味着计算非常的困难。甚至可以说,我们根本就求不出解析解。如果是单一的Gaussian Distribution:
\begin{equation}
    \log P(x_i) = \log \frac{1}{\sqrt{2 \pi} \sigma} exp\left\{ -\frac{(x_i - \mu)^2}{2\sigma} \right\}
\end{equation}

根据log函数优秀的性质,这个问题是可以解的。但是,很不幸后面是一个求和的形式。所以,直接使用MLE求解GMM,无法得到解析解。

\end{document}
