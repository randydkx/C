\documentclass[a4paper]{article}
\usepackage[UTF8]{ctex}
\usepackage{geometry}
\usepackage{graphicx}
\usepackage{url}
\usepackage{multirow}
\usepackage{array}
\usepackage{booktabs}
\usepackage{url}
\usepackage{enumitem}
\usepackage{graphicx}
\usepackage{float}
\usepackage{amssymb}
\usepackage{amsmath}
\usepackage{subfig}
\usepackage{longtable}
\usepackage{pifont}

\allowdisplaybreaks


\geometry{a4paper, scale=0.78}

% \begin{figure}[H]
%     \centering
%     \includegraphics[width=.55\textwidth]{E.png}
%     \caption{矩阵与列向量的乘法}
%     \label{fig:my_label_1}
% \end{figure}

% \left\{
% \begin{array}{ll}
%       x+2x+z=2 & \\
%       3x+8y+z=12 & \\
%       4y+z=2
% \end{array}
% \right.

% \begin{enumerate}[itemindent = 1em, itemsep = 0.4pt, parsep=0.5pt, topsep = 0.5pt]

% \end{enumerate}

\title{Linear Classification 05 Gaussian Discriminate Analysis}
\author{Chen Gong}
\date{03 November 2019}

\begin{document}
\maketitle

前面讲的方法都是概率判别模型,包括,Logistic Regression和Fisher判别分析。接下来我们将要学习的是概率生成模型部分,也就是现在讲到的Gaussian Discriminate Analysis。数据集的相关定义为:

\begin{equation}
    X=(x_1, x_2, \cdots, x_N)^T=
    \begin{pmatrix}
    x_1^T \\ 
    x_2^T \\
    \vdots\\
    x_N^T \\
    \end{pmatrix} =
    \begin{pmatrix}
    x_{11} & x_{12} & \dots & x_{1p}\\
    x_{21} & x_{32} & \dots & x_{2p}\\
    \vdots & \vdots & \ddots & \vdots\\
    x_{N1} & x_{N2} & \dots & x_{Np}\\
    \end{pmatrix}_{N\times P}
\end{equation}
\begin{equation}
    Y=
    \begin{pmatrix}
    y_1 \\ 
    y_2 \\
    \vdots\\
    y_N \\
    \end{pmatrix}_{N\times 1}
\end{equation}

那么,我们的数据集可以记为$\left\{ (x_i,y_i) \right\}_{i=1}^N$,其中,$x_i \in \mathbb{R}^p$,$y_i\in\{+1,-1\}$。我们将样本点分成了两个部分:
\begin{equation}
    \left\{
        \begin{array}{ll}
            C_1 = \left\{ x_i|y_i=1, i=1,2,\cdots,N_1 \right\} & \\
            C_2 = \left\{ x_i|y_i=0, i=1,2,\cdots,N_2 \right\} & \\
        \end{array}
    \right.
\end{equation}

并且有$|C_1|=N_1$,$|C_2|=N_2$,且$N_1+N_2=N$。

\section{概率判别模型与生成模型的区别}
什么是判别模型?所谓判别模型,也就是求
\begin{equation}
    \hat{y} = \mathop{\arg\max_{y}}\ p(y|x) \qquad y\in\{0,1\}
\end{equation}

重点在于求出这个概率来,知道这个概率的值等于多少。而概率生成模型则完全不一样。概率生成模型不需要知道概率值具体是多大,只需要知道谁大谁小即可,具体是对联合概率进行建模。举例即为$p(y=0|x)$和$p(y=1|x)$,谁大谁小的问题。而概率生成模型的求法可以用贝叶斯公式来进行求解,即为:
\begin{equation}
    p(y|x)=\frac{p(x|y)p(y)}{p(x)}=\frac{p(x,y)}{p(x)}\propto p(x,y)
\end{equation}

因为在这个公式中,比例大小$p(x)$与$y$的取值无关,所以它是一个定值。所以,概率生成模型实际上关注的就是一个求联合概率分布的问题。那么,总结一下
\begin{equation}
    p(y|x) \propto p(x|y)p(y) \propto p(x,y)
\end{equation}

其中,$p(y|x)$为Posterior function,$p(y)$为Prior function,p(x|y)为Likelihood function。所以有
\begin{equation}
    \hat{y} = \mathop{\arg\max_{y\in \{0,1\}}}p(y|x) \propto \mathop{\arg\max_{y\in \{0,1\}}}p(x|y)p(y)
\end{equation}

\section{Gaussian Discriminate Analysis模型建立}
在二分类问题中,很显然可以得到,我们的\textbf{先验概率}符合,$p(y)\sim$Bernoulli Distribution。也就是,
\begin{table}[H]
    \centering
    \begin{tabular}{c|cc}
         $y$ & 1 & 0 \\
         \hline
         $p$ & $\varphi$ & $1-\varphi$ \\
    \end{tabular}
    \caption{Bernoulli分布的概率分布表}
    \label{tab:my_label}
\end{table}

所以,可以写出:
\begin{equation}
    p(y)=
    \left\{
        \begin{array}{ll}
            \varphi^y & y=1 \\
            (1-\varphi)^{1-y} & y=0
        \end{array}
    \right.
    \Rightarrow
    \varphi^y(1-\varphi)^{1-y}
\end{equation}

而随后是要确定\textbf{似然函数},我们假设他们都符合高斯分布。对于不同的分类均值是不同的,但是不同变量之间的协方差矩阵是一样的。那么我们可以写出如下的形式:
\begin{equation}
    p(x|y)=
    \left\{
        \begin{array}{ll}
            p(x|y=1)\sim \mathcal{N}(\mu_1, \Sigma) & \\
            p(x|y=0)\sim \mathcal{N}(\mu_2, \Sigma) & \\
        \end{array}
    \right.
    \Rightarrow
    \mathcal{N}(\mu_1, \Sigma)^y\mathcal{N}(\mu_2, \Sigma)^{1-y}
\end{equation}

那么我们的Likelihood function可以被定义为:
\begin{equation}
    \begin{split}
        \mathcal{L}(\theta) = & \log\prod_{i=1}^Np(x_i,y_i) \\
         = & \sum_{i=1}^N\log p(x_i,y_i) \\
         = & \sum_{i=1}^N\log p(x_i|y_i)p(y_i) \\
         = & \sum_{i=1}^N\left[ \log p(x_i|y_i)+ \log p(y_i) \right]\\
         = & \sum_{i=1}^N\left[ \log \mathcal{N}(\mu_1, \Sigma)^{y_i}\mathcal{N}(\mu_2, \Sigma)^{1-y_i}+ \log \varphi^y_i(1-\varphi)^{1-y_i} \right]\\
         = & \sum_{i=1}^N \log \mathcal{N}(\mu_1, \Sigma)^{y_i} + \sum_{i=1}^N \log \mathcal{N}(\mu_2, \Sigma)^{1-y_i}+ \sum_{i=1}^N \log \varphi^{y_i} + \sum_{i=1}^N \log (1-\varphi)^{1-y_i} \\
    \end{split}
\end{equation}

为了方便后续的推演过程,所以,我们将Likelihood function写成,
\begin{center}
    $\mathcal{L}(\theta)$ = \ding{172}+\ding{173}+\ding{179}
\end{center}

并且,我们令:
\ding{172} = $\sum_{i=1}^N \log \mathcal{N}(\mu_1, \Sigma)^y_i$,\ding{173} = $\sum_{i=1}^N \log \mathcal{N}(\mu_2, \Sigma)^{1-y_i}$,

\ding{174} = $\sum_{i=1}^N \log \varphi^{y_i} + \sum_{i=1}^N \log (1-\varphi)^{1-y_i}$。那么上述函数我们可以表示为:
\begin{equation}
    \theta = (\mu_1,\mu_2,\Sigma,\varphi) \qquad \hat{\theta} = \mathop{\arg\max_{\theta}} \mathcal{L}(\theta)
\end{equation}

\section{Likelihood function参数的极大后验估计}
Likelihood function的参数为$\theta = (\mu_1,\mu_2,\Sigma,\varphi)$,下面我们分别用极大似然估计对这四个参数进行求解。下面引入几个公式:
\begin{gather}
    tr(AB) = tr(BA) \\
    \frac{\partial tr(AB)}{\partial A} = B^T \\
    \frac{\partial|A|}{\partial A} = |A|A^{-T} \\
    \frac{\partial \log|A|}{\partial A} = A^{-T}
\end{gather}

\subsection{求解$\varphi$}
\begin{center}
    \ding{174} = $\sum_{i=1}^N \log \varphi^{y_i} + \sum_{i=1}^N \log (1-\varphi)^{1-y_i}$ = $\sum_{i=1}^N y_i\log \varphi + \sum_{i=1}^N (1-y_i) \log (1-\varphi)$ 
\end{center}
\begin{gather}
    \frac{\partial   \textcircled{3}}{\partial \varphi} = \sum_{i=1}^Ny_i\frac{1}{\varphi} - \sum_{i=1}^N (1-y_i)\frac{1}{1-\varphi} = 0 \\
    \sum_{i=1}^N y_i(1-\varphi) - (1-y_i)\varphi = 0 \\
    \sum_{i=1}^N (y_i-\varphi) = 0\\
    \hat{\varphi} = \frac{1}{N} \sum_{i=1}^N y_i
\end{gather}

又因为$y_i=0$或$y_i=1$,所以$\hat{\varphi} = \frac{1}{N} \sum_{i=1}^N y_i = \frac{N_1}{N}$。

\subsection{求解$\mu_1$}
\begin{center}
    \ding{172} = $\sum_{i=1}^N\log \mathcal{N}(\mu_1,\Sigma)^{y_i}$ \\
    = $\sum_{i=1}^Ny_i\log \frac{1}{(2\pi)^{\frac{p}{2}}|\Sigma|^{\frac{1}{2}}}exp\left\{ -\frac{1}{2}(x_i-\mu_1)^T\Sigma^{-1}(x_i-\mu_1) \right\}$ \\
\end{center}

那么求解过程如下所示:
由于到对$\mu_1$求偏导,我们只需要关注公式中和$\mu_1$有关的部分。那么我们可以将问题简化为:
\begin{equation}
    \max_{\mu_1} \sum_{i=1}^Ny_i\log exp\left\{ -\frac{1}{2}(x_i-\mu_1)^T\Sigma^{-1}(x_i-\mu_1) \right\}
\end{equation}

然后将$exp$和$\log$抵消掉,再将括号打开,我们可以得到最终的化简形式:
\begin{equation}
     \max_{\mu_1} -\frac{1}{2}\sum_{i=1}^Ny_i\left\{ x_i^T\Sigma^{-1}x_i - 2\mu_1^T\Sigma^{-1}x_i + \mu_1^T\Sigma^{-1}\mu_1\right\}
\end{equation}

为了方便表示,我们令\ding{172} = $\triangle$。所以,极大似然法求解过程如下:
\begin{equation}
    \begin{split}
        \frac{\partial \triangle}{\partial \mu_1} = & -\frac{1}{2}\sum_{i=1}^N y_i(- 2\Sigma^{-1}x_i + 2\Sigma^{-1}\mu_1) = 0 \\
        = & \sum_{i=1}^N y_i( \Sigma^{-1}x_i - \Sigma^{-1}\mu_1) = 0 \\
        = & \sum_{i=1}^N y_i( x_i - \mu_1) = 0 \\
        = & \sum_{i=1}^N y_i x_i = \sum_{i=1}^N y_i \mu_1  \\
        \mu_1 = & \frac{\sum_{i=1}^N y_i x_i}{\sum_{i=1}^N y_i} = \frac{\sum_{i=1}^N y_i x_i}{N_1}
    \end{split}
\end{equation}

\subsection{求解$\mu_2$}
$\mu_2$的求解过程与$\mu_1$的基本保持一致性。区别点从公式(22)开始,我们有:
\begin{equation}
    \max_{\mu_2}-\frac{1}{2}\sum_{i=1}^N(1-y_i)\left\{ x_i^T\Sigma^{-1}x_i - 2\mu_2^T\Sigma^{-1}x_i + \mu_2^T\Sigma^{-1}\mu_2\right\}
\end{equation}

极大似然法的求解过程如下所示:
\begin{equation}
    \begin{split}
        \frac{\partial \triangle}{\partial \mu_2} = & -\frac{1}{2}\sum_{i=1}^N (1-y_i)(- 2\Sigma^{-1}x_i + 2\Sigma^{-1}\mu_2) = 0 \\
        = & \sum_{i=1}^N (1-y_i)( x_i - \mu_2) = 0 \\
        = & \sum_{i=1}^N x_i - \sum_{i=1}^N y_ix_i  =  N\mu_2  - \sum_{i=1}^N y_i\mu_2\\
        \mu_2 = & \frac{\sum_{i=1}^N x_i - \sum_{i=1}^N y_ix_i}{N - \sum_{i=1}^N y_i} = \frac{\sum_{i=1}^N x_i - \sum_{i=1}^N y_ix_i}{N - N_1}\\
        = & \frac{\sum_{i=1}^N(1-y_i)x_i}{N_2}
    \end{split}
\end{equation}

也可以对于求$\mu_1$来说,求$\mu_2$可以类比,将其中的$N_1$换成$N_2$,其中的$y_i$换成$1-y_i$,可以得到同样的结果。

\subsection{求解$\Sigma$}
如果要使用极大似然估计来求解$\Sigma$,这只与$\mathcal{L}(\theta)$中的\ding{172}和\ding{173}有关。并且\ding{172}+\ding{173}的表达式为:
\begin{equation}
    \sum_{i=1}^N \log \mathcal{N}(\mu_1, \Sigma)^{y_i} + \sum_{i=1}^N \log \mathcal{N}(\mu_2, \Sigma)^{1-y_i}
\end{equation}

那么,按照分类点的方法,我们可以将其改写为:
\begin{equation}
    \hat{\Sigma} = \mathop{\arg\min_{\Sigma}}  \sum_{x\in C_1} \log \mathcal{N}(\mu_1, \Sigma) + \sum_{x\in C_2} \log \mathcal{N}(\mu_2, \Sigma)
\end{equation}

公式加号前后都是一样的,所以,为了方便计算我们暂时只考虑一半的计算:
\begin{equation}
    \begin{split}
        \sum_{i=1}^N \log \mathcal{N}(\mu, \Sigma) 
        = & \sum_{i=1}^N\log \frac{1}{(2\pi)^{\frac{p}{2}}|\Sigma|^{\frac{1}{2}}}exp\left\{ -\frac{1}{2}(x_i-\mu)^T\Sigma^{-1}(x_i-\mu) \right\} \\
        = & - \sum_{i=1}^N \frac{p}{2} \log 2\pi - \sum_{i=1}^N \frac{1}{2} \log |\Sigma| -\frac{1}{2}(x_i-\mu)^T\Sigma^{-1}(x_i-\mu) \\
        = & C -  \frac{N}{2} \log |\Sigma|  -\frac{1}{2}\sum_{i=1}^{N}(x_i-\mu)^T\Sigma^{-1}(x_i-\mu) \\ 
        = & C -  \frac{N}{2} \log |\Sigma|  -\frac{1}{2}\sum_{i=1}^{N}tr\left((x_i-\mu)^T\Sigma^{-1}(x_i-\mu)\right) \\
        = & C -  \frac{N}{2} \log |\Sigma|  -\frac{1}{2}\sum_{i=1}^{N}tr\left((x_i-\mu)(x_i-\mu)^T\Sigma^{-1} \right) \\
    \end{split}
\end{equation}

而且,
\begin{equation}
    S = \frac{1}{N}\sum_{i=1}^{N} (x_i-\mu)(x_i-\mu)^T
\end{equation}

所以,
\begin{equation}
    \sum_{i=1}^N \log \mathcal{N}(\mu, \Sigma) = C - \frac{N}{2} \log |\Sigma|  -\frac{N}{2}tr(S\Sigma^{-1})
\end{equation}

那么代入公式(27)中,我们可以得到:
\begin{equation}
    \begin{split}
        \hat{\Sigma} 
        = & \mathop{\arg\max_{\Sigma}}\ C - \frac{N_1}{2} \log |\Sigma| -\frac{N_1}{2}tr(S_1\Sigma^{-1}) + C - \frac{N_2}{2} \log |\Sigma| -\frac{N_2}{2}tr(S_2\Sigma^{-1}) \\
        = & \mathop{\arg\max_{\Sigma}}\  - \frac{N}{2}  \log |\Sigma| -\frac{N_1}{2}tr(S_1\Sigma^{-1}) -\frac{N_2}{2}tr(S_2\Sigma^{-1}) \\
        = & \mathop{\arg\min_{\Sigma}}\  N  \log |\Sigma| + N_1tr(S_1\Sigma^{-1}) +N_2tr(S_2\Sigma^{-1}) \\
    \end{split}
\end{equation}

我们令函数$  N  \log |\Sigma| + N_1tr(S_1\Sigma^{-1}) +N_2tr(S_2\Sigma^{-1}) = \triangle$,那么对$\Sigma$求偏导并令其等于0可得:
\begin{equation}
    \begin{split}
         \frac{\partial \triangle}{\partial \Sigma} =  N \Sigma^{-1} - N_1\Sigma^{-1}S_1\Sigma^{-1} - N_2\Sigma^{-1}S_2\Sigma^{-1}  = 0 
    \end{split}
\end{equation}

对上式左乘$\Sigma$,又乘$\Sigma$得到$N \Sigma - N_1S_1 - N_2S_2 = 0 $。

解得:
\begin{equation}
    \Sigma = \frac{N_1S_1 + N_2S_2}{N}
\end{equation}

其中对$tr(S_1\Sigma^{-1})$求偏导的过程如下(由于$\Sigma\Sigma^{-1}=\mathbb{I}$,所以$d\, (\Sigma^{-1}\Sigma)=\mathbb{O}\Rightarrow (d\,\Sigma)\Sigma^{-1}+\Sigma d\,(\Sigma^{-1})=0
\Rightarrow d\,\Sigma^{-1} = -\Sigma^{-1}(d\,\Sigma)\Sigma^{-1}$:
\begin{equation}
    \begin{split}
        d \,tr(S_1\Sigma^{-1})= & tr(S_1d\,\Sigma^{-1}) \\
        = & tr(-S_1\Sigma^{-1}(d\,\Sigma)\Sigma^{-1})\\
        = & tr(-\Sigma^{-1}S_1\Sigma^{-1}d\,\Sigma)\\
    \end{split}
\end{equation}

于是$\frac{\partial tr(S_1\Sigma^{-1})}{\partial \Sigma}=-\Sigma^{-1}S_1\Sigma^{-1}$。同理可以知道$\frac{\partial tr(S_2\Sigma^{-1})}{\partial \Sigma}=-\Sigma^{-1}S_2\Sigma^{-1}$。

\section{总结}
下面对Gaussian Discriminate Analysis做一个简单的小结。我们使用模型为:
\begin{gather}
    \hat{y} = \mathop{\arg\max}_{y\in \{0,1\}}p(y|x) \propto \mathop{\arg\max}_{y\in \{0,1\}}p(x|y)p(y) \\ 
    \left\{
        \begin{array}{ll}
            p(y)= \varphi^y(1-\varphi)^{1-y} & \\
            p(x|y)= \mathcal{N}(\mu_1, \Sigma)^y\mathcal{N}(\mu_2, \Sigma)^{1-y} & \\
        \end{array}
    \right.
\end{gather}

利用极大似然估计得到的结果为:
\begin{equation}
    \theta = (\mu_1,\mu_2,\Sigma,\varphi)=
    \left\{
        \begin{array}{ll}
            \hat{\varphi} = \frac{N_1}{N}  & \\
            \mu_1 = \frac{\sum_{i=1}^N y_i x_i}{N_1} & \\
            \mu_2 = \frac{\sum_{i=1}^N (1-y_i)x_i}{N_2} & \\
            \Sigma = \frac{N_1S_1 + N_2S_2}{N} & \\
        \end{array}
    \right.
\end{equation}

\end{document}


