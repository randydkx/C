\documentclass[a4paper]{article}
\usepackage[UTF8]{ctex}
\usepackage{geometry}
\usepackage{graphicx}
\usepackage{url}
\usepackage{multirow}
\usepackage{array}
\usepackage{booktabs}
\usepackage{url}
\usepackage{enumitem}
\usepackage{graphicx}
\usepackage{float}
\usepackage{amssymb}
\usepackage{amsmath}
\usepackage{subfig}
\usepackage{longtable}
\usepackage{pifont}
\usepackage{color}
\usepackage{listings}
\usepackage{xcolor}

\allowdisplaybreaks

\geometry{a4paper, scale=0.78}

% \begin{figure}[H]
%     \centering
%     \includegraphics[width=.55\textwidth]{E.png}
%     \caption{矩阵与列向量的乘法}
%     \label{fig:my_label_1}
% \end{figure}

% \left\{
% \begin{array}{ll}
%       x+2x+z=2 & \\
%       3x+8y+z=12 & \\
%       4y+z=2
% \end{array}
% \right.

% \begin{enumerate}[itemindent = 1em, itemsep = 0.4pt, parsep=0.5pt, topsep = 0.5pt]

% \end{enumerate}

%\stackrel{a}{\longrightarrow}

\title{Probability Graph 01 Background}
\author{Chen Gong}
\date{23 November 2019}

\begin{document}

\maketitle
机器学习的重要思想就是,对已有的数据进行分析,然后对未知数据来进行预判或者预测等。这里的图和我们之前学习的数据结构中的图有点不太一样,俗话说有图有真相,这里的图是将概率的特征引入到图中,方便我们进行直观分析。

\section{概率的基本性质}
我们假设现在有一组高维随机变量,$p(x_1,x_2,\cdots,x_n)$,它有两个非常基本的概率,也就是条件概率和边缘概率。条件概率的描述为$p(x_i)$,条件概率的描述为$p(x_j|x_i)$。

同时,根据这两个基本的概率,我们可以得到两个基本的运算法则:Sum Rule和Product Rule。

Sum Rule:$p(x_1)=\int p(x_1,x_2)dx_2$。

Product Rule:$p(x_1,x_2) = p(x_1)p(x_2|x_1) = p(x_2)p(x_1|x_2)$。

根据这两个基本的法则,我们可以推出Chain Rule和Bayesian Rule。

Chain Rule:
\begin{equation}
   p(x_1,x_2,\cdots,x_N) = \prod_{i=1}^N p(x_i|x_1,x_2,\cdots,x_{i-1}) 
\end{equation}


Bayesian Rule:
\begin{equation}
    p(x_2|x_1) = \frac{p(x_1,x_2)}{p(x_1)} = \frac{p(x_1,x_2)}{\int p(x_1,x_2)dx_2} = \frac{p(x_1|x_2)p(x_2)}{\int p(x_1|x_2)p(x_2)dx_2}
\end{equation}

\section{条件独立性}

首先,我们想想高维随机变量所遇到的困境,也就是维度高,计算复杂度高。大家想想,当维度较高时,这个$p(x_1,x_2,\cdots,x_N) = \prod_{i=1}^N p(x_i|x_1,x_2,\cdots,x_{i-1}) $肯定会算炸去。所以,我们需要简化运算,之后我们来说说我们简化运算的思路。

1. 假设每个维度之间都是相互独立的,那么我们有$p(x_1,x_2,\cdots,x_N)=\prod_{i=1}^N p(x_N)$。比如,朴素贝叶斯就是这样的设计思路,也就是$p(x|y)=\prod_{i=1}^N p(x_i|y)$。但是,我们觉得这个假设太强了,实际情况中的依赖比这个要复杂很多。所以我们像放弱一点,增加之间的依赖关系,于是我们有提出了马尔科夫性质(Markov Propert)。


2. 假设每个维度之间是符合马尔科夫性质(Markov Property)的。所谓马尔可夫性质就是,对于一个序列$\{ x_1,x_2,\cdots,x_N \}$,第$i$项仅仅只和第$i-1$项之间存在依赖关系。用符号的方法我们可以表示为:
\begin{equation}
    x_j\bot x_{i+1}| x_i, j<i
\end{equation}

在HMM里面就是这样的齐次马尔可夫假设,但是还是太强了,我们还是要想办法削弱。自然界中经常会出现,序列之间不同的位置上存在依赖关系,因此我们提出了{\color{red} 条件独立性}。

3. 条件独立性:\textbf{条件独立性假设是概率图的核心概念。它可以大大的简化联合概率分布}。而用图我们可以大大的可视化表达条件独立性。我们可以描述为:
\begin{equation}
    X_A \bot X_B |X_C
\end{equation}
而$X_A,X_B,X_C$是变量的集合,彼此之间互不相交。

\section{概率图算法分类}
概率图的算法大致可以分为三类,也就是,表示(Representation),推断(Inference)和学习(Learning)。
\subsection{Representation}
知识表示的方法,可以分为有向图,Bayesian Network;和无向图,Markov Network,这两种图通常用来处理变量离散的情况。对于连续性的变量,我们通常采用高斯图,同时可以衍生出,Gaussian Bayesian Network和Guassian Markov Network。

\subsection{Inference}
推断可以分为精准推断和近似推断。所谓推断就是给定已知求概率分布。近似推断中可以分为确定性推断(变分推断)和随机推断(MCMC),MCMC是基于蒙特卡罗采样的。

\subsection{Learning}
学习可以分为参数学习和结构学习。在参数学习中,参数可以分为变量数据和非隐数据,我们可以采用有向图或者无向图来解决。而隐变量的求解我们需要使用到EM算法,这个EM算法在后面的章节会详细推导。而结构学习则是,需要我们知道使用那种图结构更好,比如神经网络中的节点个数,层数等等,也就是现在非常热的Automate Machine Learning。




















 
\end{document}
