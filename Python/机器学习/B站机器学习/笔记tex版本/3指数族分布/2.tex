\documentclass[a4paper]{article}
\usepackage[UTF8]{ctex}
\usepackage{geometry}
\usepackage{graphicx}
\usepackage{url}
\usepackage{multirow}
\usepackage{array}
\usepackage{booktabs}
\usepackage{url}
\usepackage{enumitem}
\usepackage{graphicx}
\usepackage{float}
\usepackage{amssymb}
\usepackage{amsmath}
\usepackage{subfig}
\usepackage{longtable}

\allowdisplaybreaks
\geometry{a4paper, scale=0.78}

\title{Exponential Family Distribution 02 Example}
\author{Chen Gong}
\date{23 October 2019}

\begin{document}

\maketitle

本节的主要内容是演示Guassian Distribution的指数族表达形式,将高斯函数的形式转换为指数族分布的通用表达形式。

指数族分布的基本形式可以表示为:
\begin{equation}
    p(x|y)=h(x)exp\left\{ \eta^T\varphi(x)-A(\eta) \right\}
\end{equation}

$\eta$:参数向量parameter,$\eta \in \mathbb{R}^p$。

$A(\eta)$:log partition function (配分函数)。

$\varphi(x)$:充分统计量sufficient statistics magnitude。

\section{思路分析}
高斯分布的概率密度函数可表示为:
\begin{equation}
    p(x|\mu,\sigma^2) = \frac{1}{\sqrt{2\pi}\sigma}exp\left\{ -\frac{(x-\mu)^2}{\sigma^2} \right\}
\end{equation}

观察指数族分布的表达形式,高斯分布的参数向量是有关于$\theta=(\mu,\sigma)$的。首先观察指数部分的第一部分$\eta^T\varphi(x)$,只有这个部分和$x$相关。那么把这个部分搞定,系数就是参数矩阵,剩下的就是配分函数了,而且配分函数是一个关于$\eta$的函数。

\section{将Guassian Distribution改写为指数族分布的形式}
具体推导过程如下所示:
\begin{align}
    p(x|\theta)= & \frac{1}{\sqrt{2\pi}\sigma}exp\left\{ -\frac{(x-\mu)^2}{2\sigma^2} \right\} \\
    = & \frac{1}{\sqrt{2\pi}\sigma}exp\left\{ -\frac{1}{2\sigma^2}(x^2-2\mu x + \mu^2) \right\} \\
    = & \frac{1}{\sqrt{2\pi}\sigma}exp\left\{ -\frac{x^2}{2\sigma^2}+\frac{\mu x}{\sigma^2}-\frac{\mu^2}{2\sigma^2} \right\} \\
    = & \frac{1}{\sqrt{2\pi}\sigma}exp\left\{
        \begin{pmatrix}
            \frac{\mu}{\sigma^2} \\
            -\frac{1}{2\sigma^2}
        \end{pmatrix}
        \begin{pmatrix}
            x & x^2 \\
        \end{pmatrix}
        -\frac{\mu^2}{2\sigma^2}
        \right\} \\
    = & exp\log \frac{1}{\sqrt{2\pi}\sigma} exp\left\{
        \begin{pmatrix}
            \frac{\mu}{\sigma^2} \\
            -\frac{1}{2\sigma^2}
        \end{pmatrix}
        \begin{pmatrix}
            x & x^2 \\
        \end{pmatrix}
        -\frac{\mu^2}{2\sigma^2}
        \right\} \displaybreak \\ 
    = & exp\left\{
        \begin{pmatrix}
            \frac{\mu}{\sigma^2} \\
            -\frac{1}{2\sigma^2}
        \end{pmatrix}
        \begin{pmatrix}
            x & x^2 \\
        \end{pmatrix}
        -\left(\frac{\mu^2}{2\sigma^2} -\frac{1}{2}\log 2\pi\sigma\right) 
        \right\}
\end{align}

令:
\begin{equation}
    \eta=
    \begin{pmatrix}
        \eta_1 \\
        \eta_2
    \end{pmatrix}
    =
    \begin{pmatrix}
        \frac{\mu}{\sigma^2} \\
        -\frac{1}{2\sigma^2}
    \end{pmatrix}
    \Longrightarrow
    \left\{
    \begin{array}{ll}
         \eta_1 = \frac{\mu}{\sigma^2} & \\
         \eta_2 = -\frac{1}{2\sigma^2} &
    \end{array}
    \right.
    \Longrightarrow
    \left\{
    \begin{array}{ll}
         \mu = -\frac{\eta_1}{2\eta_2} & \\
         \sigma^2 = -\frac{1}{2\eta_2} &
    \end{array}
    \right.
\end{equation}

到了现在,我们离最终的胜利只差一步了,
\begin{equation}
   \eta=
    \begin{pmatrix}
        \eta_1 \\
        \eta_2
    \end{pmatrix}
    \quad
    \varphi(x)=
    \begin{pmatrix}
        x \\
        x^2
    \end{pmatrix}   
\end{equation}
\begin{equation}
    A(\eta)=-\frac{\eta_1^2}{4\eta_2}+\frac{1}{2}\log (2\pi\cdot-\frac{1}{2\eta_2})=-\frac{\eta_1^2}{4\eta_2}+\frac{1}{2}\log(-\frac{\pi}{\eta_2})
\end{equation}

于是,Guassian Distribution成功的被我们化成了指数族分布的形式$exp\left\{ \eta^T\varphi(x)-A(\eta) \right\}$。
\end{document}
