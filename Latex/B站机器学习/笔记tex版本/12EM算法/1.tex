\documentclass[a4paper]{article}
\usepackage[UTF8]{ctex}
\usepackage{geometry}
\usepackage{graphicx}
\usepackage{url}
\usepackage{multirow}
\usepackage{array}
\usepackage{booktabs}
\usepackage{url}
\usepackage{enumitem}
\usepackage{graphicx}
\usepackage{float}
\usepackage{amssymb}
\usepackage{amsmath}
\usepackage{subfig}
\usepackage{longtable}
\usepackage{pifont}
\usepackage{color}

\allowdisplaybreaks

\geometry{a4paper, scale=0.78}

% \begin{figure}[H]
%     \centering
%     \includegraphics[width=.55\textwidth]{E.png}
%     \caption{矩阵与列向量的乘法}
%     \label{fig:my_label_1}
% \end{figure}

% \left\{
% \begin{array}{ll}
%       x+2x+z=2 & \\
%       3x+8y+z=12 & \\
%       4y+z=2
% \end{array}
% \right.

% \begin{enumerate}[itemindent = 1em, itemsep = 0.4pt, parsep=0.5pt, topsep = 0.5pt]

% \end{enumerate}

%\stackrel{a}{\longrightarrow}

%\underbrace{}_{} %下括号


\title{Expectation Maximization 01 Algorithm Convergence}
\author{Chen Gong}
\date{17 December 2019}

\begin{document}
\maketitle
Expectation Maximization (EM)算法,是用来解决具有隐变量的模型的概率计算问题。在比较简单的情况中,我们可以直接得出我们想要求得的参数的解析解,比如:MLE: $p(X|\theta)$。我们想要求解的结果就是:
\begin{equation}
    \theta_{MLE} = \arg\max_{\theta}\log p(X|\theta)
\end{equation}

然而一旦问题变得复杂起来以后,就不是这么简单了,特别是引入了隐变量之后。

\section{EM算法简述}
实际上,EM算法的描述也并不是很难,我们知道,通常我们想求的似然函数为$p(X|\theta)$。引入隐变量之后,原式就变成了:
\begin{equation}
    p(X|\theta) = \int \log p(X,Z|\theta)p(Z|X,\theta^{(t)})dZ \\
\end{equation}

EM算法是一种迭代的算法,我们的目标是求:
\begin{equation}
    \begin{split}
        \theta^{(t+1)} = &\arg\max_{\theta} \int_Z
        \log p(X,Z|\theta)p(Z|X,\theta^{(t)})dZ \\
        = &\arg\max_{\theta} \mathbb{E}_{Z \sim p(Z|X,\theta^{(t)})}[\log p(X,Z|\theta)]
    \end{split}
\end{equation}

也就是找到一个更新的参数$\theta$,使得$\log p(X,Z|\theta)$出现的概率更大。

\section{EM算法的收敛性}
我们想要证的是当$\theta^{(t)} \longrightarrow \theta^{(t+1)}$时,有$\log p(X|\theta^{(t)}) \leq \log p(X|\theta^{(t+1)})$。这样才能说明我们的每次迭代都是有效的。
\begin{equation}
    \log p(X|\theta) = \log \frac{p(X,Z|\theta)}{ p(Z|X,\theta)} = \log p(X,Z|\theta) - \log p(Z|X,\theta)
\end{equation}

下一步,则是同时对两边求关于$p(Z|X,\theta^{(t)})$的期望。

左边:
\begin{equation}
    \begin{split}
        \mathbb{E}_{Z\sim p(Z|X,\theta^{(t)})}[\log p(X|\theta)] 
        = & \int_Z p(Z|X,\theta^{(t)}\log p(X|\theta) dZ \\
        = & \log p(X|\theta) \int_Z p(Z|X,\theta^{(t)}) dZ \\
        = & \log p(X|\theta) \cdot 1 = \log p(X|\theta)
    \end{split}
\end{equation}

右边:
\begin{equation}
    \underbrace{\int_Z p(Z|X,\theta^{(t)}) \log p(X,Z|\theta) dZ}_{Q(\theta,\theta^{(t)})} - \underbrace{\int_Z p(Z|X,\theta^{(t)}) \log p(Z|X,\theta) dZ}_{H(\theta,\theta^{(t)})}
\end{equation}

大家很容易就观察到,$Q(\theta,\theta^{(t)})$就是我们要求的
$\theta^{(t+1)} = \arg\max_{\theta} \int_Z p(X,Z|\theta)p(Z|X,\theta^{(t)})dZ$。
那么,根据定义,我们可以很显然的得到:$Q(\theta^{(t+1)},\theta^{(t)}) \geq Q(\theta,\theta^{(t)})$。当$\theta = \theta^{(t)}$时,等式也是显然成立的,那么我们可以得到:
\begin{equation}
    Q(\theta^{(t+1)},\theta^{(t)}) \geq Q(\theta^{(t)},\theta^{(t)})
\end{equation}

这时,大家想一想,我们已经得到了$Q(\theta^{(t+1)},\theta^{(t)}) \geq Q(\theta^{(t)},\theta^{(t)})$了。如果,$H(\theta^{(t+1)},\theta^{(t)}) \leq H(\theta^{(t)},\theta^{(t)})$。我们就可以很显然的得出,$\log p(X|\theta^{(t)}) \leq \log p(X|\theta^{(t+1)})$了。

证明:
\begin{equation}
    \begin{split}
        H(\theta^{(t+1)},\theta^{(t)}) - H(\theta^{(t)},\theta^{(t)}) = & \int_Z p(Z|X,\theta^{(t)}) \log p(Z|X,\theta^{(t+1)}) dZ - \int_Z p(Z|X,\theta^{(t)}) \log p(Z|X,\theta^{(t)}) dZ \\
        = & \int_Z p(Z|X,\theta^{(t)}) \log \frac{p(Z|X,\theta^{(t+1)})}{p(Z|X,\theta^{(t)})}dZ \\
        = & -KL(p(Z|X,\theta^{(t)})||p(Z|X,\theta^{(t+1)})) \leq 0
    \end{split}
\end{equation}

或者,我们也可以使用Jensen inequality。很显然,$\log$函数是一个concave函数,那么有$\mathbb{E}[\log X] \leq \log [\mathbb{E}[X]]$,那么:
\begin{equation}
    \begin{split}
        \int_Z p(Z|X,\theta^{(t)}) \log \frac{p(Z|X,\theta^{(t+1)})}{p(Z|X,\theta^{(t)})}dZ 
        = & \mathbb{E}_{Z\sim p(Z|X,\theta^{(t)})}\left[ \log \frac{p(Z|X,\theta^{(t+1)})}{p(Z|X,\theta^{(t)})} \right] \\
        \leq & \log \left[ \mathbb{E}_{Z\sim p(Z|X,\theta^{(t)})} \left[ \frac{p(Z|X,\theta^{(t+1)})}{p(Z|X,\theta^{(t)})} \right] \right] \\
         = & \log \left[ \int_Z p(Z|X,\theta^{(t)}) \left[ \frac{p(Z|X,\theta^{(t+1)})}{p(Z|X,\theta^{(t)})} \right]dZ \right] \\
         = & \log \int_Z p(Z|X,\theta^{(t+1)}) dZ\\
         = & 0
    \end{split}
\end{equation}

所以,从两个方面我们都证明了,$\log p(X|\theta^{(t)}) \leq \log p(X|\theta^{(t+1)})$。那么,经过每次的迭代,似然函数在不断的增大。这就证明了我们的更新是有效的,也证明了算法是收敛的。


\end{document}