\documentclass[a4paper]{article}
\usepackage[UTF8]{ctex}
\usepackage{geometry}
\usepackage{graphicx}
\usepackage{url}
\usepackage{multirow}
\usepackage{array}
\usepackage{booktabs}
\usepackage{url}
\usepackage{enumitem}
\usepackage{graphicx}
\usepackage{float}
\usepackage{amssymb}
\usepackage{amsmath}
\usepackage{subfig}
\usepackage{longtable}
\usepackage{pifont}
\usepackage{color}
\usepackage{listings}
\usepackage{xcolor}

\allowdisplaybreaks

\geometry{a4paper, scale=0.78}

% \begin{figure}[H]
%     \centering
%     \includegraphics[width=.55\textwidth]{E.png}
%     \caption{矩阵与列向量的乘法}
%     \label{fig:my_label_1}
% \end{figure}

% \left\{
% \begin{array}{ll}
%       x+2x+z=2 & \\
%       3x+8y+z=12 & \\
%       4y+z=2
% \end{array}
% \right.

% \begin{enumerate}[itemindent = 1em, itemsep = 0.4pt, parsep=0.5pt, topsep = 0.5pt]

% \end{enumerate}

%\stackrel{a}{\longrightarrow}

\title{Kernel Method 02 The Definition of Positive Kernel Function}
\author{Chen Gong}
\date{21 November 2019}

\begin{document}
\maketitle
上一节中,我们已经讲了什么是核函数,也讲了什么是核技巧,以及核技巧存在的意义是什么。我们首先想想,上一小节我们提到的核函数的定义。

对于一个映射$K$,我们有两个输入空间$\mathcal{X}\times\mathcal{X},\mathcal{X}\in\mathbb{R}^p$,可以形成一个映射$\mathcal{X}\times \mathcal{X}\mapsto\mathbb{R}$。对于,$\forall\ x,z \in \mathcal{X}$,存在一个映射$\phi:\mathcal{X}\mapsto \mathbb{R}$,使得$K(x,z)=<\phi(x),\phi(z)>$。那么这个$K(\cdot)$,就被我们称为核函数。(<>代表内积运算)

这既是我们上一节中将的核函数的定义,实际上这个核函数的精准定义,应该是正定核函数。在本小节中,我们将会介绍核函数的精准定义,什么是正定核函数?并介绍内积和希尔伯特空间(Hilbert Space)的定义。这一小节虽然看着会有些枯燥,实际上非常的重要。

\section{核函数的定义}
核函数的定义,也就是对于一个映射$K$,存在一个映射$\mathcal{X}\times\mathcal{X}\mapsto \mathbb{R}$,对于$x,z\in \mathcal{X}$都成立,则称$K(x,z)$为核函数。

对比一下,我们就会发现,这个定义实际上比我们之前学的定义要简单很多。好像是个阉割版,下面我们来介绍两个正定核的定义方法。

\section{正定核的定义}
正定核函数的定义有两个,我首先分别进行描述一下:

\subsection{第一个定义}
现在存在一个映射$K:\mathcal{X}\times\mathcal{X}\mapsto\mathbb{R}$。对于$\forall x,z \in \mathcal{X}$。如果存在一个$\phi:\mathcal{X}\mapsto \mathbb{R}^p$,并且$\phi(x)\in\mathcal{H}$,使得$K(x,z) = <\phi(x),\phi(z)>$,那么称$K(x,z)$为正定核函数。

\subsection{第二个定义}
对于一个映射$K:\mathcal{X}\times\mathcal{X}\mapsto\mathbb{R}$,对于$\forall x,z\in \mathcal{X}$,都有$K(x,z)$。如果$K(x,z)$满足,1. 对称性;2. 正定性;那么称$K(x,z)$为一个正定核函数。

我们来分析一个,首先什么是对称性?这个非常的好理解,也就是$K(x,z)=K(z,x)$。那么什么又是正定性呢?那就是任取$N$个元素,$x_1,x_2,\cdots,x_N\in \mathcal{X}$,对应的Gram Matrix是半正定的,其中Gram Matrix用$K$来表示为$K=[K(x_i,x_j)]$。

对于第一个对称性,我们其实非常好理解,不就是内积嘛!有一定数学功底的同学一定知道,内积和距离是挂钩的,距离之间一定是对称的。那么正定性就要好好讨论一下了。我们知道这两个定义之间是等价的,为什么会有正定性呢?我们需要进行证明,这个证明可以被我们描述为:
\begin{center}
    {\color{red}
    $K(x,z) = <\phi(x),\phi(z)> \Longleftrightarrow$ Gram Matrix是半正定矩阵
    }
\end{center}

这个等式的证明我们留到下一节再来进行,这里我们首先需要学习一个很重要的概念叫做,希尔伯特空间($\mathcal{H}$:Hilbert Space)。小编之前被这个概念搞得头晕,特别还有一个叫再生核希尔伯特空间的玩意,太恶心了。

\section{Hilbert Space ($\mathcal{H}$)}
{\color{red} Hilbert Space是一个完备的,可能是无限维的,被赋予内积运算的线性空间。}下面我们对这个概念进行逐字逐句的分析。

\textbf{线性空间}:也就是向量空间,这个空间的元素就是向量,向量之间满足加法和乘法的封闭性,实际上也就是线性表示。空间中的任意两个向量都可以由基向量线性表示。

\textbf{完备的}:完备性简单的认为就是对极限的操作是封闭的。我们怎么理解呢?若有一个序列为$\{K_n\}$,这里强调一下Hilbert Space是一个函数空间,空间中的元素就是函数。所以,$K_n$就是一个函数。那么就会有:
\begin{equation}
    \lim_{n\longrightarrow +\infty} K_n = K \in \mathcal{H}
\end{equation}

所以,我们理解一下就是会和无限维这个重要的概念挂钩。我理解的主要是Hilbert Space在无限维满足线性关系。

\textbf{内积}:内积应该满足三个定义,1. 正定性;2. 对称性;3. 线性。下面我们逐个来进行解释:

1. 对称性:也就是$f,g\in \mathcal{H}$,那么就会有$<f,g> = <g,f>$。其中,$f,g$是函数,我们可以认为Hilbert Space是基于函数的,向量是一个特殊的表达。其实,也就是函数可以看成一个无限维的向量。大家在这里是不是看到了无限维和完备性的引用,他们的定义之间是在相互铺垫的。

2. 正定性:也就是$<f,f> \leq 0$,等号当且仅当$f=0$是成立。

3. 线性也就是满足:$<r_1f_1+r_2f_2, g> = r_1<f_1,g>+r_2<f_2,g>$。

描述上述三条性质的原因是什么呢?也就是我们要证明一个空间中加入一些运算。如果,判断这个运算是不是内积运算,我们需要知道这个运算满不满足上述三个条件。

~\\

现在我们介绍了大致的基本概念了,我们回到这样一个问题,对于正定核我们为什么要有两个定义?这个思想和我们之前学到的Kernel Trick非常的类似了,Kernel Trick跳过了寻找$\phi$这个过程。因为,直接用定义不好找,
































\end{document}
