\documentclass[a4paper]{article}
\usepackage[UTF8]{ctex}
\usepackage{geometry}
\usepackage{graphicx}
\usepackage{url}
\usepackage{multirow}
\usepackage{array}
\usepackage{booktabs}
\usepackage{url}
\usepackage{enumitem}
\usepackage{graphicx}
\usepackage{float}
\usepackage{amssymb}
\usepackage{amsmath}
\usepackage{subfig}
\usepackage{longtable}
\usepackage{pifont}
\usepackage{color}

\allowdisplaybreaks

\geometry{a4paper, scale=0.78}

% \begin{figure}[H]
%     \centering
%     \includegraphics[width=.55\textwidth]{E.png}
%     \caption{矩阵与列向量的乘法}
%     \label{fig:my_label_1}
% \end{figure}

% \left\{
% \begin{array}{ll}
%       x+2x+z=2 & \\
%       3x+8y+z=12 & \\
%       4y+z=2
% \end{array}
% \right.

% \begin{enumerate}[itemindent = 1em, itemsep = 0.4pt, parsep=0.5pt, topsep = 0.5pt]

% \end{enumerate}

%\stackrel{a}{\longrightarrow}

%\underbrace{}_{} %下括号

\title{Markov Chain Monte Carlo 05 Sampling}
\author{Cheh Gong}
\date{03 January 2020}

\begin{document}
\maketitle
在前面的章节中,我们已经基本介绍了Markov Chain Monte Carlo Sampling的基本概念,基本思路和主要方法。那么这一小节中,我们将主要来介绍一下,什么是采样?我们为什么而采样?什么样的样本是好的样本?以及我们采样中主要会遇到哪些困难?
\section{采样的动机}
这一小节的目的就是我们要知道什么是采样的动机,我们为什么而采样?

1. 首先第一点很简单,采样本身就是发出常见的任务,我们机器学习中经常需要进行采样来完成各种各样的任务。如果从一个$P(X)$中采出一堆样本。

2. 求和求积分。包括大名鼎鼎的Monte Carlo算法。我们求$P(X)$主要是为了求在$P(X)$概率分布下的一个相关函数的期望,也就是:
\begin{equation}
    \int P(x)f(x)dx = \mathbb{E}_{P(X)}[f(X)] \approx \frac{1}{N} \sum_{i=1}^N f(x^{(i)})
\end{equation}
而我们是通过采样来得到$P(X) \sim \{ x^{(1)},x^{(2)},\cdots, x^{(N)} \}$样本点。

\section{什么样的是好样本}
既然,我们知道了采样的目的和动机,下一个问题就自然是,同样是采样,什么样的样本就是好样本呢?或者说是采样的效率更高一些。

1. 首先样本的分布肯定是要趋向于原始的目标分布吧,也就是说样本要趋向于高概率选择区域。或者是说,采出来的样本出现的概率和实际的目标分布的概率保持一致。

2. 样本和样本之间是相互独立的。这个就没有那么直观了。大家想一想就知道了,如果我采出来的一堆样本之间都差不多,那么就算采出来了趋向于高概率选择区域的样本,那采样效率太低了,样本中反映的信息量太有限了。

\section{实际采样中的困难}
实际采样中,采样时困难的,为什么呢?我们这里主要介绍两点:

1. \textbf{Partation function is intractable.} 我们的后验分布往往被写成$P(X) = \frac{1}{Z} \hat{P}(X)$,上面这个$\hat{P}(X)$都比较好求,就是等于 Likelihood $\times$ Prior。而$Z$就是我们要求的归一化常数,它非常的难以计算,$Z = \int \hat{P}(X) dX$,这几乎就是不可计算的。所以,有很多采样方法就是想要跳过求$P(X)$的过程,来从一个近似的分布中进行采样,当然这个近似的分布采样要比原分布简单。比如:Rejection Sampling和Importance Sampling。

2. \textbf{The curse of high dimension}. 如果样本空间$\mathcal{X} \in \mathbb{R}^p$,每个维度都有$K$个状态的话。那么总的样本空间就有$K^p$的状态。要知道那个状态的概率高,就必须要遍历整个样本空间,不然就不知道哪个样本的概率高,如果状态的数量是这样指数型增长的话,全看一遍之后进行采样时不可能的。所以,直接采样的方法是不可行的。

\section{采样方法}
Rejection Sampling和Importance Sampling,都是借助一个$Q(x)$去逼近目标分布$P(x)$,通过从$Q(x)$中进行采样来达到在$P(x)$中采样的目的,而且在$Q(x)$中采样比较简单。当时如果$Q(x)$和$P(x)$直接的差距太大的话,采样效率会变得很低。

而MCMC方法,我们主要介绍了MH Sampling和Gibbs Sampling,我们主要是通过构建一个马氏链去逼近目标分布,具体的描述将在下一节中展开描述。


\end{document}