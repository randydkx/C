\documentclass[a4paper]{article}
\usepackage[UTF8]{ctex}
\usepackage{geometry}
\usepackage{graphicx}
\usepackage{url}
\usepackage{multirow}
\usepackage{array}
\usepackage{booktabs}
\usepackage{url}
\usepackage{enumitem}
\usepackage{graphicx}
\usepackage{float}
\usepackage{amssymb}
\usepackage{amsmath}
\usepackage{subfig}
\usepackage{longtable}
\usepackage{color}

\geometry{a4paper, scale=0.78}
\title{Linear Classification 01}
\author{Chen Gong}
\date{29 October 2019}

\begin{document}
\maketitle

本节的主要目的是,有关于机器学习的导图。对频率派的有关统计学习方法做一个大致的梳理。而在贝叶斯派的学习中,是使用有关于概率图的模型。在频率派的有关统计学习方法中,我们可以大致的分为,线性回归和线性分类。

\section{线性回归}
在前文中已经提到了,我们的线性回归模型可以写为$f(w,b)=w^Tx+b$。线性回归主要有三条性质:线性,全局性和数据未加工。而我们从每一条入手,打破其中的一条规则就是一个新的算法。

\subsection{线性}
线性可以分为,属性非线性,全局非线性和系数非线性。

\subsubsection{属性非线性}
所谓的属性非线性也就是从未知数入手,比如特征变换的方法还有将变量从一维,变换到高维。有点类似于引入二次型的思想,使用$x_1^2+x_2^2+x_1x_2+\cdots$,的方法打破属性的线性。

\subsubsection{全局非线性}
全局非线性的方法,是通过对函数的运算结果增加一个函数,来将线性函数改造成非线性函数。比如,神经网络中的激活函数,还有阈值函数来将软分类函数变成硬分类函数。

\subsubsection{系数非线性}
所谓系数非线性,感觉就是系数的生成结果并不是单一的,固定的。就像神经网络算法一样。算法的收敛结果是一个分布,也就是位于一个区间之中,这样的算法的结果一定不是线性的,这样通过了不确定的方法来引入非线性。

\subsection{全局性}
所谓全局性,也就是将所有的数据看成一个整体来进行拟合。而打破的方法很简单,也就是将数据之间分隔开,分段进行拟合。典型的方法有线性样条回归,决策树等方法。

\subsection{数据未加工}
从字面的意义上理解非常的简单,那就是输入数据不经过加工直接的输入模型中。有一系列类似的方法来打破,比如主成分分析法(PCA),流形等方法来对输入数据进行预处理。

\section{线性分类}

线性回归和线性分类之间有着很大的联系。从某种意义上说,线性分类就是线性回归函数使用激活函数的结果,同时也可以看成是线性回归降维的结果。对于一个线性回归函数,我们可以通过添加全局函数的形式来将其转换为线性分类函数。也就是
\begin{equation}
    y=w^Tx+b \longrightarrow y=f(w^Tx+b)
\end{equation}
这样就可以将值域从$[0,1]$转换为$\{0,1\}$。其中$f$被定义为activation function,$f^{-1}$定义为link function。那么这个$f$实现了这样一个功能,也就是将$w^Tx+b \mapsto \{0, 1\}$。而$f^{-1}$恰好是反过来的,也就是将$\{0, 1\} \mapsto w^Tx+b$。

而线性分类,大致上可以划分成硬分类和软分类两个部分。

\subsection{硬分类}
所谓硬分类,也就是$y\in [0,1]$,大致上可以分成线性判别分析,也就是Fisher判别分析和感知机这两类。

\subsection{软分类}
所谓硬分类,也就是$y\in \{0,1\}$,大致上可以分成生成式模型,Gaussian Distribution Analysis和著名的判别式模型,Logistic Regression。
\begin{equation}
    p(y|x)=\frac{p(x|y)p(y)}{p(x)} \propto p(x|y)p(y)
\end{equation}

也就是在求解$p(y=0|x)$或$p(y=1|x)$的时候,我们不直接求谁大谁小,而是转向求$p(x|y=0)p(y=0)$和$p(x|y=1)p(y=1)$。

\section{总结}
通过这节的学习,我们已经大体上建立了有关于统计学习方法的知识的框架,包括线性分类和线性回归的内容,并作出了一定的梳理。

\end{document}
