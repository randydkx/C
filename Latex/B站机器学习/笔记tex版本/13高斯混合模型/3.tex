\documentclass[a4paper]{article}
\usepackage[UTF8]{ctex}
\usepackage{geometry}
\usepackage{graphicx}
\usepackage{url}
\usepackage{multirow}
\usepackage{array}
\usepackage{booktabs}
\usepackage{url}
\usepackage{enumitem}
\usepackage{graphicx}
\usepackage{float}
\usepackage{amssymb}
\usepackage{amsmath}
\usepackage{subfig}
\usepackage{longtable}
\usepackage{pifont}
\usepackage{color}

\allowdisplaybreaks

\geometry{a4paper, scale=0.78}

% \begin{figure}[H]
%     \centering
%     \includegraphics[width=.55\textwidth]{E.png}
%     \caption{矩阵与列向量的乘法}
%     \label{fig:my_label_1}
% \end{figure}

% \left\{
% \begin{array}{ll}
%       x+2x+z=2 & \\
%       3x+8y+z=12 & \\
%       4y+z=2
% \end{array}
% \right.

% \begin{enumerate}[itemindent = 1em, itemsep = 0.4pt, parsep=0.5pt, topsep = 0.5pt]

% \end{enumerate}

%\stackrel{a}{\longrightarrow}

%\underbrace{}_{} %下括号

\title{Gaussian Mixture Model 03 Expectation Maximization}
\author{Chen Gong}
\date{25 December 2019}

\begin{document}
\maketitle
上一小节中,我们看到了使用极大似然估计的方法,我们根本就求不出最优参数$\theta$的解析解。所以,我们使用迭代的方法来求近似解。

EM算法的表达式,可以被我们写为:
\begin{equation}
    \theta^{(t+1)} = \arg\max_\theta \underbrace{\mathbb{E}_{P(Z|X,\theta^{(t)})} \left[ \log P(X,Z|\theta) \right]}_{Q(\theta,\theta^{(t)})}
\end{equation}

经过一系列的迭代,我们可以得到$\theta^{0},\theta^{1},\cdots,\theta^{(t)}$,迭代到一定次数以后我们得到的$\theta^{(N)}$就是我们想要得到的结果。EM算法大体上可以分成两个部分,E-step和M-step,

\section{E-Step}
\begin{equation}
    \begin{split}
        Q(\theta,\theta^{(t)}) 
        = & \int_Z \log P(X,Z|\theta)\cdot P(Z|X,\theta^{(t)}) dZ \\
        = & \sum_Z \log \prod_{i=1}^N P(x_i,z_i|\theta)\cdot \prod_{i=1}^N P(z_i|x_i,\theta^{(t)}) dZ \\
        = & \sum_{z_1,\cdots,z_N} \sum_{i=1}^N \log P(x_i,z_i|\theta)\cdot \prod_{i=1}^N P(z_i|x_i,\theta^{(t)}) dZ \\
        = & \sum_{z_1,\cdots,z_N} \left[ \log P(x_1,z_1|\theta) + \log P(x_2,z_2|\theta) + \cdots \log P(x_N,z_N|\theta) \right] \cdot \prod_{i=1}^N P(z_i|x_i,\theta^{(t)}) dZ \\
    \end{split}
\end{equation}

为了简化推导,我们首先只取第一项来化简一下,
\begin{equation}
    \begin{split}
        & \sum_{z_1,\cdots,z_N} \log P(x_1,z_1|\theta) \cdot \prod_{i=1}^N P(z_i|x_i,\theta^{(t)}) dZ \\
        = & \sum_{z_1,\cdots,z_N} \log P(x_1,z_1|\theta) \cdot P(z_1|x_1,\theta^{(t)}) \cdot \prod_{i=2}^N P(z_i|x_i,\theta^{(t)}) dZ \\
        = & \sum_{z_1} \log P(x_1,z_1|\theta) \cdot  P(z_1|x_1,\theta^{(t)}) \cdot \sum_{z_2,\cdots,z_N} \prod_{i=2}^N P(z_i|x_i,\theta^{(t)}) dZ \\
    \end{split}
\end{equation}

而:
\begin{equation}
    \begin{split}
        \sum_{z_2,\cdots,z_N} \prod_{i=2}^N P(z_i|x_i,\theta^{(t)}) 
        = & \sum_{z_2,\cdots,z_N} P(z_2|x_2,\theta^{(t)})\cdot P(z_3|x_3,\theta^{(t)})\cdots P(z_N|x_N,\theta^{(t)}) \\
        = & \sum_{z_2} P(z_2|x_2,\theta^{(t)})\cdot \sum_{z_3} P(z_3|x_3,\theta^{(t)})\cdots \sum_{z_N} P(z_N|x_N,\theta^{(t)}) \\
        = & 1 \cdot 1 \cdots 1 \\
        = & 1 
    \end{split}
\end{equation}

所以,式(3)也就等于:
\begin{equation}
    \begin{split}
        \sum_{z_1,\cdots,z_N} \log P(x_1,z_1|\theta) \cdot \prod_{i=1}^N P(z_i|x_i,\theta^{(t)}) dZ = \sum_{z_1} \log P(x_1,z_1|\theta) \cdot  P(z_1|x_1,\theta^{(t)})
    \end{split}
\end{equation}

将式(5)中得到的结果,代入到式(2)中,我们就可以得到:
\begin{equation}
    \begin{split}
         Q(\theta,\theta^{(t)}) 
        = & \sum_{z_1} \log P(x_1,z_1|\theta) \cdot  P(z_1|x_1,\theta^{(t)}) + \cdots +  \sum_{z_N} \log P(x_N,z_N|\theta) \cdot  P(z_N|x_N,\theta^{(t)}) \\
        = & \sum_{i=1}^N \sum_{Z_i} \log P(x_i,z_i|\theta) \cdot  P(z_i|x_i,\theta^{(t)})
    \end{split}
\end{equation}

那么,下一步我们就是要找到,$P(x_i,z_i|\theta)$和$P(z_i|x_i,\theta^{(t)})$的表达方式了。其中:
\begin{equation}
    P(X,Z) = P(Z)P(X|Z) = P_Z\cdot \mathcal{N}(X|\mu_Z,\Sigma_Z)
\end{equation}
\begin{equation}
    P(Z|X) = \frac{P(X,Z)}{P(X)} = \frac{P_Z\cdot \mathcal{N}(X|\mu_Z,\Sigma_Z)}{\sum_{i=1}^K P_{Zi}\cdot \mathcal{N}(X|\mu_{Zi},\Sigma_{Zi})}
\end{equation}

所以,我们将式(8)代入到式(6)中,就可以得到:
\begin{equation}
     Q(\theta,\theta^{(t)})  =    \sum_{i=1}^N \sum_{Z_i} \log P_{Z_i}\cdot \mathcal{N}(X|\mu_{Z_i},\Sigma_{Z_i}) \cdot \frac{P_{Z_i}^{\theta(t)}\cdot \mathcal{N}(x_i|\mu_{Z_i}^{\theta(t)},\Sigma_{Z_i}^{\theta(t)})}{\sum_{k=1}^K P_k^{\theta(t)}\cdot \mathcal{N}(x_i|\mu_k^{\theta(t)},\Sigma_k^{\theta(t)})}
\end{equation}

\section{M-Step}
根据我们在E-Step中的推导,我们可以得到:
\begin{equation}
    \begin{split}
        Q(\theta,\theta^{(t)})  
        = & \sum_{i=1}^N \sum_{Z_i} \log P_{Z_i}\cdot \mathcal{N}(X|\mu_{Z_i},\Sigma_{Z_i}) \cdot \underbrace{\frac{P_{Z_i}^{\theta(t)}\cdot \mathcal{N}(x_i|\mu_{Z_i}^{\theta(t)},\Sigma_{Z_i}^{\theta(t)})}{\sum_{k=1}^K P_k^{\theta(t)}\cdot \mathcal{N}(x_i|\mu_k^{\theta(t)},\Sigma_k^{\theta(t)})}}_{P(Z_i|X_i,,\theta^{(t)})} \\
        = & \sum_{Z_i} \sum_{i=1}^N \log \left( P_{Z_i}\cdot \mathcal{N}(X|\mu_{Z_i},\Sigma_{Z_i}) \right) \cdot P(Z_i|X_i,,\theta^{(t)}) \\
        = & \sum_{k=1}^K \sum_{i=1}^N \log \left( P_{k}\cdot \mathcal{N}(X|\mu_{k},\Sigma_{k}) \right) \cdot P(Z_i = C_k|X_i,,\theta^{(t)}) \quad (Z_i = C_k) \\
        = & \sum_{k=1}^K \sum_{i=1}^N \left( \log P_{k} + \log  \mathcal{N}(X_i|\mu_{k},\Sigma_{k}) \right) \cdot P(Z_i = C_k|X_i,\theta^{(t)}) \\
    \end{split}
\end{equation}

我们的目的也就是进行不断迭代,从而得出最终的解,用公式表达也就是:
\begin{equation}
    \theta^{(t+1)} = \arg\max_{\theta} Q(\theta,\theta^{(t)})
\end{equation}

我们需要求解的参数也就是,$\theta^{(t+1)}=\{ P_1^{(t+1)}, \cdots, P_k^{(t+1)}, \mu_1^{(t+1)}, \cdots, \mu_k^{(t+1)},\Sigma_1^{(t+1)},\cdots,\Sigma_k^{(t+1)} \}$。

首先,我们来展示一下怎么求解$P_K^{(t+1)}$:

由于在等式(10),$\sum_{k=1}^K \sum_{i=1}^N \left( \log P_{k} + \log  \mathcal{N}(X|\mu_{k},\Sigma_{k}) \right) \cdot P(Z_i = C_k|X_i,,\theta^{(t)})$中的$\log  \mathcal{N}(X|\mu_{k},\Sigma_{k})$部分和$P_k$并没有什么关系。所以,可以被我们直接忽略掉。所以,求解问题,可以被我们描述为:
\begin{equation}
    \left\{
        \begin{array}{ll}
            \arg\max_{P_k} \sum_{k=1}^K \sum_{i=1}^N  \log P_{k} \cdot P(Z_i = C_k|X_i,\theta^{(t)}) & \\
            s.t. \quad \sum_{k=1}^K P_k = 1 & \\
        \end{array}
    \right.
\end{equation}

使用拉格朗日算子法,我们可以写成:
\begin{equation}
    \mathcal{L}(P,\lambda) = \sum_{k=1}^K \sum_{i=1}^N  \log P_{k} \cdot P(Z_i = C_k|X_i,\theta^{(t)}) + \lambda(\sum_{k=1}^K P_k - 1)
\end{equation}

\begin{equation}
    \begin{split}
        \frac{\partial \mathcal{L}(P,\lambda)}{\partial P_k} = & \sum_{i=1}^N \frac{1}{P_k} \cdot P(Z_i = C_k|X_i,\theta^{(t)}) + \lambda = 0  \\
        \Rightarrow & \sum_{i=1}^N P(Z_i = C_k|X_i,\theta^{(t)}) + P_k \lambda = 0 \\
        \stackrel{k = 1,\cdots,K}{\Longrightarrow} & \sum_{i=1}^N\underbrace{\sum_{k=1}^K P(Z_i = C_k|X_i,\theta^{(t)})}_{1} + \underbrace{\sum_{k=1}^K P_k}_{1} \lambda = 0 \\
        \Rightarrow & N+\lambda = 0 
    \end{split}
\end{equation}

所以,我们可以轻易的得到$\lambda = -N$,所以有
\begin{equation}
    P_K^{(t+1)} = \frac{1}{N} \sum_{i=1}^N P(Z_i = C_k | X_i,\theta^{(t)})
\end{equation}

那么,我们所有想要求的参数也就是$P^{(t+1)} = (P_1^{(t+1)},P_2^{(t+1)},\cdots,P_k^{(t+1)})$。

求解$P_k^{(t+1)}$是一个有约束的求最大值问题,由于带约束所以我们要使用拉格朗日乘子法。而且这里使用到了一个track,也就是将从1到k,所有的数据集做一个整合,非常的精彩,这样就直接消掉了$P_k$无法计算的问题。而至于$\theta$的其他部分,也就是关于$\{ \mu_1^{(t+1)}, \cdots, \mu_k^{(t+1)},\Sigma_1^{(t+1)},\cdots,\Sigma_k^{(t+1)} \}$的计算,使用的方法也是一样的,这个问题就留给各位了。

为什么极大似然估计搞不定的问题,放在EM算法里面我们就可以搞定了呢?我们来对比一下两个方法中,需要计算极值的公式。
\begin{equation}
    \sum_{k=1}^K \sum_{i=1}^N \left( \log P_{k} + \log  \mathcal{N}(X_i|\mu_{k},\Sigma_{k}) \right) \cdot P(Z_i = C_k|X_i,\theta^{(t)})
\end{equation}
\begin{equation}
    \arg\max_{\theta}  \sum_{i=1}^N  \log \sum_{k=1}^K P_k \cdot \mathcal{N}(x_i|\mu_k,\Sigma_k)
\end{equation}

极大似然估计一开始计算的就是$P(X)$,而EM算法中并没有出现有关$P(X)$的计算,而是全程计算都是$P(X,Z)$。而$P(X)$实际上就是$P(X,Z)$的求和形式。所以,每次单独的考虑$P(X,Z)$就避免了在log函数中出现求和操作。

\end{document}

