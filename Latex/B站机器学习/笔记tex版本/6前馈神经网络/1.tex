\documentclass[a4paper]{article}
\usepackage[UTF8]{ctex}
\usepackage{geometry}
\usepackage{graphicx}
\usepackage{url}
\usepackage{multirow}
\usepackage{array}
\usepackage{booktabs}
\usepackage{url}
\usepackage{enumitem}
\usepackage{graphicx}
\usepackage{float}
\usepackage{amssymb}
\usepackage{amsmath}
\usepackage{subfig}
\usepackage{longtable}
\usepackage{pifont}
\usepackage{color}

\allowdisplaybreaks

\geometry{a4paper, scale=0.78}

% \begin{figure}[H]
%     \centering
%     \includegraphics[width=.55\textwidth]{E.png}
%     \caption{矩阵与列向量的乘法}
%     \label{fig:my_label_1}
% \end{figure}

% \left\{
% \begin{array}{ll}
%       x+2x+z=2 & \\
%       3x+8y+z=12 & \\
%       4y+z=2
% \end{array}
% \right.

% \begin{enumerate}[itemindent = 1em, itemsep = 0.4pt, parsep=0.5pt, topsep = 0.5pt]

% \end{enumerate}


\title{Feedforward Neural Network 01 Background}
\author{Chen Gong}
\date{10 November 2019}

\begin{document}
\maketitle

本节的主要目的是从一个较高的角度来介绍一下,什么是深度学习,并且给深度学习一个较好的总结,给大家一个较好的印象。机器学习是目前最火热的一个研究方向,而机器学习大致可以分为,频率派和贝叶斯派。频率派逐渐演变出了统计机器学习,而贝叶斯派逐渐演变出了PGM,也就是概率图模型。下面我们分开进行描述。

\section{频率派}
统计机器学习方法基本就是由频率派的估计思想得到的。统计机器学习方法大概可以分成四种。

1. 正则化:$L_1,L_2$也就是之前提到的Lasso和岭回归,这实际上并没有产生新的模型,而是在之前模型的基础上进行了改进。我们可以把它描述为Loss function + regularized。用来抑制训练的过拟合。

2. 核化:最著名的就是我们之前提到的,Kernel Support Vector Machine (SVM)了。

3. 集成化:也就是Adaboost和Random Forest。

4. 层次化:层次化主要就是我们指的Neural Network,也就是神经网络,神经网络进一步发展就得到了我们现在研究的深度学习。而神经网络中比较著名的几类就是:1. 多层感知机(Multiple Layer Perception);2. Auto-encode;3. CNN;4. RNN。这几个组合起来就是我们经常听到的Deep Network。

\section{贝叶斯派}
贝叶斯派的估计方法就演化得到了概率图模型(Probability Graphic Model。他们大致可以分成以下三类:

1. 有向图:Bayesian Network,也就是Deep Directed Network,包括大家听得很多的:Variable Automation Encode (VAE),Generative Adversarial Network (GAN)和Sigmoid Belief Network等等。

2. 无向图:Markov Network,也就是Deep Boltzmann Modeling,这就是我们的第二类图模型。

3. 有向图和无向图混合在一起,就是我们常说的Mixed Network,主要包括,Deep Belief Network等等。

而上述几个图模型,结合起来就是我们常说的Deep Generative Network,深度生成模型。

在我们狭义的深度学习的理解中,什么是深度学习,实际上就是统计学习方法中的层次化中的Deep Network。而广义的深度学习中,还应该包括,Deep Generative Network。而实际上绝大多数的深度学习者都不太了解Deep Generative Network,确实涉及到贝叶斯的理论,深度学习就会变得很难。而且它的训练也会变得非常的复杂。


\end{document}
