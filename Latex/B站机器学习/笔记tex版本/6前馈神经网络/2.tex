\documentclass[a4paper]{article}
\usepackage[UTF8]{ctex}
\usepackage{geometry}
\usepackage{graphicx}
\usepackage{url}
\usepackage{multirow}
\usepackage{array}
\usepackage{booktabs}
\usepackage{url}
\usepackage{enumitem}
\usepackage{graphicx}
\usepackage{float}
\usepackage{amssymb}
\usepackage{amsmath}
\usepackage{subfig}
\usepackage{longtable}
\usepackage{pifont}
\usepackage{color}

\allowdisplaybreaks

\geometry{a4paper, scale=0.78}

% \begin{figure}[H]
%     \centering
%     \includegraphics[width=.55\textwidth]{E.png}
%     \caption{矩阵与列向量的乘法}
%     \label{fig:my_label_1}
% \end{figure}

% \left\{
% \begin{array}{ll}
%       x+2x+z=2 & \\
%       3x+8y+z=12 & \\
%       4y+z=2
% \end{array}
% \right.

% \begin{enumerate}[itemindent = 1em, itemsep = 0.4pt, parsep=0.5pt, topsep = 0.5pt]

% \end{enumerate}

\title{Feedforward Neural Network 02 Development}
\author{Chen Gong}
\date{11 November 2019}

\begin{document}
\maketitle
本节主要是来讨论一下,机器学习的发展历史,看看如何从感知机到深度学习。
\section{从时间的发展角度来看}

1958年:up,首次提出了Perceptron Linear Algorithm (PLA),这里就是我们机器学习的开端了。

1969年:down,Marvin Lee Minsky提出了,PLA has a limitation。因为PLA算法解决不了non-linear问题,比如说XOR问题。非常戏剧的是,这一年,Marvin Lee Minsky获得了图灵奖,他也是“人工智能之父”,第一位因为AI而获得图灵奖的科学家。

1981年:up,学者提出了Multiple Layer Perceptron (MLP),可以用来解决非线性的问题,就是是最初的Feedforward Neural Network。

1986年:up,Hinton提出了将Back Propagation (BP)算法和MLP完美的融合在了一起,并且发展出了Recurrent Neural Network (RNN)算法。

1989年:up,提出了CNN。但是也迎来了人工智能的寒冬。down,在这一年中提出了一个Universal Apposhmation theorem,也就是一个大于1层的Hidden Layer就可以用来拟合任何的连续函数。那么这是就提出了一个疑问:1 layer is OK,why deep?并且,在BP算法中,随着深度的增加还会出现梯度消失的问题。

1993年和1995年,down,这一年中Support Vector Machine (SVM) + Kernel + Theory,获得了很好的效果。并且,Adaboost和Rondom Forest等Ensemble algorithm流派的提出,获得了很好的效果。

1997年,up,提出了LSTM,但是远不足以止住深度学习发展的颓势。

2006年,up,Hinton,提出了Deep Belief Network (RBM)和Deep Auto-encoder。

2009年,up,GPU的飞速发展。

2011年,up,Deep Learning运用到了语音(Speech)中。

2012年,up,斯坦福大学李飞飞教授,开办了一个非常重要的比赛和数据库ImageNet。

2013年,up,提出了Variational Automation Encode (VAE)算法。

2015年,up,提出了非常重要的Generative Adversarial Network (GAN)。

2016年,up,围棋上AlphaGo彻底引爆了Deep Learning。

2018年,up,提出了重要的Graphic Neural Network (GNN),传统的神经网络是连接主义的,而GNN中将符合主义和连接主义进行了联合,使之具有推理的功能。

\section{总结}
其实Deep Learning的崛起是很多因素融合的结果。这些年来,主要都是在实践上的发展,而在机器学习理论上基本没有什么进步。它的发展得益于以下几点:1. data的则增加,big data 时代的到来;2. 分布式计算的发展;3. 硬件水平的发展。其实最主要的说白了就是效果,效果比SVM要更好,就占据了主要的地位。

计算机学科就是一门实践为主的科学,现在在实际上取得了很好的效果。之后随着理论研究的不断深入,我们一定可以不断的完善理论知识。之后AI方向的研究,也将是以深度学习为主流,而其他机器学习学派的知识和优点将不断地丰富深度学习,扩充深度学习,来给它更强大的效果。


\end{document}
