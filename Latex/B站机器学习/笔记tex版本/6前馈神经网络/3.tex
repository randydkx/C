\documentclass[a4paper]{article}
\usepackage[UTF8]{ctex}
\usepackage{geometry}
\usepackage{graphicx}
\usepackage{url}
\usepackage{multirow}
\usepackage{array}
\usepackage{booktabs}
\usepackage{url}
\usepackage{enumitem}
\usepackage{graphicx}
\usepackage{float}
\usepackage{amssymb}
\usepackage{amsmath}
\usepackage{subfig}
\usepackage{longtable}
\usepackage{pifont}
\usepackage{color}

\allowdisplaybreaks

\geometry{a4paper, scale=0.78}

% \begin{figure}[H]
%     \centering
%     \includegraphics[width=.55\textwidth]{E.png}
%     \caption{矩阵与列向量的乘法}
%     \label{fig:my_label_1}
% \end{figure}

% \left\{
% \begin{array}{ll}
%       x+2x+z=2 & \\
%       3x+8y+z=12 & \\
%       4y+z=2
% \end{array}
% \right.

% \begin{enumerate}[itemindent = 1em, itemsep = 0.4pt, parsep=0.5pt, topsep = 0.5pt]

% \end{enumerate}

%\stackrel{a}{\longrightarrow}

\title{Feedforward Neural Network 03 Non-Linear Problem}
\author{Chen Gong}
\date{12 November 2019}

\begin{document}
\maketitle
实际上在1958年就已经成功的提出了Perceptron Linear Analysis (PLA),标志着人工智能的正式诞生。但是,Minsky在1969年提出PLA无法解决非线性分类问题,让人工智能陷入了10年的低谷。后来的发展,人们开始寻找到越来越多的,解决非线性分类问题的方法。于是,我们提出了三种解决非线性问题的方法。

\section{Non-Transformation}
这实际上就是一种明转换,将向量从input space转换到feature space,可以写做$\phi = \mathcal{X}\longmapsto\mathcal{Z}$。在Conver’s theory中提出,高维空间比低维空间更加容易线性可分。很显然对于一个异或问题(XOR)来说,我们将$x=(x_1,x_2)\stackrel{\phi}{\longrightarrow}z=(x_1,x_2,(x_1-x_2)^2)$
\begin{equation}
    \begin{matrix}
        0 & 1 & \longrightarrow & 1 \\
        1 & 0 & \longrightarrow & 1 \\
        1 & 1 & \longrightarrow & 0 \\
        0 & 0 & \longrightarrow & 0 \\
    \end{matrix}
    \quad
    \stackrel{\phi}{\longrightarrow}
    \quad
    \begin{matrix}
        0 & 1 & 1 & \longrightarrow & 1 \\
        1 & 0 & 1 & \longrightarrow & 1 \\
        1 & 1 & 0 & \longrightarrow & 0 \\
        0 & 0 & 0 & \longrightarrow & 0 \\
    \end{matrix}
\end{equation}

很显然在三维空间中,进行空间映射后,就会变得比较容易进行线性划分了。
可以自己画图来进行验证,这里不再作图。

\section{Kernel Method}
这实际上是一种暗转的思路,也就是令$K(x,x')=<\phi(x),\phi(x')>$,在这个核函数中实际上隐藏了一个$\phi$,而$x,x'\in \mathcal{X}$。

\section{Neural Network}
神经网络算法实际上就是一个Multit-Layer Perceptron (MLP),有时也会被称为,Feedforward Neural Network (FNN),所以大家在其他书上见到这几种描述都不要感到意外。我们以XOR (位运算)为例吧。在我们的逻辑运算中,大致有四种运算方法。
\begin{equation}
    \begin{matrix}
        XOR & OR & AND & NOT \\
        \oplus & \vee & \wedge & \urcorner \\
    \end{matrix}
\end{equation}

而后三种运算为基础运算,因为异或运算实际上是可以由后三种运算组成,也就是$x_1\oplus x_2 = (\urcorner x_1 \wedge x_2) \vee (x_1 \wedge \urcorner x_2) $。实际上就是先做两个与运算,然后做一个或运算。把一个线性不可分的东西来分层实现,将特征空间进行了分解而已。然后,在分层运算中插入了激活函数,来达到非线性映射的效果。这部分内容,比较的简单,而且网上也有大量的资料,此处就不再做过多的阐述。

实际上神经网络就是一个有向无环图。所以,某种意义上说可以引入概率图的模型,当然这就是后话了。

\end{document}
