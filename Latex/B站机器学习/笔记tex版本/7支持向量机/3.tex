\documentclass[a4paper]{article}
\usepackage[UTF8]{ctex}
\usepackage{geometry}
\usepackage{graphicx}
\usepackage{url}
\usepackage{multirow}
\usepackage{array}
\usepackage{booktabs}
\usepackage{url}
\usepackage{enumitem}
\usepackage{graphicx}
\usepackage{float}
\usepackage{amssymb}
\usepackage{amsmath}
\usepackage{subfig}
\usepackage{longtable}
\usepackage{pifont}
\usepackage{color}

\allowdisplaybreaks

\geometry{a4paper, scale=0.78}

% \begin{figure}[H]
%     \centering
%     \includegraphics[width=.55\textwidth]{E.png}
%     \caption{矩阵与列向量的乘法}
%     \label{fig:my_label_1}
% \end{figure}

% \left\{
% \begin{array}{ll}
%       x+2x+z=2 & \\
%       3x+8y+z=12 & \\
%       4y+z=2
% \end{array}
% \right.

% \begin{enumerate}[itemindent = 1em, itemsep = 0.4pt, parsep=0.5pt, topsep = 0.5pt]

% \end{enumerate}

%\stackrel{a}{\longrightarrow}

\title{Support Vector Machine 03 Weak Duality Proof}
\author{Chen Gong}
\date{16 November 2019}

\begin{document}
\maketitle
在前面我们已经展示的Hard Margin和Soft Margin SVM的建模和求解。前面提到的SVM有三宝,间隔,对偶,核技巧。前面我们已经分析了间隔,大家对于其中用到的对偶,虽然我们用比较直觉性的方法进行了解释,但是估计大家还是有点懵逼。这节我们希望给到通用性的证明,这里实际上就是用到了约束优化问题。

\section{弱对偶性证明}
首先,我们需要证明约束优化问题的原问题和无约束问题之间的等价性。
\subsection{约束优化问题与无约束问题的等价性}
对于一个约束优化问题,我们可以写成:
\begin{equation}
    \left\{
    \begin{array}{ll}
      \min_{x\in \mathcal{R}^p}f(x) & \\
      s.t. \quad m_i(x) \leq 0,\ i = 1,2,\cdots,N & \\
      \quad \ \; \quad n_i(x) = 0,\ i = 1,2,\cdots,N & \\
    \end{array}
    \right.
\end{equation}

我们用拉格朗日函数来进行表示:
\begin{equation}
    \mathcal{L}(x,\lambda,\eta) = f(x) + \sum_{i=1}^N\lambda_im_i + \sum_{i=1}^N\eta_in_i
\end{equation}

我们可以等价的表示为:
\begin{equation}
    \left\{
    \begin{array}{ll}
      \min_{x}\max_{\lambda,\eta}\  \mathcal{L}(x,\lambda,\eta) & \\
      s.t. \ \lambda_i \geq 0,\ i = 1,2,\cdots,N & \\
    \end{array}
    \right.
\end{equation}

这就是将一个约束优化问题的原问题转换为无约束问题。那么这两种表达形式一定是等价的吗?我们可以来分析一下:

如果,$x$违反了约束条件$m_i(x) \leq 0$,那么有,$m_i(x) > 0$。且$\lambda_i>0$,那么很显然$max_{\lambda}\ \mathcal{L} = + \infty$。

如果,$x$符合约束条件$m_i(x) \leq 0$,那么很显然$max_{\lambda}\ \mathcal{L} \neq + \infty$。

那么:
\begin{equation}
    \min_{x} \max_{\lambda,\eta} \ \mathcal{L}(x,\lambda,\eta) = \min_{x}\left\{ \max\ \mathcal{L}, +\infty \right\} = \min_{x}\left\{ \max\ \mathcal{L} \right\}
\end{equation}

其实大家可以很明显的感觉到,这个等式自动的帮助我们过滤到了一半$m_i(x) \geq 0$的情况,这实际上就是一个隐含的约束条件,帮我们去掉了一部分不够好的解。

\subsection{证明弱对偶性}
原问题我们可以写为:
\begin{equation}
    \left\{
    \begin{array}{ll}
      \min_{x}\max_{\lambda,\eta}\  \mathcal{L}(x,\lambda,\eta) & \\
      s.t. \ \lambda_i \geq 0,\ i = 1,2,\cdots,N & \\
    \end{array}
    \right.
\end{equation}

而原问题的对偶问题则为:
\begin{equation}
    \left\{
    \begin{array}{ll}
      \min_{\lambda,\eta}\max_{x}\  \mathcal{L}(x,\lambda,\eta) & \\
      s.t. \ \lambda_i \geq 0,\ i = 1,2,\cdots,N & \\
    \end{array}
    \right.
\end{equation}

原问题是一个关于$x$的函数,而对偶问题是一个关于$\lambda,\eta$的最小化问题,而弱对偶性则可以描述为:对偶问题的解$\leq$原问题的解。为了简化表达,后面对偶问题的解我们用$d$来表示,而原问题的解我们用$p$来表示。那么我们用公式化的语言表达也就是:
\begin{equation}
     \min_{\lambda,\eta}\max_{x}\  \mathcal{L}(x,\lambda,\eta) = d \leq  \min_{x}\max_{\lambda,\eta}\  \mathcal{L}(x,\lambda,\eta) = p
\end{equation}

在前面我们使用感性的方法证明了$\max \min \ \mathcal{L} \leq \min \max \ \mathcal{L}$,下面我们给出严谨的证明:

很显然可以得到:
\begin{equation}
    \min_{x}\ \mathcal{L}(x,\lambda,\eta) \leq \mathcal{L}(x,\lambda,\eta) \leq \max_{\lambda,\eta}\ \mathcal{L}(x,\lambda,\eta)
\end{equation}

那么,$\min_{x}\ \mathcal{L}(x,\lambda,\eta)$可表示为一个与$x$无关的函数$A(\lambda,\eta)$,同理$\max_{\lambda,\eta}\ \mathcal{L}(x,\lambda,\eta)$可表示为一个与$\lambda,\eta$无关的函数$B(x)$。显然,我们可以得到一个恒等式:
\begin{equation}
    A(\lambda,\eta) \leq B(x)
\end{equation}

那么接下来就有:
\begin{equation}
    \begin{split}
        A(\lambda,\eta) \leq & \min \ B(x) \\
        \max \ A(\lambda,\eta) \leq & \min \ B(x) \\
         \min_{\lambda,\eta}\max_{x}\  \mathcal{L}(x,\lambda,\eta) \leq & \min_{x}\max_{\lambda,\eta}\  \mathcal{L}(x,\lambda,\eta) 
    \end{split}
\end{equation}

弱对偶性,证毕!!

\end{document}
 