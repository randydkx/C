\documentclass[a4paper]{article}
\usepackage[UTF8]{ctex}
\usepackage{geometry}
\usepackage{graphicx}
\usepackage{url}
\usepackage{multirow}
\usepackage{array}
\usepackage{booktabs}
\usepackage{url}
\usepackage{enumitem}
\usepackage{graphicx}
\usepackage{float}
\usepackage{amssymb}
\usepackage{amsmath}
\usepackage{subfig}
\usepackage{longtable}
\usepackage{pifont}
\usepackage{color}

\allowdisplaybreaks

\geometry{a4paper, scale=0.78}

% \begin{figure}[H]
%     \centering
%     \includegraphics[width=.55\textwidth]{E.png}
%     \caption{矩阵与列向量的乘法}
%     \label{fig:my_label_1}
% \end{figure}

% \left\{
% \begin{array}{ll}
%       x+2x+z=2 & \\
%       3x+8y+z=12 & \\
%       4y+z=2
% \end{array}
% \right.

% \begin{enumerate}[itemindent = 1em, itemsep = 0.4pt, parsep=0.5pt, topsep = 0.5pt]

% \end{enumerate}

%\stackrel{a}{\longrightarrow}

\title{Variational Inference 04 Stochastic Gradient Variational Inference}
\author{Chen Gong}
\date{01 December 2019}

\begin{document}
\maketitle
在上一小节中,我们分析了Mean Field Theory Variational Inference,通过平均假设来得到变分推断的理论,是一种classical VI,我们可以将其看成Coordinate Ascend。而另一种方法是Stochastic Gradient Variational Inference (SGVI)。

对于隐变量参数$z$和数据集$x$。$z \longrightarrow x$是Generative Model,也就是$p(x|z)$和$p(x,z)$,这个过程也被我们称为Decoder。$x \longrightarrow z$是Inference Model,这个过程被我们称为Encoder,表达关系也就是$p(z|x)$。

\section{SGVI参数规范}
我们这节的主题就是Stochastic Gradient Variational Inference (SGVI),参数的更新方法为:
\begin{equation}
    \theta^{(t+1)} = \theta^{(t)} + \lambda^{(t)}\nabla \mathcal{L}(q)
\end{equation}

其中,$q(z|x)$被我们简化表示为$q(z)$,我们令$q(z)$是一个固定形式的概率分布,$\phi$为这个分布的参数,那么我们将把这个概率写成$q_{\phi}(z)$。

那么,我们需要对原等式中的表达形式进行更新,
\begin{equation}
    ELBO = \mathbb{E}_{q_{\phi}(z)}\left[ \log p_{\theta}(x^{(i)},z) - \log q_{\phi}(z) \right] = \mathcal{L}(\phi)
\end{equation}

而,
\begin{equation}
    \log p_{\theta}(x^{(i)}) = ELBO + KL(q||p) \geq \mathcal{L}(\phi)
\end{equation}

而求解目标也转换成了:
\begin{equation}
    \hat{p} = \arg\max_{\phi} \mathcal{L}(\phi)
\end{equation}

\section{SGVI的梯度推导}
\begin{equation}
    \begin{split}
        \nabla_{\phi} \mathcal{L}(\phi)
        = & \nabla_{\phi} \mathbb{E}_{q_{\phi}}\left[ \log p_{\theta}(x^{(i)},z) - \log q_{\phi} \right] \\
        = & \nabla_{\phi} \int q_{\phi}\left[ \log p_{\theta}(x^{(i)},z) - \log q_{\phi} \right]dz \\
         = &  \int \nabla_{\phi} q_{\phi}\left[ \log p_{\theta}(x^{(i)},z) - \log q_{\phi} \right]dz + 
         \int q_{\phi}\nabla_{\phi} \left[ \log p_{\theta}(x^{(i)},z) - \log q_{\phi} \right]dz \\
    \end{split}
\end{equation}

我们把这个等式拆成两个部分,其中:

$\int \nabla_{\phi} q_{\phi}\left[ \log p_{\theta}(x^{(i)},z) - \log q_{\phi} \right]dz$为第一个部分;

$ \int q_{\phi}\nabla_{\phi} \left[ \log p_{\theta}(x^{(i)},z) - \log q_{\phi} \right]dz$为第二个部分。

\subsection{关于第二部分的求解}
第二部分比较好求,所以我们才首先求第二部分的,哈哈哈!因为$\log p_{\theta}(x^{(i)},z)$与$\phi$无关。
\begin{equation}
    \begin{split}
        2 
        = & \int q_{\phi}\nabla_{\phi} \left[ \log p_{\theta}(x^{(i)},z) - \log q_{\phi} \right]dz \\
        = & -\int q_{\phi}\nabla_{\phi}\log q_{\phi} dz \\
        = & -\int q_{\phi} \frac{1}{q_{\phi}}\nabla_{\phi} q_{\phi} dz \\
        = & -\int \nabla_{\phi} q_{\phi} dz \\
        = & - \nabla_{\phi} \int q_{\phi} dz \\
        = & - \nabla_{\phi} 1 \\
        = & 0
    \end{split}
\end{equation}

\subsection{关于第一部分的求解}
在这里我们用到了一个小trick,这个trick在公式(6)的推导中,我们使用过的。那就是$\nabla_{\phi} q_{\phi} = q_{\phi}\nabla_{\phi}\log q_{\phi} $。所以,我们代入到第一项中可以得到:
\begin{equation}
    \begin{split}
        1 
        = & \int \nabla_{\phi} q_{\phi}\left[ \log p_{\theta}(x^{(i)},z) - \log q_{\phi} \right]dz \\
        = & \int q_{\phi}\nabla_{\phi}\log q_{\phi} \left[ \log p_{\theta}(x^{(i)},z) - \log q_{\phi} \right]dz \\
        = & \mathbb{E}_{q_{\phi}} \left[ \nabla_{\phi}\log q_{\phi} \log p_{\theta}(x^{(i)},z) - \log q_{\phi} \right] 
    \end{split}
\end{equation}

那么,我们可以得到:
\begin{equation}
    \nabla_{\phi} \mathcal{L}(\phi) = \mathbb{E}_{q_{\phi}} \left[ \nabla_{\phi}\log q_{\phi} \log p_{\theta}(x^{(i)},z) - \log q_{\phi} \right] 
\end{equation}

那么如何求这个期望呢?我们采用的是蒙特卡罗采样法,假设$z^l \sim q_{\phi} (z)\ l = 1, 2, \cdots, L$,那么有:
\begin{equation}
    \nabla_{\phi} \mathcal{L}(\phi) \approx \frac{1}{L} \sum_{l=1}^L \nabla_{\phi}\log q_{\phi}(z^{(l)})\left[ \log p_{\theta}(x^{(i)},z) - \log q_{\phi}(z^{(l)})\right]
\end{equation}

~\\

由于第二部分的结果为0,所以第一部分的解就是最终的解。但是,这样的求法有什么样的问题呢?因为我们在采样的过程中,很有可能采到$q_{\phi}(z) \longrightarrow 0$的点,对于log函数来说,$\lim_{x\longrightarrow 0}\log x = \infty$,那么梯度的变化会非常的剧烈,非常的不稳定。对于这样的High Variance的问题,根本没有办法求解。实际上,我们可以通过计算得到这个方差的解析解,它确实是一个很大的值。事实上,这里的梯度的方差这么的大,而$\hat{\phi} \longrightarrow q(z)$也有误差,误差叠加,直接爆炸,根本没有办法用。也就是不会work,那么我们如何解决这个问题?

\section{Variance Reduction}
这里采用了一种比较常见的方差缩减方法,称为Reparameterization Trick,也就是对$q_{\phi}$做一些简化。

我们怎么可以较好的解决这个问题?如果我们可以得到一个确定的解$p(\epsilon)$,就会变得比较简单。因为$z$来自于$q_{\phi}(z|x)$,我们就想办法将z中的随机变量给解放出来。也就是使用一个转换$z = g_{\phi}(\epsilon, x^{(i)})$,其中$\epsilon \sim p(\epsilon)$。那么这样做,有什么好处呢?原来的$\nabla_{\phi} \mathbb{E}_{q_{\phi}}[\cdot]$将转换为$ \mathbb{E}_{p(\epsilon)}[\nabla_{\phi}(\cdot)]$,那么不在是连续的关于$\phi$的采样,这样可以有效的降低方差。并且,$z$是一个关于$\epsilon$的函数,我们将随机性转移到了$\epsilon$,那么问题就可以简化为:
\begin{equation}
    z \sim q_{\phi}(z|x^{(i)}) \longrightarrow \epsilon \sim p(\epsilon)
\end{equation}

而且,这里还需要引入一个等式,那就是:
\begin{equation}
    |q_{\phi}(z|x^{(i)})dz| = |p(\epsilon)d\epsilon|
\end{equation}

为什么呢?我们直观性的理解一下,$\int q_{\phi}(z|x^{(i)})dz = 
\int p(\epsilon)d\epsilon = 1$,并且$q_{\phi}(z|x^{(i)})$和$p(\epsilon)$之间存在一个变换关系。

那么,我们将改写$\nabla_{\phi} \mathcal{L}(\phi)$:
\begin{equation}
    \begin{split}
        \nabla_{\phi} \mathcal{L}(\phi) 
        = & \nabla_{\phi} \mathbb{E}_{q_{\phi}}\left[ \log p_{\theta}(x^{(i)},z) - \log q_{\phi} \right] \\
        = & \nabla_{\phi} \int \left[ \log p_{\theta}(x^{(i)},z) - \log q_{\phi} \right]q_{\phi} dz \\
        = & \nabla_{\phi} \int \left[ \log p_{\theta}(x^{(i)},z) - \log q_{\phi} \right]p(\epsilon) d\epsilon \\
         = & \nabla_{\phi} \mathbb{E}_{p(\epsilon)}\left[ \log p_{\theta}(x^{(i)},z) - \log q_{\phi} \right] \\
         = & \mathbb{E}_{p(\epsilon)} \nabla_{\phi} \left[( \log p_{\theta}(x^{(i)},z) - \log q_{\phi}) \right] \\
         = & \mathbb{E}_{p(\epsilon)}\nabla_{z}\left[( \log p_{\theta}(x^{(i)},z) - \log q_{\phi}(z|x^{(i)}))\nabla_{\phi}z \right] \\
         = & \mathbb{E}_{p(\epsilon)}\nabla_{z}\left[( \log p_{\theta}(x^{(i)},z) - \log q_{\phi}(z|x^{(i)}))\nabla_{\phi}z \right] \\
         = & \mathbb{E}_{p(\epsilon)}\nabla_{z}\left[( \log p_{\theta}(x^{(i)},z) - \log q_{\phi}(z|x^{(i)}))\nabla_{\phi}g_{\phi}(\epsilon, x^{(i)}) \right]
    \end{split}
\end{equation}

那么我们的问题就这样愉快的解决了,$p(\epsilon)$的采样与$\phi$无关,然后对先求关于$z$的梯度,然后再求关于$\phi$的梯度,那么这三者之间就互相隔离开了。最后,我们再对结果进行采样,$\epsilon^{(l)} \sim p(\epsilon), \quad l = 1, 2, \cdots, L$:
\begin{equation}
    \nabla_{\phi} \mathcal{L}(\phi) \approx \frac{1}{L} \sum_{i=1}^L
    \nabla_{z} \left[ (\log p_{\theta}(x^{(i)},z) - \log q_{\phi}(z|x^{(i)}))\nabla_{\phi}g_{\phi}(\epsilon, x^{(i)}) \right]
\end{equation}

其中$z \longleftarrow g_{\phi}(\epsilon^{(i)},x^{(i)})$。而SGVI为:
\begin{equation}
    \phi^{(t+1)} \longrightarrow \phi^{(t)} + \lambda^{(t)}\nabla_{\phi} \mathcal{L}(\phi)
\end{equation}

\section{小结}
那么SGVI,可以简要的表述为:我们定义分布为$q_{\phi}(Z|X)$,$\phi$为参数,参数的更新方法为:
\begin{equation}
    \phi^{(t+1)} \longrightarrow \phi^{(t)} + \lambda^{(t)}\nabla_{\phi} \mathcal{L}(\phi)
\end{equation}

$\nabla_{\phi} \mathcal{L}(\phi)$为:
\begin{equation}
    \nabla_{\phi} \mathcal{L}(\phi) \approx \frac{1}{L} \sum_{i=1}^L
    \nabla_{z} \left[ \log p_{\theta}(x^{(i)},z) - \log q_{\phi}(z|x^{(i)}))\nabla_{\phi}g_{\phi}(\epsilon, x^{(i)}) \right]
\end{equation}










































\end{document}
