\documentclass[a4paper]{article}
\usepackage[UTF8]{ctex}
\usepackage{geometry}
\usepackage{graphicx}
\usepackage{url}
\usepackage{multirow}
\usepackage{array}
\usepackage{booktabs}
\usepackage{url}
\usepackage{enumitem}
\usepackage{graphicx}
\usepackage{float}
\usepackage{amssymb}
\usepackage{amsmath}
\usepackage{subfig}
\usepackage{longtable}
\usepackage{pifont}
\usepackage{color}

\allowdisplaybreaks

\geometry{a4paper, scale=0.78}

% \begin{figure}[H]
%     \centering
%     \includegraphics[width=.55\textwidth]{E.png}
%     \caption{矩阵与列向量的乘法}
%     \label{fig:my_label_1}
% \end{figure}

% \left\{
% \begin{array}{ll}
%       x+2x+z=2 & \\
%       3x+8y+z=12 & \\
%       4y+z=2
% \end{array}
% \right.

% \begin{enumerate}[itemindent = 1em, itemsep = 0.4pt, parsep=0.5pt, topsep = 0.5pt]

% \end{enumerate}

%\stackrel{a}{\longrightarrow}

\title{Variational Inference 02 Algorithm}
\author{Chen Gong}
\date{30 November 2019}

\begin{document}
\maketitle
我们将$X$:Observed data;$Z$:Latent Variable + Parameters。那么$(X,Z)$为complete data。根据我们的贝叶斯分布公式:
\begin{equation}
    p(X) = \frac{p(X,Z)}{p(Z|X)}
\end{equation}

在两边同时取对数并且引入函数$q(Z)$我们可以得到:
\begin{equation}
    \begin{split}
        \log p(X) = & \log \frac{p(X,Z)}{p(Z|X)} \\
        = & \log p(X,Z) - \log p(Z|X) \\
        = & \log\frac{p(X,Z)}{q(Z)} - \log \frac{p(Z|X)}{q(Z)} \\
    \end{split}
\end{equation}

\section{公式化简}
左边 = $p(X)$ = $\int_{Z}log\ p(X)q(Z)dZ$。

右边 = 
\begin{equation}
    \int_Z q(Z)\log\ \frac{p(X,Z)}{q(Z)}dZ - \int_Z q(Z)\log\ \frac{p(Z|X)}{q(Z)}dZ
\end{equation}

其中,$\int_Z q(Z)\log\ \frac{p(X,Z)}{q(Z)}dZ$被称为Evidence Lower Bound (ELBO),被我们记为$\mathcal{L}(q)$,也就是变分。

$- \int_Z q(Z)\log\ \frac{p(Z|X)}{q(Z)}dZ$被称为$KL(q||p)$。这里的$KL(q||p) \geq 0$。

由于我们求不出$p(Z|X)$,我们的目的是寻找一个$q(Z)$,使得$p(Z|X)$近似于$q(Z)$,也就是$KL(q||p)$越小越好。并且,$p(X)$是个定值,那么我们的目标变成了$\arg\max_{q(z)}\mathcal{L}(q)$。那么,我们理一下思路,我们想要求得一个$\widetilde{q}(Z) \approx p(Z|X)$。也就是
\begin{equation}
    \widetilde{q}(Z) = \arg\max_{q(z)} \mathcal{L}(q) \Rightarrow \widetilde{q}(Z) \approx p(Z|X)
\end{equation}

\section{模型求解}
那么我们如何来求解这个问题呢?我们使用到统计物理中的一种方法,就是平均场理论(mean field theory)。也就是假设变分后验分式是一种完全可分解的分布:
\begin{equation}
    q(z) = \prod_{i=1}^M q_i(z_i)
\end{equation}

在这种分解的思想中,我们每次只考虑第j个分布,那么令$q_i(1,2,\cdots,j-1,j+1,\cdots,M)$个分布fixed。

那么很显然:
\begin{equation}
    \mathcal{L}(q) = \int_Zq(Z)\log p(X,Z)dz - \int_Zq(Z)\log q(Z)dZ
\end{equation}

我们先来分析第一项$ \int_Zq(Z)\log p(X,Z)dZ$。
\begin{equation}
    \begin{split}
        \int_Zq(Z)\log p(X,Z)dZ 
        = & \int_Z\prod_{i=1}^M q_i(z_i)\log p(X,Z)dZ \\
        = & \int_{z_j}q_j(z_j) \left[\int_{z_1}\int_{z_2}\cdots\int_{z_M} \prod_{i=1}^M q_i(z_i)\log p(X,Z)dz_1dz_2\cdots dz_M \right] dz_j \\
        = & \int_{z_j}q_j(z_j) \mathbb{E}_{\prod_{i \neq j}^Mq_i(x_i)}\left[ \log p(X,Z) \right] dz_j
    \end{split}
\end{equation}

然后我们来分析第二项$\int_Zq(Z)\log q(Z)dZ$,
\begin{equation}
    \begin{split}
        \int_Zq(Z)\log q(Z)dZ 
        = & \int_Z \prod_{i=1}^M q_i(z_i) \sum_{i=1}^M \log q_i(z_i)dZ \\
        = & \int_Z \prod_{i=1}^M q_i(z_i) \left[ \log q_1(z_1) + q_2(z_2) + \cdots + q_M(z_M)\right] dZ \\
    \end{split}
\end{equation}

这个公式的计算如何进行呢?我们抽出一项来看,就会变得非常的清晰:
\begin{equation}
    \begin{split}
        \int_Z \prod_{i=1}^M q_i(z_i) \log q_1(z_1) dZ
        = &  \int_{z_1z_2\cdots z_M} q_1q_2\cdots q_M \log q_1 dz_1dz_2 \cdots z_M \\
        = & \int_{z_1}q_1\log q_1 dz_1 \cdot \int_{z_2}q_2dz_2 \cdot \int_{z_3}q_3dz_3 \cdots \int_{z_M}q_Mdz_M \\
        = & \int_{z_1}q_1\log q_1 dz_1
    \end{split}
\end{equation}

因为,$\int_{z_2}q_2dz_2$每一项的值都是1。所以第二项可以写为:
\begin{equation}
    \sum_{i=1}^M \int_{z_i} q_i(z_i)\log q_i(z_i)  dz_i =  \int_{z_j} q_j(z_j)\log q_i(z_i) dz_j + C
\end{equation}

因为我们仅仅只关注第$j$项,其他的项都不关注。为了进一步表达计算,我们将:
\begin{equation}
    \mathbb{E}_{\prod_{i \neq j}^Mq_i(z_i)}\left[ \log p(X,Z) \right] = \log \hat{p}(X,z_j)
\end{equation}

那么(8)式可以写作:
\begin{equation}
    \int_{z_j}q_j(z_j) \log \hat{p}(X,z_j) dz_j
\end{equation}

这里的$\hat{p}(X,z_j)$表示为一个相关的函数形式,假设具体参数未知。那么(7)式将等于(13)式减(11)式:

\begin{equation}
    \int_{z_j} q_j(z_j)\log q_i(z_i) dz_j - \int_{z_j}q_j(z_j) \log \hat{p}(X,z_j) dz_j + C = -KL(q_j || \hat{p}(x,z_j)) + C
\end{equation}

$\arg\max_{q_j(z_j)}-KL(q_j || \hat{p}(x,z_j))$等价于$\arg\min_{q_j(z_j)}KL(q_j || \hat{p}(x,z_j))$。那么这个$KL(q_j || \hat{p}(x,z_j))$要如何进行优化呢?我们下一节将回归EM算法,并给出求解的过程。



%  \begin{table}[!htbp]
% \renewcommand\arraystretch{1.5}
% \footnotesize
% \centering
% \caption{观察记录结果}
%  \begin{tabular}{p{3.1cm}p{0.75cm}p{0.75cm}p{0.75cm}p{0.75cm}p{0.75cm}p{0.75cm}p{0.75cm}p{0.75cm}p{0.75cm}p{0.9cm}p{0.75cm}}
%   \hline
%   &课堂1&课堂2&课堂3&课堂4&课堂5&课堂6&课堂7&课堂8&课堂9&课堂10&总计\\ 
%     \hline
% 管理性提问(次)&16&10&10&5&5&4&3&4&3&6&66\\
% 识记性提问(次)&20&30&36&11&10&13&45&12&15&13&205\\
% 提示性提问(次)&4&2&10&6&3&7&5&4&3&3&47\\
% 重复性提问(次)	&26&8&1&4&9&7&25&15&6&7&108\\
% 创造性提问(次)&1&3&3&1&6&8&3&5&4&2&36\\
% 评价性提问(次)&8&3&0&1&6&4&2&3&2&0&29\\
% 学生回答完全正确(次)&69&53&57&27&34&35&80&41&30&29&455\\
% 学生回答不完全正确(次)&2&1&0&0&2&3&1&2&1&1&13\\
% 学生回答错误(次)&4&2&2&1&3&5&2&0&2&1&22\\
% 打断学生/自己代答(次)&1&1&1&0&1&7&0&0&0&1&12\\
% 评价缺失(次)&0&3&3&0&0&0&0&0&0&2&8\\
% 有指向的称赞或肯定(次)&0&1&3&0&5&2&4&2&1&0&18\\
% 无指向的称赞或肯定(次)&8&4&2&8&8&5&8&3&0&5&51\\
% 有指向的质疑或否定(次)&3&0&0&0&1&0&0&0&0&0&4\\
% 无指向的质疑或否定(次)&3&0&0&1&1&0&0&0&1&0&6\\
% \hline
% \end{tabular}
% \end{table}









\end{document}
