\documentclass[a4paper]{article}
\usepackage[UTF8]{ctex}
\usepackage{geometry}
\usepackage{graphicx}
\usepackage{url}
\usepackage{multirow}
\usepackage{array}
\usepackage{booktabs}
\usepackage{url}
\usepackage{enumitem}
\usepackage{graphicx}
\usepackage{float}
\usepackage{amssymb}
\usepackage{amsmath}
\usepackage{subfig}
\usepackage{longtable}
\usepackage{pifont}
\usepackage{color}

\allowdisplaybreaks

\geometry{a4paper, scale=0.78}

% \begin{figure}[H]
%     \centering
%     \includegraphics[width=.55\textwidth]{E.png}
%     \caption{矩阵与列向量的乘法}
%     \label{fig:my_label_1}
% \end{figure}

% \left\{
% \begin{array}{ll}
%       x+2x+z=2 & \\
%       3x+8y+z=12 & \\
%       4y+z=2
% \end{array}
% \right.

% \begin{enumerate}[itemindent = 1em, itemsep = 0.4pt, parsep=0.5pt, topsep = 0.5pt]

% \end{enumerate}

%\stackrel{a}{\longrightarrow}

\title{Variational Inference 01 Background}
\author{Chen Gong}
\date{30 November 2019}

\begin{document}
\maketitle

这一小节的主要目的是清楚我们为什么要使用Variational Inference,表达一下Inference到底有什么用。机器学习,我们可以从频率角度和贝叶斯角度两个角度来看,其中频率角度可以被解释为优化问题,贝叶斯角度可以被解释为积分问题。

\section{优化问题}
为什么说频率派角度的分析是一个优化问题呢?我们从回归和SVM两个例子上进行分析。我们将数据集描述为:$D = \{ (x_i,y_i) \}_{i=1}^N,x_i \in \mathbb{R}^p,y_i \in \mathbb{R}$。
\subsection{回归}
回归模型可以被我们定义为:$f(w) = w^Tx$,其中loss function被定义为:$L(w) = \sum_{i=1}^N || w^Tx_i - y_i ||^2$,优化可以表达为$\hat{w} = argmin\ L(w)$。这是个无约束优化问题。

求解的方法可以分成两种,数值解和解析解。解析解的解法为:
\begin{equation}
    \frac{\partial L(w)}{\partial w} = 0 \Rightarrow w^{\ast} = (X^TX)^{-1}X^TY
\end{equation}

其中,$X$是一个$n\times p$的矩阵。而数值解中,我们常用的是GD算法,也就是Gradient Descent,或者Stochastic Gradient descent (SGD)。

\subsection{SVM (Classification)}
SVM的模型可以被我们表述为:$f(w) = sign(w^T+b)$。loss function被我们定义为:
\begin{equation}
    \left\{
    \begin{array}{ll}
        \min\ \frac{1}{2}w^Tw & \\
        s.t. \quad y_i(w^Tx_i + b) \geq 1 & \\
    \end{array}
    \right.
\end{equation}

很显然这是一个有约束的Convex优化问题。常用的解决条件为,QP方法和Lagrange 对偶。

\subsection{EM算法}
我们的优化目标为:
\begin{equation}
    \hat{\theta} = \arg\max_{\theta}\ \log p(X|\theta)
\end{equation}

优化的迭代算法为:
\begin{equation}
    \theta^{(t+1)} = \arg\max_{\theta}\int_{z} \log p(X,Z|\theta)\cdot p(Z|X,\theta^{(t)}) dz
\end{equation}

\section{积分问题}
从贝叶斯的角度来说,这就是一个积分问题,为什么呢?我们看看Bayes公式的表达:
\begin{equation}
    p(\theta|x) = \frac{p(x|\theta)p(\theta)}{p(x)} 
\end{equation}

其中,$p(\theta|x)$称为后验公式,$p(x|\theta)$称为似然函数,$p(\theta)$称为先验分布,并且$p(x) = \int_{\theta}p(x|\theta)p(\theta)d\theta$。什么是推断呢?通俗的说就是求解后验分布$p(\theta|x)$。而$p(\theta|x)$的计算在高维空间的时候非常的复杂,我们通常不能直接精确的求得,这是就需要采用方法来求一个近似的解。而贝叶斯的方法往往需要我们解决一个贝叶斯决策的问题,也就是根据数据集$X$(N个样本)。我们用数学的语言来表述也就是,$\widetilde{X}$为新的样本,求$p(\widetilde{X}|X)$:
\begin{equation}
    \begin{split}
        p(\widetilde{X}|X) 
        = & \int_{\theta} p(\widetilde{X},\theta|X) d\theta \\
        = & \int_{\theta} p(\widetilde{X}|\theta)\cdot p(\theta|X)d\theta \\
        = & \mathbb{E}_{\theta|X} [p(\widetilde{X}|\theta)]
    \end{split}
\end{equation}

其中$p(\theta|X)$为一个后验分布,那么我们关注的重点问题就是求这个积分。
\section{Inference}
Inference的方法可以被我们分为精确推断和近似推断,近似推断可以被我们分为确定性推断和随机近似。确定性推断包括Variational Inference (VI);随机近似包括MCMC,MH,Gibbs Sampling等





























\end{document}
