\documentclass[a4paper]{article}
\usepackage[UTF8]{ctex}
\usepackage{geometry}
\usepackage{graphicx}
\usepackage{url}
\usepackage{multirow}
\usepackage{array}
\usepackage{booktabs}
\usepackage{url}
\usepackage{enumitem}
\usepackage{graphicx}
\usepackage{float}
\usepackage{amssymb}
\usepackage{amsmath}
\usepackage{subfig}
\usepackage{longtable}
\usepackage{pifont}
\usepackage{color}

\allowdisplaybreaks

\geometry{a4paper, scale=0.78}

% \begin{figure}[H]
%     \centering
%     \includegraphics[width=.55\textwidth]{E.png}
%     \caption{矩阵与列向量的乘法}
%     \label{fig:my_label_1}
% \end{figure}

% \left\{
% \begin{array}{ll}
%       x+2x+z=2 & \\
%       3x+8y+z=12 & \\
%       4y+z=2
% \end{array}
% \right.

% \begin{enumerate}[itemindent = 1em, itemsep = 0.4pt, parsep=0.5pt, topsep = 0.5pt]

% \end{enumerate}

%\stackrel{a}{\longrightarrow}

\title{Variational Inference 03 Algorithm Solution}
\author{Chen Gong}
\date{01 December 2019}

\begin{document}
\maketitle

在上一小节中,我们介绍了Mean Field Theory Variational Inference的方法。在这里我需要进一步做一些说明,{\color{red} $z_i$表示的不是一个数,而是一个数据维度的集合,它表示的不是一个维度,而是一个类似的最大团,也就是多个维度凑在一起。}在上一节中,我们得出:
\begin{equation}
    \log q_j(z_j) = \mathbb{E}_{\prod_{i \neq j}q_i(z_i)}\left[ \log p(X,Z|\theta) \right] + C
\end{equation}

并且,我们令数据集为$X = \{ x^{(i)} \}_{i=1}^N$,$Y = \{ y^{(i)} \}_{i=1}^N$。variation的核心思想是在于用一个分布$q$来近似得到$p(z|x)$。其中优化目标为,$\hat{q} = \arg\min\ KL(q||p)$。其中:
\begin{equation}
    \log p(X|\theta) = ELBO (\mathcal{L}(q)) + KL(q||p) \geq  \mathcal{L}(q)
\end{equation}

在这个求解中,我们主要想求的是$q(x)$,那么我们需要弱化$\theta$的作用。所以,我们计算的目标函数为:
\begin{equation}
    \hat{q} = \arg\min_{q} KL(q||p) = \arg\max_q \mathcal{L}(q)
\end{equation}

在上一小节中,这是我们的便于观察的表达方法,但是我们需要严格的使用我们的数学符号。

\section{数学符号规范化}
在这里我们弱化了相关参数$\theta$,也就是求解过程中,不太考虑$\theta$起到的作用。我们展示一下似然函数,
\begin{equation}
    \log p_{\theta}(X) = \log \prod_{i=1}^N p_{\theta}(x^{(i)}) = \sum_{i=1}^N \log p_{\theta}(x^{(i)})
\end{equation}

我们的目标是使每一个$x^{(i)}$最大,所以将对ELBO和$KL(p||q)$进行规范化表达:

ELBO:
\begin{equation}
    \begin{split}
        \mathbb{E}_{q(z)}\left[ \log \frac{p_{\theta}(x^{(i)},z)}{q(z)} \right] = \mathbb{E}_{q(z)}\left[ \log p_{\theta}(x^{(i)},z) \right]+ H(q(z))
    \end{split}
\end{equation}

KL:
\begin{equation}
    KL(q||p) = \int q(z)\cdot \log \frac{q(z)}{p_{\theta}(z|x^{(i)})} dz
\end{equation}

而,
\begin{equation}
    \begin{split}
        \log q_j(z_j) 
        = & \mathbb{E}_{\prod_{i \neq j} q_i(z_i)}\left[ \log p_{\theta} (x^{(i)},z) \right] + C \\
        = & \int_{q_1} \int_{q_2} \cdots \int_{q_{j-1}}\int_{q_{j+1}} \cdots \int_{q_{M}} q_1q_2\cdots q_{j-1}q_{j+1} \cdots q_M dq_1dq_2 \cdots dq_{j-1}dq_{j+1} \cdots dq_{M}  \\
    \end{split}
\end{equation}

\section{迭代算法求解}
在上一步中,我们已经将所有的符号从数据点和划分维度上进行了规范化的表达。在这一步中,我们将使用迭代算法来进行求解:
\begin{gather}
    \hat{q}_1(z_1) = \int_{q_2} \cdots \int_{q_{M}} q_2 \cdots q_M \left[ \log p_{\theta}(x^{(i)},z) \right]dq_2 \cdots dq_{M}  \\
    \hat{q}_2(z_2) = \int_{\hat{q}_1(z_1)}\int_{q_3} \cdots \int_{q_{M}} \hat{q}_1q_3 \cdots q_M \left[ \log p_{\theta}(x^{(i)},z) \right]\hat{q}_1dq_2 \cdots dq_{M}  \\
    \nonumber \vdots \\
    \hat{q}_M(z_M) = \int_{\hat{q}_1} \cdots \int_{\hat{q}_{M-1}} \hat{q}_1 \cdots \hat{q}_{M-1} \left[ \log p_{\theta}(x^{(i)},z) \right]d\hat{q}_1 \cdots d\hat{q}_{M-1}
\end{gather}

如果,我们将${q}_1,{q}_2,\cdots,{q}_M$看成一个个的坐标点,那么我们知道的坐标点越来越多,这实际上就是一种坐标上升的方法(Coordinate Ascend)。

这是一种迭代算法,那我们怎么考虑迭代的停止条件呢?我们设置当$\mathcal{L}^{(t+1)} \leq \mathcal{L}^{(t)}$时停止迭代。

\section{Mean Field Theory的存在问题}
1. 首先假设上就有问题,这个假设太强了。在假设中,我们提到,假设变分后验分式是一种完全可分解的分布。实际上,这样的适用条件挺少的。大部分时候都并不会适用。

2. Intractable。本来就是因为后验分布$p(Z|X)$的计算非常的复杂,所以我们才使用变分推断来进行计算,但是有个很不幸的消息。这个迭代的方法也非常的难以计算,并且

\begin{equation}
    \log q_j(z_j) = \mathbb{E}_{\prod_{i \neq j}q_i(z_i)}\left[ \log p(X,Z|\theta) \right] + C
\end{equation}

\noindent 的计算也非常的复杂。所以,我们需要寻找一种更加优秀的方法,比如Stein Disparency等等。Stein变分是个非常Fashion的东西,机器学习理论中非常强大的算法,我们以后会详细的分析。





















































\end{document}
