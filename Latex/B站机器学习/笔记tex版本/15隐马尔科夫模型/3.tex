\documentclass[a4paper]{article}
\usepackage[UTF8]{ctex}
\usepackage{geometry}
\usepackage{graphicx}
\usepackage{url}
\usepackage{multirow}
\usepackage{array}
\usepackage{booktabs}
\usepackage{url}
\usepackage{enumitem}
\usepackage{graphicx}
\usepackage{float}
\usepackage{amssymb}
\usepackage{amsmath}
\usepackage{subfig}
\usepackage{longtable}
\usepackage{pifont}
\usepackage{color}

\allowdisplaybreaks

\geometry{a4paper, scale=0.78}

% \begin{figure}[H]
%     \centering
%     \includegraphics[width=.55\textwidth]{E.png}
%     \caption{矩阵与列向量的乘法}
%     \label{fig:my_label_1}
% \end{figure}

% \left\{
% \begin{array}{ll}
%       x+2x+z=2 & \\
%       3x+8y+z=12 & \\
%       4y+z=2
% \end{array}
% \right.

% \begin{enumerate}[itemindent = 1em, itemsep = 0.4pt, parsep=0.5pt, topsep = 0.5pt]

% \end{enumerate}

%\stackrel{a}{\longrightarrow}

%\underbrace{}_{} %下括号

%\tableofcontents %目录,并且目录页不记录页码
% \tableofcontents
% \newpage
% \setcounter{page}{1} %new page
% \clearpage

\title{Hidden Markov Model 03 Learning}
\author{Chen Gong}
\date{09 January 2020}

\begin{document}
\maketitle
首先我们回顾一下,上一节讲的有关Evaluation的问题。Evaluation可以被我们描述为在已知模型$\lambda$的情况下,求观察序列的概率。也就是:
\begin{equation}
    P(O|\lambda) = \sum_I P(O,I|\lambda) = \sum_{i_1}\cdots\sum_{i_T} \pi_{i_1} \prod_{t=2}^T a_{i_{t-1},i_{t}} \prod_{t=1}^T b_{i_1}(o_t)
\end{equation}

此时的算法复杂度为$\mathcal{O}(N^T)$。算法的复杂度太高了,所以,就有了后来的forward和backward算法。那么就有如下定义:
\begin{equation}
    \begin{split}
        & \alpha_t(i) = P(o_1,\cdots,o_t,i_t=q_i|\lambda) \\
        & \beta_t(i) = P(o_{t+1}, \cdots, o_T|i_t=q_i,\lambda) \\
        & \alpha_T(i) = P(O,i_T=q_i) \rightarrow P(O|\lambda) = \sum_{i=1}^N \alpha_{T}(i) \\
        & \beta_1(i) = P(o_2,\cdots,o_T|i_1=q_i,\lambda) \rightarrow P(O|\lambda) = \sum_{i=1}^N \pi_i b_i(o_1)\beta_1(i)
    \end{split}
\end{equation}

而使用forward和backward算法的复杂度为$\mathcal{O}(TN^2)$。这一节,我们就要分析Learning的部分,Learning就是要在已知观测数据的情况下求参数$\lambda$,也就是:
\begin{equation}
    \lambda_{MLE} = \arg\max_{\lambda} P(O|\lambda)
\end{equation}

\section{Learning}
我们需要计算的目标是:
\begin{equation}
    \lambda_{MLE} = \arg\max_{\lambda} P(O|\lambda)
\end{equation}

又因为:
\begin{equation}
    P(O|\lambda) =  \sum_{i_1}\cdots\sum_{i_T} \pi_{i_1} \prod_{t=2}^T a_{i_{t-1},i_{t}} \prod_{t=1}^T b_{i_1}(o_t)
\end{equation}

对这个方程的$\lambda$求偏导,实在是太难算了。所以,我们考虑使用EM算法。我们先来回顾一下EM算法:
\begin{equation}
    \theta^{(t+1)} = \arg\max_\theta \int_z \log P(X,Z|\theta)\cdot P(Z|X,\theta^{(t)}) dZ
\end{equation}

而$X\rightarrow O$为观测变量;$Z\rightarrow I$为隐变量,其中$I$为离散变量;$\theta \rightarrow \lambda$为参数。那么,我们可以将公式改写为:

\begin{equation}
    \lambda^{(t+1)} = \arg\max_\lambda \sum_I \log P(O,I|\lambda)\cdot P(I|O,\lambda^{(t)}) 
\end{equation}

这里的$\lambda^{(t)}$是一个常数,而:
\begin{equation}
    P(I|O,\lambda^{(t)}) = \frac{P(I,O|\lambda^{(t)})}{P(O|\lambda^{(t)})}
\end{equation}

并且$P(O|\lambda^{(t)})$中$\lambda^{(t)}$是常数,所以这项是个定量,与$\lambda$无关,所以$\frac{P(I,O|\lambda^{(t)})}{P(O|\lambda^{(t)})} \propto P(I,O|\lambda^{(t)})$。所以,我们可以将等式(7)改写为:
\begin{equation}
    \lambda^{(t+1)} = \arg\max_\lambda \sum_I \log P(O,I|\lambda)\cdot P(I,O|\lambda^{(t)})
\end{equation}

这样做有什么目的呢?很显然这样可以把$\log P(O,I|\lambda)$和$P(I,O|\lambda^{(t)})$变成一种形式。其中,$\lambda^{(t)} = (\pi^{(t)}, \mathcal{A}^{(t)}, \mathcal{B}^{(t)})$,而$\lambda^{(t+1)} = (\pi^{(t+1)}, \mathcal{A}^{(t+1)}, \mathcal{B}^{(t+1)})$。

我们定义:
\begin{equation}
    Q(\lambda,\lambda^{(t)}) = \sum_I \log P(O,I|\lambda)\cdot P(O,I|\lambda^{(t)}) 
\end{equation}

而其中,
{\color{red}
\begin{equation}
    P(O|\lambda) =  \sum_{i_1}\cdots\sum_{i_T} \pi_{i_1} \prod_{t=2}^T a_{i_{t-1},i_{t}} \prod_{t=1}^T b_{i_1}(o_t)
\end{equation}
}

所以,
\begin{equation}
    Q(\lambda,\lambda^{(t)}) = \sum_I \left[ \left( \log \pi_{i_1} + \sum_{t=2}^T \log a_{i_{t-1},i_t} + \sum_{t=1}^T \log b_{i_t}(o_t) \right)\cdot P(O,I|\lambda^{(t)})  \right]
\end{equation}

\section{以$\pi^{(t+1)}$为例}
这小节中我们以$\pi^{(t+1)}$为例,在公式$Q(\lambda,\lambda^{(t)})$中,$\sum_{t=2}^T \log a_{i_{t-1},i_t}$与$\sum_{t=1}^T \log b_{i_t}(o_t)$与$\pi$无关,所以,
\begin{equation}
    \begin{split}
        \pi^{(t+1)} = & \arg\max_{\pi} Q(\lambda,\lambda^{(t)}) \\
        = & \arg\max_{\pi} \sum_I [\log \pi_{i_1} \cdot P(O,I|\lambda^{(t)})] \\
        = & \arg\max_{\pi} \sum_{i_1}\cdots \sum_{i_T}[\log \pi_{i_1} \cdot P(O,i_1,\cdots,i_T|\lambda^{(t)})]
    \end{split}
\end{equation}

我们观察$\{i_2,\cdots,i_T\}$就可以知道,联合概率分布求和可以得到边缘概率。所以:
\begin{equation}
    \begin{split}
        \pi^{(t+1)} = & \arg\max_{\pi} \sum_{i_1} [\log \pi_{i_1} \cdot P(O,i_1|\lambda^{(t)})] \\
        = & \arg\max_{\pi} \sum_{i=1}^N [\log \pi_{i} \cdot P(O,i_1=q_i|\lambda^{(t)})] \qquad (s.t. \ \sum_{i=1}^N \pi_i = 1) \\
    \end{split}
\end{equation}

\subsection{拉格朗日乘子法求解}
根据拉格朗日乘子法,我们可以将损失函数写完:
\begin{equation}
    \mathcal{L}(\pi,\eta) = \sum_{i=1}^N \log \pi_{i} \cdot P(O,i_1=q_i|\lambda^{(t)}) + \eta(\sum_{i=1}^N \pi_i - 1)
\end{equation}

使似然函数最大化,则是对损失函数$\mathcal{L}(\pi,\eta)$求偏导,则为:
\begin{align}
    & \frac{\mathcal{L}}{\pi_i} = \frac{1}{\pi_i} P(O,i_1=q_i|\lambda^{(t)}) + \eta = 0 \\
    & P(O,i_1=q_i|\lambda^{(t)}) + \pi_i\eta = 0 
\end{align}

又因为$\sum_{i=1}^N \pi_i = 1$,所以,我们将公式(17)进行求和,可以得到:
\begin{equation}
    \sum_{i=1}^N P(O,i_1=q_i|\lambda^{(t)}) + \pi_i\eta = 0 \Rightarrow P(O|\lambda^{(t)}) + \eta = 0
\end{equation}

所以,我们解得$\eta = -P(O|\lambda^{(t)})$,从而推出:
\begin{equation}
    \pi_i^{(t+1)} = \frac{P(O,i_1=q_i|\lambda^{(t)})}{P(O|\lambda^{(t)})}
\end{equation}

进而,我们就可以推导出$\pi^{(t+1)} = (\pi_1^{(t+1)},\pi_2^{(t+1)},\cdots,\pi_N^{(t+1)}$。而$\mathcal{A}^{(t+1)}$和$\mathcal{B}^{(t+1)}$也都是同样的求法。这就是大名鼎鼎的Baum Welch算法,实际上思路和EM算法一致。不过在Baum Welch算法诞生之前,还没有系统的出现EM算法的归纳。所以,这个作者还是很厉害的。

































\end{document}
