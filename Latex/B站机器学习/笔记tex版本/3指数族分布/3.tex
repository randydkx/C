\documentclass[a4paper]{article}
\usepackage[UTF8]{ctex}
\usepackage{geometry}
\usepackage{graphicx}
\usepackage{url}
\usepackage{multirow}
\usepackage{array}
\usepackage{booktabs}
\usepackage{url}
\usepackage{enumitem}
\usepackage{graphicx}
\usepackage{float}
\usepackage{amssymb}
\usepackage{amsmath}
\usepackage{subfig}
\usepackage{longtable}

\allowdisplaybreaks
\geometry{a4paper, scale=0.78}

\title{Exponential Family Distribution 03 Property}
\author{Chen Gong}
\date{24 October 2019}

\begin{document}

\maketitle

本小节主要介绍Exponential Distribution中对数配分函数和充分统计量,还有极大似然估计和充分统计量的关系。

指数族分布的基本形式可以表示为:
\begin{gather}
    p(x|\eta) = h(x)exp\left\{ \eta^T\varphi(x)-A(\eta) \right\} \\
    p(x|\eta) = \frac{1}{exp \{A(\eta)\}} h(x)exp\left\{ \eta^T\varphi(x)\right\}
\end{gather}
    

\section{对数配分函数和充分统计量}
现在有一个问题,那就是我们如何求得对数配分函数$exp\{ A(\eta) \}$,或者说我们可不可以简单的求得对数配分函数。于是,就可以很自然的想到,前面所提到的充分统计量$\varphi(x)$的概念。对数配分函数的目的是为了归一化,那么我们很自然的求出对数配分函数的解析表达式:
\begin{equation}
    \begin{split}
        \int p(x|\eta) dx = & 
        \int \frac{1}{exp \{A(\eta)\}} h(x)exp\left\{ \eta^T\varphi(x)\right\} dx\\
        \int p(x|\eta) dx = & \frac{\int h(x)exp\left\{ \eta^T\varphi(x)\right\} dx}{exp \{A(\eta)\}} = 1 \\
        exp \{A(\eta)\} = & \int h(x)exp\left\{ \eta^T\varphi(x)\right\} dx 
    \end{split}
\end{equation}

下一步则是在$exp \{A(\eta)\}$中对$\eta$进行求导。
\begin{equation}
    \begin{split}
        \frac{\partial exp \{A(\eta)\}}{\partial \eta} = & \nabla_\eta A(\eta)exp \{A(\eta)\}  \\
        = & \frac{\partial}{\partial \eta}\int h(x)exp\left\{ \eta^T\varphi(x)\right\} dx \\
        = & \int \frac{\partial}{\partial \eta} h(x)exp\left\{ \eta^T\varphi(x)\right\} dx \\
        = & \int h(x)exp\left\{ \eta^T\varphi(x)\right\}\varphi(x) dx \\
    \end{split}
\end{equation}

将等式的左边的$exp \{A(\eta)\} $移到等式的右边可得,
\begin{gather}
    \nabla _{\eta}A(\eta) = \int h(x)exp\left\{ \eta^T\varphi(x) - A(\eta)\right\}\varphi(x) dx \\
    \nabla _{\eta}A(\eta) = \int p(x|\eta)\varphi(x)dx \\
    \nabla _{\eta}A(\eta) = \mathbb{E}_{x \sim p(x|\eta)}[\varphi(x)]
\end{gather}

其实通过同样的方法可以证明出,
\begin{equation}
    \nabla{\eta}^2A(\eta) = Var_{x \sim p(x|\eta)}[\varphi(x)]
\end{equation}

又因为,协方差矩阵总是正定的矩阵,于是有$\nabla_{\eta}^2A(\eta)\succeq  0$。所以,由此得出$A(\eta)$是一个凸函数。并且,由$\mathbb{E}_{x \sim p(x|\eta)}[\varphi(x)]$和$Var_{x \sim p(x|\eta)}[\varphi(x)]$就可以成功的求解得到$A(\eta)$函数。那么我们做进一步思考,知道了$\mathbb{E}[x]$和$\mathbb{E}[x^2]$,我们就可以得到所有想要的信息。那么:
\begin{equation}
    \mathbb{E}[\varphi(x)]
    =
    \begin{pmatrix}
        \mathbb{E}[x] \\
        \mathbb{E}[x^2]
    \end{pmatrix}
\end{equation}

\section{极大似然估计和充分统计量}
假设有一组观察到的数据集:$D=\left\{ x_1, x_2, x_3, \cdots, x_N \right\}$,那么我们的求解目标为:
\begin{equation}
    \begin{split}
        \eta_{MLE} = & argmax \log \prod_{i=1}^N p(x_i|\eta) \\
        = & argmax \sum_{i=1}^N\log p(x_i|\eta) \\
        = & argmax \sum_{i=1}^N\log h(x_i) exp \left\{ \eta^T\varphi(x_i) - A(\eta) \right\} \\
        = & argmax \sum_{i=1}^N\log h(x_i) + \sum_{i}^N\left(\eta^T\varphi(x_i) - A(\eta)\right) \\
    \end{split}
\end{equation}
\begin{gather}
    \frac{\partial}{\partial \eta} \left\{ \sum_{i=1}^N\log h(x_i) + \sum_{i=1}^N\left(\eta^T\varphi(x_i) - A(\eta)\right) \right\} = 0 \\
    \sum_{i=1}^N\varphi(x_i) = N \cdot \nabla_{\eta}A(\eta) \\ 
    \nabla_{\eta}A(\eta) = \frac{1}{N}\sum_{i=1}^N\varphi(x_i)
\end{gather}

或者说,我们可以认为是:$\nabla_{\eta}A(\eta_{MLE}) = \frac{1}{N}\sum_{i=1}^N\varphi(x_i)$。并且,$\nabla_{\eta}A(\eta_{MLE})$是一个关于$\eta_{MLE}$的函数。那么反解,我们就可以得到$\eta_{MLE}$。所以我们要求$\eta_{MLE}$,我们只需要得到$\frac{1}{N}\sum_{i=1}^N\varphi(x_i)$即可。所以,$\varphi(x)$为一个充分统计量。

\section{总结}
在本小节中,我们使用了极大似然估计和对数配分函数来推导了,充分统计量,这将帮助我们理解Exponential Distribution的性质。 

\end{document}
