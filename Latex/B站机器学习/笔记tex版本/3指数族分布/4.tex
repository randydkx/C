\documentclass[a4paper]{article}
\usepackage[UTF8]{ctex}
\usepackage{geometry}
\usepackage{graphicx}
\usepackage{url}
\usepackage{multirow}
\usepackage{array}
\usepackage{booktabs}
\usepackage{url}
\usepackage{enumitem}
\usepackage{graphicx}
\usepackage{float}
\usepackage{amssymb}
\usepackage{amsmath}
\usepackage{subfig}
\usepackage{longtable}

\DeclareMathOperator*{\argmax}{argmax}
\DeclareMathOperator*{\argmin}{argmin}
\allowdisplaybreaks
\geometry{a4paper, scale=0.78}

\title{Exponential Family Distribution 04 Maximum Entropy}
\author{Chen Gong}
\date{26 October 2019}

\begin{document}
\maketitle
从这节开始,我们将从最大熵的角度来解析指数族分布。首先,我们需要定义一下什么是熵?所谓熵,就是用来衡量信息反映的信息量的多少的单位。这里我们首先介绍一下,什么是熵?

\section{最大熵原理}
假设$p$是一个分布,所谓信息量就是分布的对数的相反数(p是小于1的,为了使信息量的值大于0),即为$-\log p$。而熵则被我们定义为:
\begin{equation}
    \begin{split}
        \mathbb{E}_{x\sim p(x)}[-\log p(x)]= & \int_{x} -p(x)\log p(x) dx \\
        = & - \sum_{x} p(x)\log p(x)
    \end{split}
\end{equation}

而最大熵原理实际上就可以定义为等可能。这是一种确定无信息先验分布的方法,它的原理就是是所有的可能都尽可能的出现,而不会出现类似于偏见的情况。接下来,我们令
\begin{equation}
    H(x)=-\sum_x p(x)\log p(x)
\end{equation}

假设$x$是离散的,
\begin{table}[H]
    \centering
    \begin{tabular}{c|cccc}
         $x$ & 1 & 2 & $\cdots$ & $k$ \\
         \hline
         $p$ & $p_1$ & $p_2$ & $\cdots$ & $p_k$ \\ 
    \end{tabular}
    \caption{随机变量x的概率密度分布情况}
    \label{tab:my_label}
\end{table}

并且,需要满足约束条件,
\begin{equation}
    s.t. \qquad \sum_{i=1}^Np_i=1
\end{equation}

那么,总结一下上述的描述,优化问题可以写为:
\begin{equation}
    \left\{
    \begin{array}{ll}
         \mathop{\argmax} -\sum_x p(x)\log p(x) & \\
         s.t. \qquad \sum_{i=1}^Np_i=1 &
    \end{array}
    \right.
\end{equation}

可以将其改写为:
\begin{equation}
    \left\{
    \begin{array}{ll}
          \mathop{\argmin} \sum_x p(x)\log p(x) & \\
         s.t. \qquad \sum_{i=1}^Np_i=1 &
    \end{array}
    \right.
\end{equation}

实际上也就是求$\hat{p_i} = \mathop{\argmin} -H(p(x))$,其中$p=\begin{pmatrix} p_1 & p_2 & \cdots & p_k \end{pmatrix}^T$。我们使用拉格朗日乘子法来求带约束的方程的极值。定义损失函数为:
\begin{equation}
    \mathcal{L}(p,\lambda) = \sum_{i=1}^N p(x_i)\log p(x_i) + \lambda(1-\sum_{i=1}^k p_i)
\end{equation}

下面是对$\hat{p_i}$的求解过程,
\begin{equation}
    \begin{split}
        \frac{\partial \mathcal{L}}{\partial p_i} = & \log p_i + p_i \frac{1}{p_i} - \lambda = 0 \\
    \end{split}
\end{equation}

解得:
\begin{equation}
    \begin{split}
        p_i = exp(\lambda-1)
    \end{split}
\end{equation}

又因为$\lambda$是一个常数,所以$\hat{p}_i$是一个常数,那么我们可以轻易得到
\begin{equation}
    \hat{p}_1 = \hat{p}_2 = \hat{p}_3 = \cdots = \hat{p}_k = \frac{1}{k}
\end{equation}

很显然$p(x)$是一个均匀分布,\textbf{那么关于离散变量的无信息先验的最大熵分布就是均匀分布}。

\section{指数族分布的最大熵原理}

我们首先写出指数族分布的形式:
\begin{equation}
    p(x|\eta)=h(x)exp\left\{ \eta^T\varphi(x)-A(\eta) \right\}
\end{equation}

我们可以换一种形式来定义,为了方便之后的计算:
\begin{equation}
    p(x|\eta)=\frac{1}{Z(\eta)}h(x)exp\left\{ \eta^T\varphi(x) \right\}
\end{equation}

但是,我们用最大熵原理来求指数族分布的时候,还差一个很重要的东西,也就是经验约束。也就是我们的分布要满足既定的事实上基础上进行运算。那么,我们需要怎么找到这个既定事实的分布呢?假设我们有一个数据集$Data = \{x_1, x_2, x_3, \cdots, x_N\}$。那么,我们定义分布为,
\begin{equation}
    \hat{p}(X=x)=\hat{p}(x)=\frac{Count(x)}{N}
\end{equation}

那么我们可以得到一系列的统计量$\mathbb{E}_{\hat{p}}(x)$,$Var_{\hat{p}}(x)$,$\cdots$。那么假设,$f(x)$是关于任意$x$的函数向量。那么我们定义$f(x)$为:
\begin{equation}
    f(x) = 
    \begin{pmatrix}
        f_1(x) \\
        f_2(x) \\
        \vdots \\
        f_Q(x) 
    \end{pmatrix}
    \qquad
    \Delta = 
    \begin{pmatrix}
        \Delta_1 \\
        \Delta_2 \\
        \vdots \\
        \Delta_Q 
    \end{pmatrix}
\end{equation}

其中,假设$\mathbb{E}_{\hat{p}}[f(x)]=\Delta$(已知)。同样,我们将熵表达出来,
\begin{equation}
    H[p] = - \sum_x p(x)\log p(x)
\end{equation}

那么,这个优化问题,可以被我们定义为:
\begin{equation}
    \left\{
    \begin{array}{lll}
          \mathop{\argmin} \sum_x p(x)\log p(x) & \\
         s.t. \qquad \sum_{i=1}^Np_i=1 & \\
         \qquad \quad \ \ \mathbb{E}_p[f(x)]=\mathbb{E}_{\hat{p}}[f(x)]=\Delta & \\
    \end{array}
    \right.
\end{equation}

其中,我们期望在总体数据上的特征和在给定数据上的特征一致。同样,我们使用拉格朗日乘子法来求带约束的方程的极值。定义损失函数为:
\begin{equation}
    \mathcal{L}(p,\lambda_0, \lambda) = \sum_{i=1}^N p(x_i)\log p(x_i) + \lambda_0(1-\sum_{x} p)+\lambda^T(\Delta - \mathbb{E}_p[f(x)])
\end{equation}

将$\mathbb{E}_p[f(x)])$进行改写为:
\begin{equation}
    \mathcal{L}(p,\lambda_0, \lambda) = \sum_{i=1}^N p(x_i)\log p(x_i) + \lambda_0(1-\sum_{x} p)+\lambda^T(\Delta - \sum_x p(x)f(x))
\end{equation}

我们的目的是求一个$\hat{p}(x)$,那么使用求偏导的方法(关于一个给定的x,对于$p(x)$求偏导):
\begin{equation}
    \frac{\mathcal{L}(p,\lambda_0, \lambda)}{p(x)}=\left( \log p(x) + p(x)\frac{1}{p(x)} \right) - \lambda_0 + \lambda^Tf(x) = 0
\end{equation}
\begin{equation}
    \log p(x) + 1 - \lambda_0 - \lambda^Tf(x) = 0
\end{equation}
\begin{equation}
    \log p(x) = \lambda_0 - 1 + \lambda^Tf(x) 
\end{equation}
\begin{equation}
    p(x) = exp\left\{\lambda_0 - 1 + \lambda^Tf(x)\right\} 
\end{equation}

整理一下即可得到$p(x) = exp\left\{\lambda^Tf(x) - ( 1 - \lambda_0) \right\} $,那么我们可以将$\eta = \begin{pmatrix} \lambda_0 \\ \lambda  \end{pmatrix}$,$f(x)=\varphi(x)$,$(1-\lambda_0)=A(\eta)$。很显然,$p(x)$是一个指数族分布。
\textbf{那么我们可以得到一个结论,一个无先验信息先验的分布的最大熵分布是一个指数族分布。}



\end{document}
