\documentclass[a4paper]{article}
\usepackage[UTF8]{ctex}
\usepackage{geometry}
\usepackage{graphicx}
\usepackage{url}
\usepackage{multirow}
\usepackage{array}
\usepackage{booktabs}
\usepackage{url}
\usepackage{enumitem}
\usepackage{graphicx}
\usepackage{float}
\usepackage{amssymb}
\usepackage{amsmath}
\usepackage{subfig}
\usepackage{longtable}

\geometry{a4paper, scale=0.78}
\title{Linear Regression 02}
\author{Chen Gong}
\date{14 October 2019}

\begin{document}

\maketitle

数据集$D=\{(x_1, y_1), (x_2, y_2), \cdots, (x_N, y_N)\}$,其中$x_i\in\mathbb{R}^{p}$,$y_i\in\mathbb{R}$,$i=1, \ 2,\cdots,\ N$。

数据矩阵为:(这样可以保证每一行为一个数据点)

\begin{equation}
    X=(x_1, x_2, \cdots, x_N)^T=
    \begin{pmatrix}
    x_1^T \\ 
    x_2^T \\
    \vdots\\
    x_N^T \\
    \end{pmatrix} =
    \begin{pmatrix}
    x_{11} & x_{12} & \dots & x_{1p}\\
    x_{21} & x_{32} & \dots & x_{2p}\\
    \vdots & \vdots & \ddots & \vdots\\
    x_{N1} & x_{N2} & \dots & x_{Np}\\
    \end{pmatrix}_{N\times P}
\end{equation}
\begin{equation}
    Y=
    \begin{pmatrix}
    y_1 \\ 
    y_2 \\
    \vdots\\
    y_N \\
    \end{pmatrix}_{N\times 1}
\end{equation}

设拟合的函数为:$f(w)=W^T x$,根据上一节我们推导出的结果,损失函数定义为:
\begin{equation}
    L(w)=\sum_{i=1}^N||w^Tx_i-y_i||^2
\end{equation}

解出,$\hat{w} = (X^TX)^{-1}X^TY$


\section{正则化概述}
过拟合问题(over-fitting)问题是深度学习中一个很重要的问题,往往是由少量的数据拟合高维的向量所造成的。解决over-fitting的方法有很多,通常是使用这几种思路:1.增加数据量;2.特征选择/特征提取(PCA);3.增加正则项的方法。

正则项通常可以描述为Loss Function + Penalty,也就是$L(w)+\lambda P(w)$。正则化的方法通常有以下两种:
\begin{enumerate}[itemindent = 1em, itemsep = 0.4pt, parsep=0.5pt, topsep = 0.5pt]
\item Lasso,其中$P(w) = ||w||_1 = \sum_{i=1}^Nw_i$
\item Redge,岭回归,也就是$P(w)=||w||_2^2=\sum_{i=1}^Nw_i^2$
\end{enumerate}

\section{岭回归频率派角度}
Loss function可写为$ L(w)=\sum_{i=1}^N||w^Tx_i-y_i||^2 + \lambda W^TW$
\begin{align}
    J(w) = & \sum_{i=1}^N||w^Tx_i-y_i||^2 + \lambda W^TW \\
    \nonumber = & (W^TX^T - Y^T)(XW-Y)+\lambda W^TW \\
    \nonumber = & W^TX^TXW - W^TX^TY - Y^TXW - Y^TY + \lambda W^TW \\ 
    \nonumber = & W^TX^TXW - 2W^TX^TY - Y^TY + \lambda W^TW \\
    \nonumber = & W^T(X^TX + \lambda I)W - 2W^TX^TY - Y^TY 
\end{align}

我们的求解目标是$\hat{w} = argmin_w J(w)$,求解过程为:
\begin{align}
    \frac{\partial J(w)}{\partial w} = 2(X^TX + \lambda I)W - 2X^TY = 0
\end{align}

解得:
\begin{equation}
    W = (X^TX + \lambda I)^{-1}X^TY 
\end{equation}

根据以上的推导我们可以得出,首先$(X^TX + \lambda I)$一定是可逆的。因为,半正定矩阵+单位矩阵=正定矩阵。这里不需要再求伪逆了。

\section{岭回归贝叶斯派估计角度}
类似于前文提到的贝叶斯回归的角度,假设一个分布$\varepsilon \sim \mathcal{N}(0,\sigma^2)$,那么所有的观测值可看为$y = w^Tx + \varepsilon$。因为$\varepsilon \sim \mathcal{N}(0,\sigma^2)$,那么$p(y|x;w) \sim \mathcal{N}(w^Tx, \sigma^2)$。假设$w$符合一个先验分布$\mathcal{N}(0, \sigma_{0}^{2})$。于是,我们可以得到$p(w)$和$p(y|w)$的解析表达式:
\begin{equation}
    p(y|w) = \frac{1}{\sqrt{2\pi}\sigma}exp\left( -\frac{(y - w^Tx)^2}{2\sigma^2} \right)
\end{equation}
\begin{equation}
    p(w) = \frac{1}{\sqrt{2\pi}\sigma_0}exp\left( -\frac{||w||^2}{2\sigma_0^2} \right)
\end{equation}

我们的目标是求$w$的最大后验估计(MAP),也就是定义为求$\hat{w} = argmax_w p(w|y)$。由于

\begin{equation}
    p(w|y) = \frac{p(y|w)p(w)}{p(y)}
\end{equation}

但是$y$是我们的观察量,所以$p(y)$是一个常量,在求解优化问题的时候可以不考虑进来。而且,可以加入$\log$函数来简化运算,而且与计算结果无关,于是
\begin{equation}
    argmax_w p(w|y)= \log p(y|w)p(w)
\end{equation}

代入可得:
\begin{align}
    argmax_w p(w|y) = & \sum_{i=1}^{N}\log \frac{1}{\sqrt{2\pi}\sigma}exp\left( -\frac{(y_i - w^Tx_i)^2}{2\sigma^2}  \right) \frac{1}{\sqrt{2\pi}\sigma_0}exp\left( -\frac{||w||^2}{2\sigma_0^2} \right) \\
    = & \sum_{i=1}^{N}\log \frac{1}{2\pi\sigma\sigma_0}exp\left( -\frac{(y_i - w^Tx_i)^2}{2\sigma^2}  -\frac{||w||^2}{2\sigma_0^2} \right) \\
    = & \sum_{i=1}^{N} \log \frac{1}{2\pi\sigma\sigma_0} + \log exp\left( -\frac{(y_i - w^Tx_i)^2}{2\sigma^2}  -\frac{||w||^2}{2\sigma_0^2} \right) 
\end{align}

由于$\log \frac{1}{2\pi\sigma\sigma_0}$与求解无关,所以
\begin{align}
    argmax_w p(w|y) 
    = & \sum_{i=1}^{N}  \log exp\left( -\frac{(y_i - w^Tx_i)^2}{2\sigma^2}  -\frac{||w||^2}{2\sigma_0^2} \right) \\
    = & \sum_{i=1}^{N}  -\frac{(y_i - w^Tx_i)^2}{2\sigma^2}  -\frac{||w||^2}{2\sigma_0^2} \\
\end{align}

公式可以转化为:
\begin{equation}
    argmin_x p(w|y) =  \sum_{i=1}^{N} (y_i - w^Tx_i)^2  + \frac{\sigma^2}{\sigma_0^2}||w||^2
\end{equation}

然后我们惊奇的发现将$\frac{\sigma^2}{\sigma_0^2}$换成$\lambda$就又变成了和之前从频率角度看岭回归一样的结果。所以,对于上节我们得出的结论:最小二乘估计$\Longleftrightarrow$极大似然估计(噪声符合高斯分布)。那么我们的最小二乘估计中隐藏了一个假设条件,那就是噪声符合高斯分布。我们进一步补充可得,Regularized LSE可以等价为最大后验估计(MAP)其中噪声为Guassian Distribution,并且$w$的先验也为Guassian Distribution。

\end{document}
