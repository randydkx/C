\documentclass[12pt]{article}
\usepackage[a4paper, left=3.17cm, right=3.17cm, top=2.54cm, bottom=2.54cm]{geometry}
\usepackage[fontset=mac]{ctex}
\usepackage[T1]{fontenc}
\usepackage{mathptmx}
\usepackage{amsfonts}
\usepackage{amsmath,amssymb,amsthm}
\usepackage{enumerate}

\usepackage{chemformula}
\usepackage{cite}
\usepackage[colorlinks, linkcolor=black, anchorcolor=black, citecolor=black]{hyperref}
\usepackage{indentfirst}

\usepackage{graphicx}
\setlength{\parskip}{0.5em}
\title{《高性能计算引论》第一次作业}
\author{\textup{罗文水}}
\begin{document}
	
	\begin{titlepage}
		\newcommand{\HRule}{\rule{\linewidth}{0.5mm}}
		\begin{center}
			\includegraphics[width=8cm]{title}			
		\end{center}
		
		\center 
		\quad\\[1.5cm]
		\textsl{\Large \textbf{Nanjing University of Science and Technology} }\\[0.5cm] 
		\textsl{\large School of Computer Science and Engineering}\\[0.5cm] 
		\makeatletter
		\HRule \\[0.4cm]
		{ \huge \bfseries \@title}\\[0.25cm] 
		\HRule \\[1.5cm]
	\begin{minipage}{0.42\textwidth}
		\begin{flushleft}
			
			\Large{\emph{姓名:罗文水}}
			\\
			\Large{\emph{学号:918106840738}}
			\\
			\Large{\emph{班级:计科一班}}
			\\
			\Large{\emph{课程:高性能计算引论}}
			\\
			\Large{\emph{授课教师:李翔宇}}
			\\
		\end{flushleft}
	\end{minipage}
		\vspace{7em} 
		
		{\large \today}\\[2cm] 
		\vfill 
	\end{titlepage}
	
	\newpage
\section{问题一}
(1)Intel~4004 与2020年第二季度发布的Core\;i9-10900T之间的主要参数对比如表1所示:
	\begin{table}[h] 
		\centering
		\caption{Intel~4004与Inte~Core~i9-10900T部分参数对比}
		
		\begin{tabular}{|l|c|c|}\hline
			方面&Intel 4004&Core i9-10900T\\\hline
			核心数目 & 1 & 10 \\
			制造工艺  & $10 \mu m\;$ & 14$nm$\\
			主频 & 108$kHz$ & 1.9 $GHz$\\
			最高时钟频率 & 740 $kHz$  & 4.6  $GHz$\\
			支持内存大小&-&最大128~$GB$\\
			运作电压& $15V$ & - \\
			散热功耗 & - & $35W$ \\
			元件数量&2250个& - \\
			指令宽度&$8~bit$&$64~bit$\\
			地址总线宽度&$12~bit$&-\\
			运算速度 & 6万次/秒 & -\\
			外形大小 & $3mm\cdot 4mm$  & $37.5mm\cdot 37.5mm$\\ 
			使用者 & Busicom 141-PF计算器& PC/Client/Tablet\\
			造价 & 小于$\$100$& $\$439$\\
			发布时间&1971&2020年第二季度\\
			\hline
		\end{tabular}
		
		\label{table1}
	\end{table}

	(2)Core~i9具有如下Intel~4004不具备的结构:
	\begin{itemize}
		\item \textbf{多核心}。Core~i9~10900T具有10个核心,而Intel~4004只有一个核心。
		\item \textbf{多级缓存存储系统}。Core~i9~10900T处理器支持$20MB$共享一级高速缓存。
		\item \textbf{处理器显卡}。Core~i9~10900T处理器配置有英特尔超核芯显卡,显卡最大视频内存为$64GB$,提供了高分辨率与多显示器支持。
		\item \textbf{并行计算架构}。Core~i9~10900T处理器提供了超线程技术,英特尔® 超线程技术提供每个物理内核两个处理线程。高线程应用可并行完成更多工作,从而更快地完成任务。
		\item \textbf{虚拟化技术}。Core~i9~10900T处理器使用的虚拟化技术使得系统吞吐率与资源利用率显著提高。
		\item \textbf{安全性与保护机制}。Core~i9~10900T处理器提供了加密算法加速硬件支持与安全密钥中的随机过程支持。同时也提供了操作系统保护机制以及防止恶意代码的硬件保护机制。使得该处理器比Intel~4004具有更高的安全性。
		\item \textbf{硬件自身安全性保护机制}。Core~i9~10900T处理器的温度监视架构通过几项散热管理功能防止处理器封装和系统出现散热故障。片内数字温度传感器 (DTS) 检测内核的温度,散热管理功能则降低封装功耗,从而在需要时降低温度,以保持在正常操作限制以内。有效保护CPU硬件,这是Intel~4004所不具备的保护架构。
		\item \textbf{外部设备与驱动支持,总线架构也是Intel~4004所不具备的}。
	\end{itemize}
	(3)Intel40年来维持摩尔定律的技术突破有如下几项:
	\begin{enumerate}
		\item 1978年,$64kb$动态随机存储器产生,不足0.1~$cm^{2}$的硅片上集成有14万个晶体管,即超大规模集成电路初具规模。
		\item 1988年,$16M$~DRAM问世,1~$cm^{2}$大小的硅片上集成有3500万个晶体管,集成电路规模有一次扩大。
		\item 1989年,芯片制作工艺进一步发展,推出的芯片先后经历了$1\mu m$到$0.8\mu m$的迭代。进而芯片上可以容纳更多的晶体管。
	\end{enumerate}
\section{问题二}

	首先,假设消息传递时间为$t_1$,则用于计算和消息传递的总时间计算方式如下:
\begin{eqnarray}\label{px}
	T_{total}&=&T_{message}+T_{compute}\\ \notag
	&=&
	\frac{10^{6}}{p}+10^{9}\cdot(p-1)\cdot t_1 \quad sec
\end{eqnarray}

	(1)将$t_1=10^{-9},\;p=1000$带入上式得$T_{total}=1999\;sec$
	
	(2)将$t_1=10^{-3},\;p=1000$带入上式得$T_{total}=999001\cdot10^{3}\;sec$
	
	(3)根据问题一和问题二,容易知道问题一中消息传递的时间占总体时间的比重约为$
	50\%$,而问题二中消息传递时间几乎占据了总时间的$100\%$。从中可以得到结论:在多处理机环境下,消息传递的时间是不容忽视的部分,在消息传递时间非常小的情况下,也可能占据任务总体时间花费的$50\%$,与计算时间相当。而当消息传递时间较大时(本题中为0.001sec),则会严重增加任务的绝对执行时间,并且执行时间中的绝大部分都用在了消息传递上。所以,在考虑多处理机计算环境时,消息传递机制与策略是至关重要的部分。处理机之间的消息传递(计算结果之间的相互访问)若是以互联网络方式实现则需要充分考虑到节点之间访问的逻辑结构与开销,若是在共享存储器的方式下,可以访问公有的存储体从而开销降低。
\end{document}
