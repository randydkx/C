\documentclass[12pt]{article}
\usepackage[a4paper, left=3.17cm, right=3.17cm, top=2.54cm, bottom=2.54cm]{geometry}
\usepackage[fontset=mac]{ctex}
\usepackage[T1]{fontenc}
\usepackage{mathptmx}
\usepackage{amsfonts}
\usepackage{amsmath,amssymb,amsthm}
\usepackage{enumerate}

\usepackage{chemformula}
\usepackage{cite}
\usepackage[colorlinks, linkcolor=black, anchorcolor=black, citecolor=black]{hyperref}
\usepackage{indentfirst}

\usepackage{graphicx}
\setlength{\parskip}{0.5em}
\title{《高性能计算引论》第二次作业}
\author{\textup{罗文水}}
\begin{document}
	
	\begin{titlepage}
		\newcommand{\HRule}{\rule{\linewidth}{0.5mm}}
		\begin{center}
			\includegraphics[width=8cm]{../HPC_P1/title}			
		\end{center}
		
		\center 
		\quad\\[1.5cm]
		\textsl{\Large \textbf{Nanjing University of Science and Technology} }\\[0.5cm] 
		\textsl{\large School of Computer Science and Engineering}\\[0.5cm] 
		\makeatletter
		\HRule \\[0.4cm]
		{ \huge \bfseries \@title}\\[0.25cm] 
		\HRule \\[1.5cm]
	\begin{minipage}{0.42\textwidth}
		\begin{flushleft}
			
			\Large{\emph{姓名:罗文水}}
			\\
			\Large{\emph{学号:918106840738}}
			\\
			\Large{\emph{班级:计科一班}}
			\\
			\Large{\emph{课程:高性能计算引论}}
			\\
			\Large{\emph{授课教师:李翔宇}}
			\\
		\end{flushleft}
	\end{minipage}
		\vspace{7em} 
		
		{\large \today}\\[2cm] 
		\vfill 
	\end{titlepage}
	
	\newpage
\section{问题一}

\section{问题二}

	
\end{document}
